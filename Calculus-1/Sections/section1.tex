\section{Элементы теории множеств}
    \subsection{Условности и обозначения}
        \begin{definition}
            Кванторами будем называть символы, заменяющие слова в выражениях.
        \end{definition}
        \begin{comm}
            Пока что кванторы не подразумевают логические операции, мы будем использовать их только для более удобной и формальной записи.
        \end{comm}
        \begin{itemize}
            \item $\forall$ - квантор всеобщности
            \item $\exists$ - квантор существования
            \item $!$ - квантор единственности
            \item Запись $A \Rightarrow B$ обозначает, что из высказывания $A$, следует высказывание $B$. 
            \item Запись $A \lra B$ обозначает, что высказывание $A$ равносильно высказыванию $B$.
            \item Запись $a \in A$ означает, что $a$ является элементом множества $A$, отрицанием такой записи будет $a \notin A$
            \item Если $x$ - объект, а $P$ - свойство, то запись $\{x : P(x)\}$ означает класс всех объектов обладающих свойством $P$.
        \end{itemize}
        \begin{definition}
            Множество, не содержащее ни одного элемента, называется пустым и обозначается $\emptyset$.
        \end{definition}
        \begin{definition}
            Множество $A^{\prime}$ является подмножеством множества $A$, \\ если $\forall a: a\in A^{\prime}\Rightarrow a\in A$. Если $A^{\prime}$ - подмножество A, то пишут $A^{\prime}\subset A$.
        \end{definition}        
        \begin{definition}
            Для любого множества $A$ выполнено:
            \begin{enumerate}
                \item $\emptyset \subset A$.
                \item $A \subset A$.
            \end{enumerate}
        \end{definition}
        \begin{definition}
            Если $A\subset B$ и $A\ne B$, то $A$ называется собственным подмножеством множества $B$.
        \end{definition}
    \subsection{Операции над множествами}
        \begin{definition}
            Множество $C=A\cup B$ называется объединением множеств $A$ и $B$, если $\forall a: (a\in A \Rightarrow a\in C)$ и $\forall b: (b\in B\Rightarrow b\in C)$, а также $\forall c: c\in C \Rightarrow (c\in A$ или $c\in B)$.        \end{definition}
        \begin{definition}
            Множество $C=A\cap B$ называется пересечением множеств $A$ и $B$, если $\forall c: c\in C \Rightarrow (c\in A$ и $c\in B)$, а также $\forall c: (c\in A$ и $c\in B) \Rightarrow c\in C$.        \end{definition}
        \begin{definition}
            Множество $C=A\setminus B$ называется разностью множеств $A$ и $B$, если $\forall c: (c\in A$ и $c\not\in B) \Rightarrow c\in C$, а также $\forall c: c\in C \Rightarrow (c\in A$ и $c\not\in B)$
        \end{definition}
        \begin{statement}
            $A\cup(B\cap C)=(A\cup B)\cap(A\cup C)$.
        \end{statement}
        \begin{proof} 
            $a\in (A\cup(B\cap C)) \lra a\in A$ или $a\in (B\cap C) \lra a\in A$ или $(a\in B$ и $a\in C) \lra (a\in A$ или $a\in B)$ и $(a\in A$ или $a\in C)$.
        \end{proof}
        \begin{statement}
            $A\cap(B\cup C)=(A\cap B)\cup(A\cap C)$.
        \end{statement}
        \begin{proof}
            $a\in (A\cap(B\cup C)) \lra a\in A$ и $a\in (B\cap C)\lra a\in A$ и \\ $(a\in B$ или $a\in C)\lra (a\in A$ и $a\in B)$ или $(a\in A$ и $a\in C)$.
        \end{proof}
    \subsection{Декартово произведение множеств}
        \begin{definition}
            Множество $A$ называется одноэлементным, если $\exists\ a\in A$ такое, что $A\setminus\{a\} = \emptyset$.
        \end{definition}
        \begin{definition}
            Множество $A$ называется двуэлементным, если $\exists\ a\in A$ такое, что $A\setminus\{a\}$ - одноэлементное.
        \end{definition}
        \begin{definition}
            Пусть $x\in X, y\in Y$. Упорядоченной парой называется двуэлементное множество $\{x,\{x,y\}\}$, упорядоченную пару обозначают $(x,y)$.
        \end{definition}
        \begin{definition}
            Множество всех упорядоченных пар $X\times Y = \{(x,y)\}$, где \\
            $x\in X$ и $y\in Y$ называется декартовым произведением множеств $X$ и $Y$.
        \end{definition}
    \subsection{Отображения}
        \begin{definition}
            Пусть $X, Y$ - множества. Подмножество $f\subset X\times Y$ такое, что $\forall\ (x_1,y_1),\ (x_2, y_2)\in f: (y_1\ne y_2 \Rightarrow x_1\ne x_2)$ называется отображением из\\
            $X$ в $Y$, и обозначается $f: X\to Y$.
        \end{definition}
        \begin{comm}
            Запись $(x,y)\in f$ часто заменяют на $y=f(x)$.
        \end{comm}
        Далее пусть $f:X\to Y$. 
        \begin{definition}
            Множество $D_f := \{x: \exists\ (x,y) \in f\}$ называется областью определения функции $f$.
        \end{definition}
        \begin{definition}
            Множество $R_f := \{y: \exists\ (x,y) \in f\}$ называется областью значений функции $f$.
        \end{definition}
        \begin{definition}
            $f$ - инъекция $\lra \forall\ (x_1,y_1),\ (x_2,y_2)\in f: (x_1\ne x_2 \Rightarrow y_1\ne y_2)$.
        \end{definition}
        \begin{definition}
            $f$ - сюръекция $\lra Y=R_f$
        \end{definition}
        \begin{comm}
        Обычно используют определение $f$ - сюръекция $\lra \forall y\in Y \\ \exists\ x\in X: y=f(x)$. Определения, очевидно, эквивалентны.
        \end{comm}
        \begin{definition}
            $f$ - биекция $\lra f$ - инъекция и $f$ - сюръекция.
        \end{definition}
        \begin{definition}
            Пусть $f:X\to Y,\ X_1\subset X$. Множество $\{(x,y)\in f: x\in X_1\}=f|\begin{matrix}
                \null \\ X_1
            \end{matrix}$ называется ограничением $f$ на $X_1$.
        \end{definition}
        \begin{definition}
            Пусть $f:X\to Y,\ X_1\subset X$. Множество $f(X_1)=\\ \{y\in Y: \exists\ x\in X_1 : (x,y)\in f\}$ называют образом множества $X_1$.
        \end{definition}
        \begin{definition}
            Пусть $f:X\to Y,\ Y_1\subset Y$. Множество $f^{-1}(Y_1)=\\ \{x\in X: \exists\ y\in Y_1 : (x,y)\in f\}$ называют полным прообразом множества $Y_1$.
        \end{definition}
        \begin{definition}
            Пусть $f:X\to Y$. Если $\forall y\in R_f: f^{-1}(y)$ - одноэлементное множество, то подмножество $f^{-1}\subset Y\times X=\{(y,x)\}$ является отображением и называется обратным отображением к $f$. Если у отображения $f$ существует обратное отображение $f^{-1}$, то оно называется обратимым.
        \end{definition}
        \begin{statement}
            $f$ - обратимое $\lra f$ - биекция.
        \end{statement}
        \begin{comm}
        Иногда $f: X\to Y$ записывают в виде $y_x$ и называют индексацией $y$ элементами $x$.
        \end{comm}
    \subsection{Правила де Моргана}
        \begin{statement}
            $\bigcup\limits_{\alpha}(A\setminus A_{\alpha}) = A\setminus (\bigcap\limits_{\alpha}A_{\alpha})$.
        \end{statement}
        \begin{proof}
            $a\in \bigcup\limits_{\alpha}(A\setminus A_{\alpha})\lra (a\in A$ и $a\notin A_{\alpha_1})$ или $\dots$ или $(a\in A$ и $a\notin A_{\alpha_n})\lra a\in A$ и $(a\notin A_{\alpha_1}$ и $\dots$ и $a\notin A_{\alpha_n})\lra a\in A\setminus (\bigcap\limits_{\alpha}A_{\alpha})$.
        \end{proof}
        \begin{statement}
            $\bigcap\limits_{\alpha}(A\setminus A_{\alpha}) = A\setminus (\bigcup\limits_{\alpha}A_{\alpha})$.
        \end{statement}
        \begin{proof}
            $a\in \bigcap\limits_{\alpha}(A\setminus A_{\alpha})\lra (a\in A$ и $a\notin A_{\alpha_1})$ и $\dots$ и $(a\in A$ и $a\notin A_{\alpha_n})\lra a\in A$ и $(a\notin A_{\alpha_1}$ или $\dots$ или $a\notin A_{\alpha_n})\lra a\in A\setminus (\bigcup\limits_{\alpha}A_{\alpha})$.
        \end{proof}
\newpage