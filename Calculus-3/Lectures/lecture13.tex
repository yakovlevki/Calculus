\newpage
\section{Эйлеровы интегралы}
\subsection{Гамма-функция Эйлера}
Рассмотрим последовательность
\[\Gamma_n(x)=\cfrac{(n-1)!\ n^x}{\prod\limits_{k=0}^{n-1}(x+k)}\ \ \  x\neq 0, -1, -2,\dots\]
Преобразуем это выражение:
\begin{multline*}
    \prod\limits_{k=0}^{n-1}(x+k)=\prod\limits_{k=1}^{n-1}k\cdot x(1+x)(1+\frac{x}{2})\dots(1+\frac{x}{n-1})=\\
    =(n-1)!\cdot x(1+x)(1+\frac{x}{2})\dots(1+\frac{x}{n-1})
\end{multline*}
\[n^x=\frac{n^x}{(n-1)^x}\cdot \frac{(n-1)^x}{(n-2)^x}\cdot \dots\cdot \frac{2^x}{1^x}=\prod\limits_{k=1}^{n-1}\left(1+\frac{1}{k}\right)^x\]
таким образом
\begin{multline*}
    \Gamma_n(x)=\frac{1}{x(1+x)(1+\frac{x}{2})\dots(1+\frac{x}{n-1})}\cdot \frac{n^x}{(n-1)^x}\cdot \frac{(n-1)^x}{(n-2)^x}\cdot \dots \cdot \frac{2^x}{1^x}=\\
    =\frac{1}{x}\cdot \prod\limits_{k=1}^{n-1}\left(1+\frac{x}{k}\right)^{-1}\left(1+\frac{1}{k}\right)^x
\end{multline*}
Поскольку $(1+\frac{x}{k})^{-1}(1+\frac{1}{k})^x=1+O(\frac{1}{k^2})$, то существует предел
\[\lim\limits_{n\to\infty}\Gamma_n(x)=\frac{1}{x}\cdot \prod\limits_{k=1}^{\infty}\left(1+\frac{x}{k}\right)^{-1}\left(1+\frac{1}{k}\right)^x\]
Этот предел обозначают $\Gamma(x)$.
\begin{definition}
    $\Gamma(x)$ называется гамма-функцией Эйлера.
\end{definition}
\begin{theorem} (Основное функциональное соотношение)\\
    $\forall x\in D(\Gamma)$:
    \[\Gamma(x+1)=x\cdot \Gamma(x)\]
\end{theorem}
\begin{proof}
    \[\frac{\Gamma(x+1)}{\Gamma(x)}=\lim\limits_{n\to\infty}\frac{\Gamma_n(x+1)}{\Gamma_n(x)}=\lim\limits_{n\to\infty}\ \cfrac{(n-1)!\ n^{x+1}\cdot \prod\limits_{k=0}^{n-1}(x+k)}{\prod\limits_{k=0}^{n-1}(x+k+1) (n-1)!\ n^x}=\lim\limits_{n\to\infty}\frac{nx}{x+n}=x\]
\end{proof}
\begin{theorem} (Интегральное представление гамма-функции)\\
    $\forall x>0$:
    \[\Gamma(x)=\int\limits_{0}^{+\infty}t^{x-1}e^{-t}\ dt\]
\end{theorem}
\begin{proof} Для начала вычислим интеграл по частям:
    \begin{multline*}
        \int\limits_{0}^{1}t^{x-1}(1-t)^n\ dt=\frac{1}{x}\cdot \int\limits_{0}^{1}(1-t)^n\ dt^x=\frac{n}{x}\int\limits_{0}^{1}t^x(1-t)^{n-1}\ dt=\dots=\\
        =\frac{n!}{x(x+1)\dots(x+n-1)}\cdot \int\limits_{0}^{1}t^{x+n-1}\ dt=\frac{n!}{\prod\limits_{k=0}^{n}(x+k)}=\\=\frac{n!\ (n+1)^x}{\prod\limits_{k=0}^{n}(x+k)}\cdot \frac{1}{(n+1)^x}=\frac{1}{(n+1)^x}\cdot \Gamma_{n+1}(x)
    \end{multline*}
    \[\Gamma_{n+1}(x)=(n+1)^x\cdot \int\limits_{0}^{1}t^{x-1}(1-t)^n\ dt=\left|t=\frac{\tau}{n}\right|=\frac{(n+1)^x}{n^x}\cdot \int\limits_{0}^{n}\tau^{x-1}\left(1-\frac{\tau}{n}\right)^n\ d\tau\]
    Поскольку
    \[\lim\limits_{n\to\infty}\frac{(n+1)^x}{n^x}=1\]
    то существует предел
    \[\lim\limits_{n\to\infty}\ \int\limits_{0}^{n}\tau^{x-1}\left(1-\frac{\tau}{n}\right)^n\ d\tau=\Gamma(x)\]
    и существует предел
    \[\lim\limits_{n\to\infty}\ \int\limits_{0}^{n}t^{x-1} e^{-t}\ dt=\int\limits_{0}^{+\infty}t^{x-1}e^{-t}\ dt\]
    Таким образом, условия теоремы равносильны тому, что разность этих пределов равно нулю:  
    \[\lim\limits_{n\to\infty}\ \int\limits_{0}^{n}t^{x-1}\left(e^{-t}-\left(1-\frac{t}{n}\right)^n\right)\ dt=0\]
    \[e^{\alpha}\geq 1+\alpha \ \Rightarrow\ e^{-\frac{t}{n}}\geq 1-\frac{t}{n}\ \Rightarrow\ e^{-t}\geq \left(1-\frac{t}{n}\right)^n\]
    отсюда
    \[\int\limits_{0}^{n}t^{x-1}\left(e^{-t}-\left(1-\frac{t}{n}\right)^n\right)\ dt\geq 0\]
    \[e^{\alpha}\geq 1+\alpha \ \Rightarrow\ e^{\frac{t}{n}}\geq 1+\frac{t}{n}\ \Rightarrow\ e^t\geq \left(1+\frac{t}{n}\right)^n\]
    \begin{multline*}
        \int\limits_{0}^{n}t^{x-1}\left(e^{-t}-\left(1-\frac{t}{n}\right)^n\right)\ dt=\int\limits_{0}^{n}t^{x-1}e^{-t}\left(1-e^t\left(1-\frac{t}{n}\right)^n\right)\ dt\leq\\
        \leq \int\limits_{0}^{n}t^{x-1}e^{-t}\left(1-\left(1-\frac{t}{n}\right)^n\left(1+\frac{t}{n}\right)^n\right)\ dt=\int\limits_{0}^{n}t^{x-1}e^{-t}\left(1-\left(1-\frac{t^2}{n^2}\right)^n\right)\ dt\overset{(1)}{\leq}\\
        \overset{(1)}{\leq}\int\limits_{0}^{n}t^{x-1}e^{-t}\left(1-\left(1-\frac{t^2}{n}\right)\right)\ dt=\frac{1}{n}\int\limits_{0}^{n}t^{x+1}e^{-t}\ dt\leq \frac{C}{n}\to 0
    \end{multline*}
    (1): По неравентсву Бернулли:
    \[(1+x)^n\geq 1+nx\]
\end{proof}
\begin{statement}
    \[\Gamma(n+1)=n!\]
\end{statement}
\begin{proof}
    Индукция по $n$. База $n=0$:
    \[\Gamma(1)=\int\limits_{0}^{+\infty}t^{1-1}e^{-t}\ dt=\int\limits_{0}^{+\infty}e^{-t}\ dt=1\]
    Таким образом
    \[\Gamma(0+1)=\Gamma(1)=1=0!\]
    Шаг: Пусть верно для $n$, то есть
    \[\Gamma(n)=(n-1)!\]
    По основному функциональному соотношению:
    \[\Gamma(n+1)=n\cdot \Gamma(n)=n\cdot (n-1)!=n!\]
\end{proof}
% тут есть какое-то замечание про распространение на отр часть но я хз
\begin{theorem} (Формула дополнения для гамма-функции Эйлера)\\
    $\forall x\not\in \Z$:
    \[\Gamma(x)\cdot \Gamma(1-x)=\frac{\pi}{\sin{(\pi x)}}\]
\end{theorem}
\begin{proof}
    Заметим, что
    \[\Gamma(1-x)=-x\cdot \Gamma(-x)\]
    отсюда
    \begin{multline*}
        \Gamma(x)\cdot \Gamma(1-x)=\Gamma(x)(-x)\Gamma(-x)=\\
        =\frac{1}{x}\cdot \prod\limits_{k=1}^{\infty}\left(1+\frac{x}{k}\right)^{-1}\left(1+\frac{1}{k}\right)^x\cdot (-x)\cdot \prod\limits_{k=1}^{\infty}\left(1-\frac{x}{k}\right)^{-1}\left(1+\frac{1}{k}\right)^{-x}=\\
        =\frac{\pi}{\pi x}\cdot \prod\limits_{k=1}^{\infty}\left(1-\frac{x^2}{k^2}\right)^{-1}\cdot 1\overset{(1)}{=}\frac{\pi}{\sin{(\pi x)}}
    \end{multline*}
    (1): по формуле разложения синуса в бесконечное произведение:
    \[\sin{(\pi x)}=\pi x\cdot \prod\limits_{k=1}^{\infty}\left(1-\frac{x^2}{k^2}\right)\] 
\end{proof}
% тут какое-то странное следствие
\begin{example} (Интеграл Эйлера-Пуассона)\\
    Заметим, что
    \[\Gamma\left(\frac{1}{2}\right)\cdot \Gamma\left(\frac{1}{2}\right)=\frac{\pi}{\sin{\frac{\pi}{2}}}=\pi\]
    Таким образом, можем посчитать интеграл
    \[\int\limits_{-\infty}^{+\infty}e^{-x^2}\ dx=\int\limits_{0}^{+\infty}e^{-t}\cdot t^{-\frac{1}{2}}\ dt=\Gamma\left(\frac{1}{2}\right)=\sqrt{\pi}\]
\end{example}
