\begin{theorem}
    Пусть $f_n(x)\overset{A}{\rightrightarrows} f(x),\ f_n(x)$ непрерывна в точке $x_0\in A$. Тогда $f(x)$ непрерывна в точке $x_0$.
\end{theorem}
\begin{proof}
    \[\forall \epsilon>0\ \exists\ N_{\epsilon},\ \forall n>N,\ \forall x\in A: |f_n(x)-f(x)|<\epsilon\]
    \[n>N_{\epsilon},\ \forall \epsilon>0\ \exists\ \delta_{\epsilon}>0,\ \forall x\in B_{\delta_{\epsilon}}(x_0)\cap A: |f_n(x)-f_n(x_0)|<\epsilon\]
    \begin{multline*}
        |f(x)-f(x_0)|=|f(x)-f_n(x)+f_n(x)-f_n(x_0)+f_n(x_0)-f(x_0)|\leq\\
        \leq |f(x)-f_n(x)|+|f_n(x)-f_n(x_0)|+|f_n(x_0)-f(x_0)|<3\epsilon
    \end{multline*}
\end{proof}
\begin{theorem}
    Пусть $f_n(x)\overset{[a,b]}{\rightrightarrows} f(x),\ f_n\in \mathcal{C}[a,b]$. Тогда 
    \[\int\limits_{a}^{x}f_n(t)\ dt\ \rightrightarrows \int\limits_{a}^{x}f(t)\ dt\]
\end{theorem}
\begin{proof}
        \[\left|\ \int\limits_{a}^{x}f_n(x)-\int\limits_{a}^{x}f(x)\ \right|\leq \max\limits_{[a,b]}|f_n(x)-f(x)|\cdot |b-a|<\epsilon\cdot |b-a|\]
\end{proof}
\begin{theorem}
    Пусть $\{f_n(x)\}_{n=1}^{\infty}$ такая, что $f_n(x)\in \mathcal{C}^1[a,b],\ \exists\ x_0\in [a,b]:\\
    f_n(x_0)\to \alpha$. Пусть $f_n'(x)\overset{[a,b]}\rightrightarrows g(x)$. Тогда $f_n(x)\overset{[a,b]}\rightrightarrows f(x),\ f(x)\in \mathcal{C}^1[a,b]$ и\\
    $f(x)=g(x)$.
\end{theorem}
\begin{proof}
    \[\int\limits_{x_0}^{x}f_n'(x)\ dx\rightrightarrows \int\limits_{x_0}^{x}g(x)\ dx \Rightarrow f_n(x)-f_n(x_0) \rightrightarrows \int\limits_{x_0}^{x}g(x)\ dx\]
    отсюда
    \[f_n(x)\rightrightarrows \int\limits_{x_0}^{x}g(x)\ dx+\alpha=f(x)\]
\end{proof}
\begin{definition}
    Рассмотрим $\{a_n(x)\}$, определенные на $A\subset \R$. Пара последовательностей 
    \[\{\{a_n(x)\},\ \{S_n(x)=\sum_{k=1}^{n}a_k(x)\}\}\]
    называется функциональным рядом и обозначается 
    \[\sum_{n=1}^{\infty}a_n(x)\eqno{(*)}\]
    \begin{enumerate}
        \item Если $\forall x\in A$ ряд $(*)$ сходится к $S(x)$, то говорят, что ряд сходится поточечно.
        \item Если $S_n(x)\rightrightarrows S(x)$, то говорят, что сходится равномерно.
        \item Если сходится ряд
            \[\sum_{n=1}^{\infty}|a_n(x)| \eqno{(1)}\]
            то говорят, что ряд $(*)$ сходится абсолютно, если ряд $(1)$ расходится, а ряд $(*)$ сходится, то ряд $(*)$ сходится условно.
    \end{enumerate}
\end{definition}
\begin{theorem} (Критерйи Коши)\\
    \[\sum_{n=1}^{\infty}a_n(x)\overset{A}{\rightrightarrows}S(x)\]
    тогда и только тогда когда
    \[\forall \epsilon>0\ \exists\ N_{\epsilon},\ \forall m,k>N_{\epsilon},\ \forall x\in A: \left|\ \sum_{n=k}^{\infty}a_n(x)\ \right|<\epsilon\]
\end{theorem}
\begin{proof}
    Очевидно из критерия Коши для функциональных последовательностей.
\end{proof}
\begin{theorem} (Необходимое условие сходимости)\\
    Если
    \[\sum_{n=1}^{\infty}a_n(x)\overset{A}{\rightrightarrows}S(x)\]
    то $|a_n(x)|\overset{A}{\rightrightarrows}0$
\end{theorem}
\begin{proof}
    Очев.
\end{proof}
\begin{theorem}
    Пусть $\forall n: a_n(x)$ определены на $A$ и непрерывны в точке $x_0\in A$. Пусть
    \[\sum_{n=1}^{\infty}a_n(x)\overset{A}{\rightrightarrows}S(x)\]
    Тогда $S(x)$ непрерывная в точке $x_0$.
\end{theorem}
\begin{proof}
    Очев по теореме для последовательностей.
\end{proof}
\begin{theorem}
    Пусть $a_n(x)\in \mathcal{C}[a,b]$
    \[\sum_{n=1}^{\infty}a_n(x)\overset{[a,b]}{\rightrightarrows} S(x)\]
    Тогда 
    \[\sum_{n=1}^{\infty}\left(\int\limits_{a}^{x}a_n(t)\ dt\right)\rightrightarrows \int\limits_{a}^{x}S(t)\ dt=\int\limits_{a}^{x}\left(\sum_{n=1}^{\infty}a_n(x)\right)\]
\end{theorem}
\begin{proof}
    очев.
\end{proof}
\begin{theorem}
    Пусть $a_n(x)\in \mathcal{C}^1{[a,b]}$
    \[\sum_{n=1}^{\infty}a_n(x)=\alpha\]
    $x_0\in [a,b]$ и 
    \[\sum_{n=1}^{\infty}a_n'(x)\overset{[a,b]}\rightrightarrows g(x)\]
    Тогда
    \[\sum_{n=1}^{\infty}a_n(x)\overset{[a,b]}{\rightrightarrows}\ S(x)\in \mathcal{C}^1[a,b],\ S'(x)=g(x)\]
\end{theorem}
\begin{proof}
    очев.
\end{proof}
% тут мб пример с картинкой такой мемной ломаной
\begin{theorem} (Признак Вейерштрасса)\\
    Пусть $\forall n: |a_n(x)|<\alpha_n,\ \forall x\in A$. Если ряд
    \[\sum_{n=1}^{\infty}\alpha_n\]
    сходится, то
    \[\sum_{n=1}^{\infty}a_n(x)\overset{A}\rightrightarrows S(x)\]
\end{theorem}
\begin{proof}
    \[\sum_{n=k}^{n}a_n(x)\leq \sum_{n=k}^{m}|a_n(x)|\leq \sum_{n=k}^{n}\alpha_n<\epsilon\]
\end{proof}
\begin{theorem} (Признаки Абеля и Дирихле)\\
    Рассмотрим пару последовательностей $\{a_n(x)\}_{n=1}^{\infty},\ \{a_n(x)\}_{n=1}^{\infty}$ на $A$.
    \begin{itemize}
        \item[($\mathcal{A}$:)] Пусть 
        \[\sum_{n=1}^{\infty}a_n(x)\overset{A}\rightrightarrows S(x)\]
        и $\exists\ M,\ \forall x\in A,\ \forall n: |b_n(x)|<M,\ \{b_n\}_{n=1}^{\infty}$ монотонна по номеру $\forall x\in A$.
        \item[($\mathcal{D}$:)] Пусть $\forall x\in A,\ \forall N:$ 
        \[\left|\sum_{n=1}^{N}a_n(x)\right|\leq M\]
        $\exists\ M,\ \forall x\in A,\ \forall n: |b_n(x)|<M,\ \{b_n(x)\}$ монотонна $\forall x\in A,\ b_n(x)\overset{A}\rightrightarrows 0$. Тогда 
        \[\sum_{n=1}^{\infty}a_n(x)b_n(x) \overset{A}\rightrightarrows S(x)\]
    \end{itemize}
\end{theorem}
\begin{proof}
        \[\sum_{n=k}^{m}A_n(x)(b_n(x)-b_{n+1}(x))+A_m(x)\cdot b_m(x) \eqno{(*)}\]
        \begin{itemize}
            \item[($\mathcal{A}$):] $|*|\leq \epsilon\cdot |b_{k}(x)-b_{m}(x)|+|b_m(x)|<\epsilon\cdot 3M$
            \item[($\mathcal{D}$):] $|*|\leq 2M\cdot 3\epsilon$
        \end{itemize}
\end{proof}