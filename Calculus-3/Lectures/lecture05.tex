\begin{theorem}
    Пусть $f_n(x)\overset{A}{\rightrightarrows} f(x),\ f_n(x)$ непрерывна в точке $x_0\in A$. Тогда $f(x)$ непрерывна в точке $x_0$.
\end{theorem}
\begin{proof}
    По определению равномерной сходимости:
    \[\forall \epsilon>0\ \exists\ N_{\epsilon}: \forall n>N,\ \forall x\in A: |f_n(x)-f(x)|<\epsilon\]
    Возьмем $n>N_{\epsilon}$ и запишем определение непрерывности:
    \[\forall \epsilon>0\ \exists\ \delta_{\epsilon}>0,\ \forall x\in B_{\delta_{\epsilon}}(x_0)\cap A: |f_n(x)-f_n(x_0)|<\epsilon\]
    \begin{multline*}
        |f(x)-f(x_0)|=|f(x)-f_n(x)+f_n(x)-f_n(x_0)+f_n(x_0)-f(x_0)|\leq\\
        \leq |f(x)-f_n(x)|+|f_n(x)-f_n(x_0)|+|f_n(x_0)-f(x_0)|<3\epsilon
    \end{multline*}
\end{proof}
\begin{theorem}
    Пусть $f_n(x)\overset{[a,b]}{\rightrightarrows} f(x),\ f_n\in \mathcal{C}[a,b]$. Тогда 
    \[\int\limits_{a}^{x}f_n(t)\ dt\ \rightrightarrows \int\limits_{a}^{x}f(t)\ dt\]
\end{theorem}
\begin{proof}
        \[\left|\ \int\limits_{a}^{x}f_n(t)\ dt-\int\limits_{a}^{x}f(t)\ dt\ \right|\leq \max\limits_{[a,b]}|f_n(t)-f(t)|\cdot |b-a|<\epsilon\cdot |b-a|\]
\end{proof}
\begin{theorem} (Теорема о почленном дифференцировании функциональных последовательностей)\\
    Пусть $\{f_n(x)\}_{n=1}^{\infty}$ такая, что $f_n(x)\in \mathcal{C}^1[a,b],\ \exists\ x_0\in [a,b]:
    \vspace{7pt}\\
    f_n(x_0)\to \alpha$. Пусть $f_n'(x)\overset{[a,b]}\rightrightarrows g(x)$. Тогда $f_n(x)\overset{[a,b]}\rightrightarrows f(x),\ f(x)\in \mathcal{C}^1[a,b]$ и\\
    $f'(x)=g(x)$.
\end{theorem}
\begin{proof} По предыдущей теореме и формуле Ньютона-Лейбница:
    \[\int\limits_{x_0}^{x}f_n'(t)\ dt\rightrightarrows \int\limits_{x_0}^{x}g(t)\ dt\ \Rightarrow\ f_n(x)-f_n(x_0) \rightrightarrows \int\limits_{x_0}^{x}g(t)\ dt\]
    отсюда
    \[f_n(x)\rightrightarrows \int\limits_{x_0}^{x}g(t)\ dt+\alpha=f(x)\]
\end{proof}
\begin{definition}
    Рассмотрим $\{a_n(x)\}$, определенные на $A\subset \R$. Пара последовательностей 
    \[\{\{a_n(x)\},\ \{S_n(x)=\sum_{k=1}^{n}a_k(x)\}\}\]
    называется функциональным рядом и обозначается 
    \[\sum_{n=1}^{\infty}a_n(x)\eqno{(1)}\]
    \begin{enumerate}
        \item Если $\forall x\in A$ ряд $(1)$ сходится к $S(x)$, то говорят, что ряд сходится поточечно.
        \item Если $S_n(x)\rightrightarrows S(x)$, то говорят, что ряд $(1)$ сходится равномерно.
        \item Если сходится ряд
            \[\sum_{n=1}^{\infty}|a_n(x)| \eqno{(2)}\]
            то говорят, что ряд $(1)$ сходится абсолютно, если ряд $(2)$ расходится, а ряд $(1)$ сходится, то ряд $(1)$ сходится условно.
    \end{enumerate}
\end{definition}
\begin{theorem} (Критерий Коши для функциональных рядов)
    \[\sum_{n=1}^{\infty}a_n(x)\overset{A}{\rightrightarrows}S(x)\]
    тогда и только тогда, когда
    \[\forall \epsilon>0\ \exists\ N_{\epsilon}: \forall m,k>N_{\epsilon},\ \forall x\in A: \left|\ \sum_{n=k+1}^{m}a_n(x)\ \right|<\epsilon\]
\end{theorem}
\begin{proof}
    По критерию Коши для функциональных последовательностей
    \[\left|\ \sum_{n=k+1}^{m}a_n(x)\ \right|=|S_m(x)-S_k(x)|<\epsilon\]
\end{proof}
\begin{theorem} (Необходимое условие равномерной сходимости)\\
    \[\sum_{n=1}^{\infty}a_n(x)\overset{A}{\rightrightarrows}S(x)\]
    тогда и только тогда, когда $|a_n(x)|\overset{A}{\rightrightarrows}0$
\end{theorem}
\begin{proof}
    \[|a_n(x)|=|S_n-S_{n-1}|\rightrightarrows S(x)-S(x)=0\]
\end{proof}
\begin{theorem}
    Пусть $\forall n: a_n(x)$ определены на $A$ и непрерывны в точке $x_0\in A$ и
    \[\sum_{n=1}^{\infty}a_n(x)\overset{A}{\rightrightarrows}S(x)\]
    Тогда $S(x)$ непрерывна в точке $x_0$.
\end{theorem}
\begin{proof}
    $S_n(x)\overset{A}{\rightrightarrows}S(x)$ и $\forall n: a_n(x)$ непрерывна в точке $x_0$. Значит, $S_n(x)$ непрерывна в точке $x_0$ и по теореме для функциональных последовательностей, $S(x)$ непрерывна в точке $x_0$.
\end{proof}
\begin{theorem}
    Пусть $a_n(x)\in \mathcal{C}[a,b]$ и
    \[\sum_{n=1}^{\infty}a_n(x)\overset{[a,b]}{\rightrightarrows} S(x)\]
    Тогда 
    \[\sum_{n=1}^{\infty}\left(\int\limits_{a}^{x}a_n(t)\ dt\right)\rightrightarrows \int\limits_{a}^{x}S(t)\ dt\]
\end{theorem}
\begin{proof}
    \begin{multline*}
        \int\limits_{a}^{x}\left(\sum_{n=1}^{\infty}a_n(t)\right) dt=\lim\limits_{k\to \infty}\int\limits_{a}^{x}\left(\sum\limits_{n=1}^{k}a_n(t)\right)dt=\\
        =\lim\limits_{k\to \infty}\sum\limits_{n=1}^{k}\left(\int\limits_{a}^{x}a_n(t)\ dt\right)=\sum_{n=1}^{\infty}\left(\int\limits_{a}^{x}a_n(t)\ dt\right)
    \end{multline*}
\end{proof}
\begin{theorem} (Теорема о почленном дифференцировании функциональных рядов)\\
    Пусть $a_n(x)\in \mathcal{C}^1{[a,b]}$
    \[\sum_{n=1}^{\infty}a_n(x)=\alpha\]
    $x_0\in [a,b]$ и 
    \[\sum_{n=1}^{\infty}a_n'(x)\overset{[a,b]}\rightrightarrows g(x)\]
    Тогда
    \[\sum_{n=1}^{\infty}a_n(x)\overset{[a,b]}{\rightrightarrows}\ S(x)\in \mathcal{C}^1[a,b],\ S'(x)=g(x)\]
\end{theorem}
\begin{proof}
    Очевидно по аналогичной теореме для функциональной последовательности $S_n(x)$.
\end{proof}
% тут мб пример с картинкой такой мемной ломаной
\begin{theorem} (Признак Вейерштрасса равномерной сходимости)\\
    Пусть $\forall n: |a_n(x)|<\alpha_n,\ \forall x\in A$. Если ряд
    \[\sum_{n=1}^{\infty}\alpha_n\]
    сходится, то
    \[\sum_{n=1}^{\infty}a_n(x)\overset{A}\rightrightarrows S(x)\]
\end{theorem}
\begin{proof} По критерию Коши:
    \[\left|\ \sum_{n=k+1}^{m}a_n(x)\ \right|\leq \sum_{n=k+1}^{m}|a_n(x)|\leq \sum_{n=k+1}^{m}\alpha_n<\epsilon\]
\end{proof}
\begin{theorem} (Признаки Абеля и Дирихле)\\
    Рассмотрим пару последовательностей $\{a_n(x)\}_{n=1}^{\infty},\ \{b_n(x)\}_{n=1}^{\infty}$ на $A$.
    \begin{itemize}
        \item[($\mathcal{A}$:)] Пусть 
        \[\sum_{n=1}^{\infty}a_n(x)\overset{A}\rightrightarrows S(x)\]
        и $\exists\ M>0,\ \forall x\in A,\ \forall n: |b_n(x)|<M$ и $b_n(x)$ монотонна $\forall x\in A$.
        \item[($\mathcal{D}$:)] Пусть $\forall x\in A,\ \exists\ M>0,\ \forall N\in \N:$ 
        \[\left|\sum_{n=1}^{N}a_n(x)\right|\leq M\]
        и $b_n(x)$ монотонна $\forall x\in A$, причем $b_n(x)\overset{A}\rightrightarrows 0$. Тогда 
        \[\sum_{n=1}^{\infty}a_n(x)b_n(x) \overset{A}\rightrightarrows S(x)\]
    \end{itemize}
\end{theorem}
\begin{proof} Аналогично как для числовых рядов. Введем:
    \[A_p(x)=\sum_{n=k}^{p}a_n(x),\ A_{k-1}(x)=0\ \Rightarrow\ a_n(x)=A_n(x)-A_{n-1}(x)\]
    отсюда
    \begin{multline*}
        \sum_{n=k}^{m}a_n(x)b_n(x)=\sum_{n=k}^{m}(A_n(x)-A_{n-1}(x))\ b_n(x)=\\
        =\sum_{n=k}^{m}A_n(x) b_n(x)-\sum_{n=k+1}^{m}A_{n-1}(x) b_n(x)=\tab[3.5cm]\\
        \tab[3.5cm]=\sum_{n=k}^{m}A_n(x) b_n(x)-\sum_{n=k}^{m-1}A_n(x) b_{n+1}(x)=\\
        =\sum_{n=k}^{m-1}A_n(x)(b_n(x)-b_{n+1}(x))+A_m(x) b_m(x)
    \end{multline*}
        \begin{itemize}
            \item[($\mathcal{A}$):] $|*|\leq \epsilon\cdot |b_{k}(x)-b_{m}(x)|+|b_m(x)|<\epsilon\cdot 3M$
            \item[($\mathcal{D}$):] $|*|\leq 2M\cdot 3\epsilon$
        \end{itemize}
\end{proof}