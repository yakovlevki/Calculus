\begin{theorem}
    Пусть $f$ - $2\pi$ периодическая, $f\in \mathcal{C}(\R)$, везде кроме конечного числа точек (на периоде) $\exists\ f^{\prime}(x)$ и $f^{\prime}(x)\in \mathcal{R}[-\pi, \pi]$.
    Тогда
    \[\frac{a_0}{2}+\sum\limits_{n=1}^{\infty}(a_n\cdot \cos{(nx)}+b_n\cdot \sin{(nx)})\rightrightarrows f(x)\]
\end{theorem}
\begin{proof}
    Из условий теоремы существует интеграл
    \[\int\limits_{-\pi}^{\pi}(f'(x))^2\ dx\]
    то есть у $f'(x)$ есть ряд Фурье:
    \[ f'(x)\sim \frac{\alpha_0}{2}+\sum\limits_{n=1}^{\infty}(\alpha_n\cdot \cos{(nx)}+\beta_n\cdot \sin{(nx)}) = \]
    Найдём его коэффициенты (кроме $\alpha_0$):
    \[\alpha_n = \int\limits_{-\pi}^{\pi}f'(x)\cos(nx)dx = -n\int\limits_{-\pi}^{\pi}f'(x)d(\sin(nx)) = n\int\limits_{-\pi}^{\pi}f(x)\sin(nx)\ dx = nb_n\]
    Отсюда (и аналогично для $\beta_n$) имеем, что
    \[a_n=-\frac{\beta_n}{n},\ b_n=\frac{\alpha_n}{n}\]
    и, поскольку (используя неравенство: $2|ab| \leq a^2 + b^2$ )
    \[|a_n| = \frac{|\alpha_n|}{n}\leq \frac{1}{2}\left(\alpha_n^2+\frac{1}{n^2}\right),\ |b_n| = \frac{|\beta_n|}{n}\leq \frac{1}{2}\left(\beta_n^2+\frac{1}{n^2}\right)\]
    верно неравенство
    \[|a_n\cdot \cos{(nx)}+b_n\cdot \sin{(nx)}| \leqslant |a_n| + |b_n| \leqslant \frac{1}{n^2} + \frac{\alpha_n^2 + \beta_n^2}{2}\]
    Ряд с такими коэффициентами сойдётся из неравенства Бесселя, то есть изначальный ряд Фурье сходится равномерно по признаку Вейерштрасса.
\end{proof}% по первой теореме из раздела сходится именно к f
\begin{consequense}
    Если $f'(x)\in \mathcal{C}(\R)$ и Гёльдерова во всех точках, то ряд Фурье $f(x)$ можно почленно дифференцировать.
\end{consequense}
\begin{proof}
    очев.
\end{proof}
\begin{comm} (Про почленное интегрирование)\\
    Пусть
    \[f(x)\sim \frac{a_0}{2}+\sum\limits_{n=1}^{\infty}(a_n\cdot \cos{(nx)}+b_n\cdot \sin{(nx)})\]
    \[\int\limits_{0}^{x}\left(f(x)-\frac{a_0}{2}\right) dx=\sum\limits_{n=1}^{\infty}\left(\frac{a_n}{n}\cdot \sin{(nx)}-\frac{b_n}{n}\cdot \cos{(nx)}\right)+\sum\limits_{n=1}^{\infty}\frac{b_n}{n}\]
    \[\frac{|\alpha_n|}{n}\leq \frac{1}{2}\left(\alpha_n^2+\frac{1}{n^2}\right),\ \frac{|\beta_n|}{n}\leq \frac{1}{2}\left(\beta_n^2+\frac{1}{n^2}\right)\]
\end{comm}
\begin{examples}\tab % говорил что обязательные для экзамена примеры
    \begin{enumerate}
        \item Возьмем $f(x)=\frac{\pi-x}{2}$ на $(0,\pi)$ и продолжим ее периодически нечетным образом.
        \begin{multline*}
            b_n=\frac{2}{\pi}\int\limits_{0}^{\pi}\frac{\pi-x}{2}\cdot \sin{(nx)}\ dx=\\
            =\frac{2}{\pi n}\cdot \frac{\pi-x}{2}\cdot \cos{(nx)}|_0^{\pi}-\frac{1}{\pi n}\int\limits_{0}^{\pi}\cos{(nx)}\ dx=\\
            =-\frac{2}{\pi n}\cdot \left(-\frac{\pi}{2}\right)=\frac{1}{n}
        \end{multline*}
        \[f(x)=\sum\limits_{n=1}^{\infty}\frac{\sin{(nx)}}{n}\]
        \item Возьмем $f(x)=|x|$ на $[-\pi,\pi]$ и периодически продолжим.
            \[a_0=\frac{1}{\pi}\int\limits_{-\pi}^{\pi}|x|\ dx=\frac{2}{\pi}\cdot \frac{\pi^2}{2}=\pi\]
            \begin{multline*}
                a_n=\frac{2}{\pi}\int\limits_{0}^{\pi}x\cdot \cos{(nx)}\ dx=-\frac{2}{\pi n}\int\limits_{0}^{\pi}\sin{(nx)}\ dx=\\=
                \frac{2}{\pi n^2}\cdot \cos{(nx)}|_0^{\pi}=\frac{2}{\pi n^2}((-1)^n-1)=\begin{cases}
                    0, n=2k\\
                    -\frac{4}{\pi(2k+1)^2},\ n=2k+1
                \end{cases}
            \end{multline*}
            \[f(x)=\frac{\pi}{2}-\frac{4}{\pi}\cdot \sum\limits_{k=0}^{\infty}\frac{\cos{((2k+1)x)}}{(2k+1)^2}\]
            Отсюда, поскольку ряд сходится равномерно от непрерывной функции к непрерывной функции, то можно подставить 0 и получить
            \[\sum\limits_{k=0}^{\infty}\frac{1}{(2k+1)^2}=\frac{\pi^2}{8}\]
            отсюда можно получить
            \[\sum\limits_{n=1}^{\infty}\frac{1}{n^2}=\frac{\pi^2}{6}\]
    \end{enumerate}
\end{examples}
\subsection{Ядро Фейера}

\begin{definition}
    Пусть 
    \[f(x)\sim \frac{a_0}{2}+\sum\limits_{n=1}^{\infty}(a_n\cdot \cos{(nx)}+b_n\cdot \sin{(nx)})\]
    Обозначим:
    \[F_N(x) = \frac{a_0}{2}+\sum\limits_{n=1}^{N}(a_n\cdot \cos{(nx)}+b_n\cdot \sin{(nx)})\] 
    Тогда:
    \[\Phi_N(x)=\frac{1}{N+1}\cdot \sum\limits_{n=0}^{N}F_n(x)\]
    называются суммами Фейера. %это типо средние арифметические частичных сумм ряда фурье
    \[\Psi_N(x)=\frac{1}{N+1}\cdot \sum\limits_{n=0}^{N}D_n(x)\]
    называются ядрами Фейера. %это типо средние арифмет сумм фейера
\end{definition}
\begin{lemma} (Свойства ядер Фейера)
    \begin{enumerate}
        \item $\psi_N(x)$ - $2\pi$ периодическая, непрерывная, четная.
        \item 
        \[\Psi_N(x)=\frac{\sin^2{(\frac{N+1}{2}\cdot x)}}{2(N+1)\cdot \sin^2(\frac{x}{2})}\]
        \item
        \[\frac{1}{\pi}\int\limits_{-\pi}^{\pi}\Psi_N(x)\ dx=1\]
        \item $\forall \delta>0: $
        \[\int\limits_{\delta}^{\pi}\Psi_N(x)\ dx\to 0\]
        при $N\to \infty$.
        \item $\Psi_N(x)\geq 0$
    \end{enumerate}
\end{lemma}
\begin{proof}\tab
    \begin{itemize}
        \item[2.] \[\sum\limits_{n=0}^N D_n(x) = \frac{1}{2\sin{(\frac{x}{2})}}\sum\limits_{n=0}^N \sin{(nx+\frac{1}{2}x)} = \frac{1}{2\sin{(\frac{x}{2})}} \cdot \frac{\sin^2{(\frac{N+1}{2}x)}}{\sin{(\frac{x}{2})}}\]
        \item[4.] \[0\leq \int\limits_{\delta}^{\pi}\frac{\sin^2(\frac{N+1}{2}\cdot x)}{2(N+1)\cdot \sin^2({\frac{x}{2}})}\ dx\leq \frac{1}{2(N+1)}\cdot \int\limits_{\delta}^{\pi}\frac{dx}{\sin^2({\frac{x}{2}})}\leq \frac{C}{2(N+1)}\]
    \end{itemize}
\end{proof}