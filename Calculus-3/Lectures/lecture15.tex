\subsection{Бета-функция Эйлера}
\begin{definition}
    \[B(\alpha, \beta)=\int\limits_{0}^{1}x^{\alpha-1}(1-x)^{\beta-1}\ dx,\ \alpha,\ \beta>0\]
\end{definition}
\begin{statement}
    $B(\alpha, \beta)=B(\beta, \alpha)$
\end{statement}
\begin{proof}
    \[B(\alpha, \beta)=\int\limits_{0}^{1}x^{\alpha-1}(1-x)^{\beta-1}\ dx=\left|t=1-x\right|=\int\limits_{0}^{1}t^{\beta-1}(1-t)^{\alpha-1}\ dt=B(\beta, \alpha)\]
\end{proof}
\begin{statement}
    \[B(\alpha+1, \beta)=\frac{\alpha}{\alpha+\beta}\cdot B(\alpha, \beta)\]
\end{statement} 
\begin{proof}
    \begin{multline*}
        B(\alpha+1, \beta)=\int\limits_{0}^{1}x^{\alpha}(1-x)^{\beta-1}\ dx=-\frac{1}{\beta}\int\limits_{0}^{1}x^{\alpha}\ d((1-x)^{\beta})=\\
        =\frac{\alpha}{\beta}\int\limits_{0}^{1}x^{\alpha-1}(1-x)^{\beta}\ dx=\frac{\alpha}{\beta}\int\limits_{0}^{1}(x^{\alpha-1}(1-x)^{\beta-1})(1-x)\ dx=\tab[2cm]\\
        \tab[2cm]=\frac{\alpha}{\beta}\int\limits_{0}^{1}x^{\alpha-1}(1-x)^{\beta-1}\ dx-\frac{\alpha}{\beta}\int\limits_{0}^{1}x^{\alpha}(1-x)^{\beta-1}\ dx=\\
        =\frac{\alpha}{\beta}B(\alpha, \beta)-\frac{\alpha}{\beta}B(\alpha+1, \beta)
    \end{multline*}
    % дальше произошел очев, нужно дописать
\end{proof}
\begin{statement} % поговаривают что очевидна какая-то симметричность 
    \[B(\alpha, \beta)=\int\limits_{0}^{+\infty}\frac{x^{\alpha-1}}{(1+x)^{\alpha+\beta}}\ dx\]
\end{statement}
\begin{proof}
    \begin{multline*}
        \int\limits_{0}^{1}t^{\beta-1}(1-t)^{\alpha-1}\ dt=\left|t=\frac{1}{1+x}\right|=\\
        =\int\limits_{+\infty}^{0}\frac{x^{\alpha-1}}{(1+x)^{\beta-1}(1+x)^{\alpha-1}}\cdot \left(-\frac{1}{(1+x)^2}\ dx\right)=\\
        =\int\limits_{0}^{+\infty}\frac{x^{\alpha-1}}{(1+x)^{\alpha+\beta}}\ dx
    \end{multline*}
\end{proof}
\begin{theorem} (Связь между гамма и бета функциями)\\
    \[B(\alpha,\ \beta)=\frac{\Gamma(\alpha)\cdot \Gamma(\beta)}{\Gamma(\alpha+\beta)},\ \alpha,\beta>0\]
\end{theorem}
\begin{proof} Сначала рассмотрим $\alpha,\ \beta>1$
    \begin{multline*}
        \Gamma(\alpha+\beta)\cdot B(\alpha, \beta)=\int\limits_{0}^{+\infty}\frac{x^{\alpha-1}}{(1+x)^{\alpha+\beta}}\cdot \Gamma(\alpha+\beta)\ dx=\\
        =\int\limits_{0}^{+\infty}\frac{x^{\alpha-1}}{(1+x)^{\alpha+\beta}}\cdot \left(\int\limits_{0}^{+\infty}t^{\alpha+\beta-1}e^{-t}\ dt\right) dx=\left|t=(1+x)y\right|=\\
        =\int\limits_{0}^{+\infty}\frac{x^{\alpha-1}}{(1+x)^{\alpha+\beta}}\cdot \left(\int\limits_{0}^{+\infty}(1+x)^{\alpha+\beta}\cdot y^{\alpha+\beta-1}\cdot e^{-y-yx}\ dy\right)dx=\\
        =\int\limits_{0}^{+\infty}\left(\int\limits_{0}^{+\infty}x^{\alpha-1}\cdot y^{\alpha+\beta-1}\cdot e^{-y}\cdot e^{-yx}\ dy\right)dx=\\
        =\int\limits_{0}^{+\infty}y^{\alpha+\beta-1}\cdot e^{-y}\cdot \left(\int\limits_{0}^{+\infty}x^{\alpha-1}\cdot e^{-yx}\ dx\right)dy=\\
        =\int\limits_{0}^{+\infty}y^{\beta-1}\cdot e^{-y}\cdot \left(\int\limits_{0}^{+\infty}(xy)^{\alpha-1}\cdot e^{-yx}\ d(yx)\right)dy=\\=\Gamma(\alpha)\cdot \Gamma(\beta)
    \end{multline*}% тут из дифференциала x появился и по теореме для интегр с парам
    Далее, используя рекуррентную формулу, распространяем на случай\\
    $\alpha, \beta>0$.
\end{proof}
\begin{comm}
    Интегралы вида
    \[\int\limits_{0}^{\frac{\pi}{2}}\sin^{\theta}x\cdot \cos^{\kappa}x\ dx\]
    сводятся к бета-функции с помощью замены $t=\sin^2{x}$.
\end{comm}
\newpage
\section{Ряды Фурье}
\begin{definition}Выражения вида
    \[\frac{a_0}{2}+\sum\limits_{n=1}^{N}(a_n\cdot \cos{(nx)}+b_n\cdot \sin{(nx)})\]
    называются тригонометрическими полиномами.
\end{definition}
\begin{definition} Система функций 
    \[\{1,\ \{\cos{(nx)}\}_{n=1}^{\infty},\ \{\sin{(nx)}\}_{n=1}^{\infty}\}\]
    называется тригонометрической системой.
\end{definition}
\begin{definition} Ряды вида
    \[\frac{a_0}{2}+\sum\limits_{n=1}^{\infty}(a_n\cdot \cos{(nx)}+b_n\cdot \sin{(nx)})\]
    называются тригонометрическими рядами.
\end{definition}
\begin{statement} (Свойства тригонометрической системы)\\
    \[\int\limits_{-\pi}^{\pi}1\ dx=2\pi\]
    \[\forall n\geq 1\ \int\limits_{-\pi}^{\pi}\sin^2{(nx)}\ dx=\int\limits_{-\pi}^{\pi}\cos^2{(nx)}\ dx=\pi\]
    \[\forall m,n\geq 1,\ m\neq n\ \int\limits_{-\pi}^{\pi}\sin{(nx)}\cdot \sin{(mx)}\ dx=\int\limits_{-\pi}^{\pi}\cos{(nx)}\cdot \cos{(mx)}\ dx=0\]
    \[\forall n\geq 0,m\geq 1\ \int\limits_{-\pi}^{\pi}\cos{(nx)}\cdot \sin{(mx)}\ dx=0\]
\end{statement}
\begin{proof}
    Очев.
\end{proof}
\begin{consequense}
    Тригонометрическая система ортогональна относительно канонического скалярного произведения для функций (интеграл).
\end{consequense}
\begin{theorem}
    Пусть 
    \[\frac{a_0}{2}+\sum\limits_{n=1}^{\infty}(a_n\cdot \cos{(nx)}+b_n\cdot \sin{(nx)})\overset{[-\pi,\pi]}{\rightrightarrows} f(x)\]
    Тогда
    \[a_n=\frac{1}{\pi}\int\limits_{-\pi}^{\pi}f(x)\cos{(nx)}\ dx,\ \ \ \ b_n=\frac{1}{\pi}\int\limits_{-\pi}^{\pi}f(x)\sin{(nx)}\ dx\]
\end{theorem}
\begin{proof}
    $\forall k$
    \[\frac{a_0}{2}\sin{(kx)}+\sum\limits_{n=1}^{\infty}(a_n\cdot \sin{(kx)}\cos{(nx)}+b_n\cdot \sin{kx}\sin{nx})\overset{[-\pi,\pi]}{\rightrightarrows} f(x)\sin{(kx)}\]
    По теореме о почленном интегрировании:
    \[\frac{1}{\pi}\int\limits_{-\pi}^{\pi}f(x)\sin{kx}\ dx=\frac{1}{\pi}b_k\cdot \int\limits_{-\pi}^{\pi}\sin^2{(kx)}\ dx\]
    Аналогично $a_k$.
\end{proof}
