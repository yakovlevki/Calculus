\section{Ряды}
\subsection{Определение ряда и простейшие свойства}
\begin{definition}
    Пара последовательностей
    \[a_n,\ S_n=\sum_{k=1}^{n}a_k\]
    называется числовым рядом и обозначается
    \[\sum_{n=1}^{\infty}a_n\]
    $a_n$ называется общим членом ряда, $S_n$ называется частичной суммой ряда.
\end{definition}
\begin{definition}
    Если существует предел
    \[\lim\limits_{n\to \infty}S_n=S\]
    то 
    \[\sum_{n=1}^{\infty}a_n\]
    называется сходящимся, а $S$ суммой ряда.
\end{definition}
\begin{definition}
    Рассмотрим ряд
    \[\sum_{n=1}^{\infty}a_n\]
    тогда ряд 
    \[r_k=\sum_{n=k}^{\infty}a_n\]
    называется остаточным рядом.
\end{definition}
\begin{theorem}(Критерий Коши сходимости ряда)\\
    Ряд
    \[\sum_{n=1}^{\infty}a_n\]
    сходится тогда и только тогда, когда
    \[\forall \epsilon>0\ \exists\ N_{\epsilon}\in \N,\ \forall k,m>N_{\epsilon}: \left|\sum_{n=k}^{m}a_n\right|<\epsilon\]
\end{theorem}
\begin{proof}
    Очевидно по критерию коши для последовательности
    \[\sum_{n=k}^{m}a_n=S_m-S_{k-1}\]
\end{proof}
\begin{theorem}
    Пусть даны ряды
    \[\sum_{n=1}^{\infty}a_n,\ \sum_{n=1}^{\infty}b_n\]
    и они сходятся, тогда $\forall c, \alpha, \beta\in \R$ ряды
    \begin{enumerate}
        \item
        \[\sum_{n=1}^{\infty}c\cdot a_n=c\cdot \sum_{n=1}^{\infty}a_n\]
        \item 
        \[\sum_{n=1}^{\infty}(\alpha\cdot a_n+\beta\cdot b_n)=\alpha\cdot \sum_{n=1}^{\infty}a_n+\beta\cdot \sum_{n=1}^{\infty}b_n\]
    \end{enumerate}
    также сходятся.
\end{theorem}
\begin{proof}
    Очев.
\end{proof}
\begin{theorem} (Необходимое условие сходимости ряда)\\
    Если
    \[\sum_{n=1}^{\infty}a_n\]
    сходится, то $a_n\to 0$.
\end{theorem}
\begin{proof}
    \[\lim\limits_{n\to\infty}a_n=\lim\limits_{n\to\infty}(S_n-S_{n-1})=0\]
\end{proof}
\subsection{Знакопостоянные ряды}
В этом разделе считаем, что $\forall n\in \N: a_n\geq 0$ или $a_n\leq 0$.
\begin{theorem}
    Рассмотрим ряд
    \[\sum_{n=1}^{\infty}a_n,\ a_n\geq 0\]
    Если последовательность $S_n$ ограничена, то этот ряд сходится.
\end{theorem}
\begin{proof}
    По теореме Вейерштрасса для последовательности $S_n$.
\end{proof}
\begin{theorem} (Признак сравнения)\\
    Пусть даны два ряда
    \[\sum_{n=1}^{\infty}a_n\ (a_n\geq 0),\ \sum_{n=1}^{\infty}b_n\ (b_n\geq 0)\]
    и $0\leq a_n\leq b_n$. Тогда если ряд
    \[\sum_{n=1}^{\infty}b_n\]
    сходится, то ряд
    \[\sum_{n=1}^{\infty}a_n\]
    сходится. Если ряд
    \[\sum_{n=1}^{\infty}a_n\]
    расходится, то ряд
    \[\sum_{n=1}^{\infty}b_n\]
    расходится.
\end{theorem}
\begin{proof} Очевидно из неравенства
    \[\sum_{n=1}^{N}a_n\leq \sum_{n=1}^{N}b_n\]
\end{proof}
\begin{theorem} (Признак сравнения в предельной форме)\\
    Пусть даны два ряда
    \[\sum_{n=1}^{\infty}a_n\ (a_n\geq 0),\ \sum_{n=1}^{\infty}b_n\ (b_n > 0)\]
    Если существует предел
    \[\lim\limits_{n\to\infty}\frac{a_n}{b_n}=c>0\]
    то ряды сходятся или расходятся одновременно.
\end{theorem}
\begin{proof}
    Выберем $\epsilon>0$ такой, что
    \[0<c-\epsilon<\frac{a_n}{b_n}<c+\epsilon\]
    \[(c-\epsilon)\cdot b_n<a_n<(c+\epsilon)\cdot b_n\]
    по признаку сравнения получаем утверждение теоремы.
\end{proof}
\begin{example}
    Ряд
    \[\sum_{n=1}^{\infty}\frac{1}{n}\]
    расходится по Критерию Коши.
\end{example}
\begin{example}
    \[\sum_{n=1}^{\infty}\frac{1}{n^{\alpha}}\]
    при $\alpha<1$ расходится по признаку сравнения с рядом из предыдущего примера.
\end{example}
\begin{example} Ряд
    \[\sum_{n=1}^{\infty}\frac{1}{(n+1)n}=\sum_{n=1}^{\infty}\left(\frac{1}{n}-\frac{1}{n+1}\right)=1\]
    сходится.
\end{example}
\begin{example} Ряд
    \[\sum_{n=1}^{\infty}\frac{1}{(n+1)^2}\]
    сходится по признаку сравнения с рядом из предыдущего примера.
\end{example}
\begin{exercise} Доказать, что при $\alpha>1$ ряд
    \[\sum_{n=1}^{\infty}\frac{1}{n^{\alpha}}\]
    сходится.
\end{exercise}