\begin{theorem} (Признак Коши)\\
    Пусть дан знакоположительный ряд 
    \[\sum_{n=1}^{\infty}a_n \eqno{(*)}\]
    \begin{enumerate}
        \item Если $\forall n>N: \sqrt[n]{a_n}\leq q<1$, то ряд ($*$) сходится.
        \item Если $\exists\ \{n_k\}_{k=1}^{\infty}$, что $\sqrt[n]{a_{n_k}}\geq 1$, то ряд ($*$) расходится.
    \end{enumerate}
\end{theorem}
\begin{proof} \tab
    \begin{enumerate}
        \item $\sqrt[n]{a_n}\leq q<1 \Rightarrow a_n\leq q^n \Rightarrow$ ряд ($*$) сходится по признаку сравнения с рядом геометрической прогрессии.
        \item $\sqrt[n]{a_{n_k}}\geq 1 \Rightarrow a_n\not\to 0 \Rightarrow$ ряд ($*$) расходится.
    \end{enumerate}
\end{proof}
\begin{consequense} (Признак Коши в предельной форме)
    \begin{enumerate}
        \item Если 
        \[\uplim\limits_{n\to \infty}\sqrt[n]{a_n}=q<1\] 
        то ряд ($*$) сходится
        \item Если 
        \[\uplim\limits_{n\to \infty}\sqrt[n]{a_n}=q>1\]
        то ряд ($*$) расходится
        \item Если
        \[\lim\limits_{n\to\infty}\sqrt[n]{a_n}=q=1\]
        то ряд $(*)$ может как сходиться так и расходиться.
    \end{enumerate}
\end{consequense}
\begin{proof}\tab
    \begin{enumerate}
    \item $q<1$, значит, начиная с некоторого номера, выполнено: $\sqrt[n]{a_n}<Q<1$ следовательно, по утверждению теоремы ряд $(*)$ сходится.
    \item $q>1$, значит, начиная с некоторого номера, выполнено: $\sqrt[n]{a_n}>1$ следовательно, по утверждению теоремы ряд $(*)$ расходится.
    \item $q=1$, приведем пример рядов
        \begin{enumerate}
            \item 
            \[\sqrt[n]{\frac{1}{n}} \to 1\]
            и ряд
            \[\sum_{n=1}^{\infty}\frac{1}{n}\]
            расходится.
            \item 
            \[\sqrt[n]{\frac{1}{n^2}} \to 1\]
            и ряд
            \[\sum_{n=1}^{\infty}\frac{1}{n^2}\]
            сходится.
        \end{enumerate}
    \end{enumerate}
\end{proof}
\begin{theorem} (Признак Д'Аламбера)\\
    Пусть дан знакоположительный ряд
    \[\sum\limits_{n=1}^{\infty}a_n \eqno{(*)}\]
    \begin{enumerate}
        \item Если $\forall n>N: \frac{a_{n+1}}{a_n}\leq q<1$, то ряд $(*)$ сходится.
        \item Если $\exists\ \{n_k\}_{k=1}^{\infty}: \frac{a_{n_k + 1}}{a_{n_k}}\geq 1$, то ряд $(*)$ расходится.
    \end{enumerate}
\end{theorem}
\begin{proof}\tab
    \begin{enumerate}
        \item $\forall n>N: a_{n+1}\leq a_n\cdot q$, значит
        \[a_{N+1}\leq a_N\cdot q,\ \ a_{N+2}\leq a_{N+1}\cdot q\leq a_N\cdot q^2,\ \ \dots\]
        таким образом, $\forall k\in \N$:
        \[a_{N+k}\leq a_N\cdot q^k\]
        сложив неравенства, получим
        \[\sum\limits_{k=0}^{\infty}a_{N+k}\leq a_N\cdot \sum\limits_{k=0}^{\infty}q^k\]
        $q<1$,\ значит ряд $(*)$ сходится по признаку сравнения с геометрической прогрессией.
        \item $\exists\ \{n_k\}_{k=1}^{\infty}: \frac{a_{n_k + 1}}{a_{n_k}}\geq 1 \Rightarrow a_n\not\to 0 \Rightarrow$ ряд ($*$) расходится.
    \end{enumerate}
\end{proof}
\begin{consequense} (Признак Д’Аламбера в предельной форме)\\
    Пусть существует предел
    \[\lim\limits_{n\to\infty} \frac{a_{n+1}}{a_n}=q\]
    \begin{enumerate}
        \item Если $q<1$, то ряд $(*)$ сходится.
        \item Если $q>1$, то ряд $(*)$ расходится.
        \item Если то ряд $(*)$ может как сходиться так и расходиться.
    \end{enumerate}
\end{consequense}
\begin{proof} \tab
    \begin{enumerate}
        \item $q<1$, значит, начиная с некоторого номера, выполнено: $\frac{a_{n+1}}{a_n}<Q<1$ следовательно, по утверждению теоремы ряд $(*)$ сходится.
        \item $q>1$, значит, начиная с некоторого номера, выполнено: $\frac{a_{n+1}}{a_n}>1$ следовательно, по утверждению теоремы ряд $(*)$ расходится.
        \item $q=1$, приведем пример рядов
        \begin{enumerate}
            \item 
            \[\cfrac{\frac{1}{n+1}}{\frac{1}{n}}=\frac{n}{n+1} \to 1\]
            и ряд
            \[\sum_{n=1}^{\infty}\frac{1}{n}\]
            расходится.
            \item 
            \[\cfrac{\frac{1}{(n+1)^2}}{\frac{1}{n^2}}=\frac{n^2}{(n+1)^2} \to 1\]
            и ряд
            \[\sum_{n=1}^{\infty}\frac{1}{n^2}\]
            сходится.
        \end{enumerate}
    \end{enumerate}
\end{proof}
\begin{theorem} (Интегральный признак) \\
    Пусть $f(x)$ определена на $[1,+\infty)$, монотонно убывает и неотрицательна (монотонно возрастает и неположительна). Тогда ряд и интеграл
    \[\sum_{n=1}^{\infty}f(n),\ \int\limits_{1}^{\infty}f(x)\ dx\]
    сходятся или расходятся одновременно.
\end{theorem}
\begin{proof}
    $f(x)$ монотонно убывает, значит $\forall k\in \N$ и $x\in [k, k+1]: f(k)\geq f(x)\geq f(k+1)$. Проинтегрируем неравенство на этом отрезке:
    \[f(k)\geq \int\limits_{k}^{k+1}f(x)\ dx\geq f(k+1)\]
    теперь для каждого $k = 1, \dots, N$ сложим полученные неравенства:
    \[\sum_{k=1}^{N}f(k)\geq \int\limits_{1}^{N+1}f(x)\ dx\geq \sum_{k=2}^{N+1}f(k)\]
    отсюда получаем утверждение теоремы.
\end{proof}
\begin{example}(Степенно-логарифмическая шкала)
    \begin{enumerate}
        \item 
    \[\int\limits_{1}^{\infty}\frac{dx}{x^{\alpha}}\ \text{при}\ 
    \begin{cases}
        \alpha>1 - \text{сходится}\\
        \alpha\leq 1 - \text{расходится}
    \end{cases}\]
    значит
    \[\sum_{n=1}^{\infty}\frac{1}{n^{\alpha}}\ \text{при}\ 
    \begin{cases}
        \alpha>1 - \text{сходится}\\
        \alpha\leq 1 - \text{расходится}
    \end{cases}\]
    \item
    \[\int\limits_{2}^{\infty}\frac{dx}{x\cdot \ln^{\beta}{x}}\ \text{при}\ 
    \begin{cases}
        \beta>1 - \text{сходится}\\
        \beta\leq 1 - \text{расходится}
    \end{cases}\]
    значит
    \[\sum_{n=2}^{\infty}\frac{1}{n\cdot \ln^{\beta}{n}}\ \text{при}\ 
    \begin{cases}
        \beta>1 - \text{сходится}\\
        \beta\leq 1 - \text{расходится}
    \end{cases}\]
    \item
    \[\int\limits_{10}^{\infty}\frac{dx}{x\cdot \ln{x}\cdot \ln^{\gamma}{(\ln{x})}}\ \text{при}\ 
    \begin{cases}
        \gamma>1 - \text{сходится}\\
        \gamma\leq 1 - \text{расходится}
    \end{cases}\]
    значит
    \[\sum_{n=10}^{\infty}\frac{1}{n\cdot \ln{n}\cdot \ln^{\gamma}{(\ln{n})}}\ \text{при}\ 
    \begin{cases}
        \gamma>1 - \text{сходится}\\
        \gamma\leq 1 - \text{расходится}
    \end{cases}\]
    \end{enumerate}
    шкалу можно продолжать дальше.
\end{example}
\begin{theorem} (Схема Куммера)\\
    Рассмотрим знакоположительный ряд
    \[\sum_{n=1}^{\infty}a_n \eqno{(*)}\]
    \begin{enumerate}
        \item Если $\forall n\geq N$ существует последовательность $\{c_n\}_{n=N}^{\infty},\ c_n>0$ и существует $\alpha>0$ такие, что
        \[c_n\cdot \frac{a_n}{a_{n+1}}-c_{n+1}\geq \alpha\]
        то ряд ($*$) сходится.
        \item Если ряд
    \[\sum_{n=1}^{\infty}\frac{1}{c_n}\]
    расходится и 
    \[c_n\cdot \frac{a_n}{a_{n+1}}-c_{n+1}\leq 0\]
    то ряд ($*$) расходится.
    \end{enumerate}
\end{theorem}
\begin{proof}\tab
    \begin{enumerate}
        \item 
        Рассмотрим неравенства для $n=N,\ N+1,\ \dots,\ N+k-1$:
        \[\begin{cases}
            c_N\cdot a_N-c_{N+1}\cdot a_{N+1}\geq \alpha\cdot a_{N+1},\\
            \tab[3.5cm] \vdots\\
            c_{N+k-1}\cdot a_{N+k-1}-c_{N+k}\cdot a_{N+k}\geq \alpha\cdot a_{N+k}.
        \end{cases}\]
        сложив все неравенства, получим
        \[c_N\cdot a_N-c_{N+k}\cdot a_{N+k}\geq \alpha\cdot \sum_{m=1}^{k}a_{N+m}\]
        поскольку $c_{N+k}\cdot a_{N+k}>0$, то
        \[c_N\cdot a_N\geq \alpha\cdot \sum_{m=1}^{k}a_{N+m}\]
        значит
        \[\sum_{m=1}^{k}a_{N+m}\leq \frac{c_N\cdot a_N}{\alpha}\]
        отсюда получаем, что начиная с какого-то номера, частичные суммы ряда ограничены. Значит ряд, поскольку он знакоположительный, сходится.
        \item 
        \[c_n\cdot a_n-c_{n+1}\cdot a_{n+1}\leq 0\]
        \[\frac{c_n}{c_{n+1}}\leq \frac{a_{n+1}}{a_n} \Leftrightarrow \cfrac{\frac{1}{c_{n+1}}}{\frac{1}{c_n}}\leq \frac{a_{n+1}}{a_n}\]
        Рассмотрим неравенства для $n=N,\ N+1,\ \dots,\ N+k-1$:
        \[\begin{cases}
            \cfrac{\frac{1}{c_{N+1}}}{\frac{1}{c_N}}\leq \frac{a_{N+1}}{a_N}\\
            \tab[1.3cm] \vdots\\
            \cfrac{\frac{1}{c_{N+k}}}{\frac{1}{c_{N+k-1}}}\leq \frac{a_{N+k}}{a_{N+k-1}}
        \end{cases}\]
        перемножив все неравенства, получим
        \[\frac{\frac{1}{c_{N+k}}}{\frac{1}{c_N}}\leq \frac{a_{N+k}}{a_N}\]
        \[\frac{1}{c_{N+k}}\leq \frac{1}{a_N\cdot c_N}\cdot a_{N+k}\]
        отсюда, по признаку сравнения, ряд
        \[\sum_{k=1}^{\infty}a_{N+k}\]
        расходится, значит расходится и ряд $(*)$.
    \end{enumerate}
\end{proof}
\begin{examples}\tab
        Рассмотрим ряд
        \[\sum_{n=1}^{\infty}a_n \eqno(*)\]
        \begin{enumerate}
        \item (Признак Д'Аламбера)\\
        Возьмем $c_n=1$:
        \[1\cdot \frac{a_n}{a_{n+1}}-1\geq \alpha\]
        значит если
        \[\frac{a_n}{a_{n+1}}\geq 1+\alpha\]
        то ряд ($*$) сходится. Если
        \[1\cdot \frac{a_n}{a_{n+1}}-1\leq 0\]
        значит если
        \[\frac{a_n}{a_{n+1}}\leq 1\]
        то ряд ($*$) расходится. Запишем в предельной форме:\\
        Если 
        \[\exists\ \lim\limits_{n\to\infty}\frac{a_n}{a_{n+1}}=q>1\]
        то ряд ($*$) сходится. Если
        \[\exists\ \lim\limits_{n\to\infty}\frac{a_n}{a_{n+1}}=q<1\]
        то ряд ($*$) расходится.
        \item (Признак Раабе)\\
        Возьмем $c_n=n$:
        \[n\cdot \frac{a_n}{a_{n+1}}-(n+1)\geq \alpha\]
        Значит если
        \[n\cdot (\frac{a_n}{a_{n+1}}-1)\geq 1+\alpha\]
        то ряд ($*$) сходится. Теперь
        \[n\cdot \frac{a_n}{a_{n+1}}-n-1\leq 0\]
        Значит если
        \[n\cdot (\frac{a_n}{a_{n+1}}-1)\leq 1\]
        то ряд ($*$) расходится.
        Запишем в предельной форме:\\
        Если
        \[\exists\ \lim\limits_{n\to\infty}n\cdot(\frac{a_n}{a_{n+1}}-1)=q>1\]
        то ряд ($*$) сходится. Если
        \[\exists\ \lim\limits_{n\to\infty}n\cdot(\frac{a_n}{a_{n+1}}-1)=q<1\]
        то ряд ($*$) расходится.
        \item (Признак Бертрана, без доказательства, знать формулировку)\\
        Возьмем $c_n=n\cdot \ln(n)$
        Если
        \[\left(\ln{n}\cdot\left(n\left(\frac{a_n}{a_{n+1}}\right)-1\right)-1\right)\geq 1+\alpha\]
        то ряд $(*)$ сходится. Если
        \[\left(\ln{n}\cdot\left(n\left(\frac{a_n}{a_{n+1}}\right)-1\right)-1\right)\leq 1\]
        то ряд $(*)$ расходится. Запишем в предельной форме:\\
        Если 
        \[\exists\ \lim\limits_{n\to\infty}\left(\ln{n}\cdot\left(n\left(\frac{a_n}{a_{n+1}}\right)-1\right)-1\right)=q>1\]
        то ряд ($*$) сходится. Если
        \[\exists\ \lim\limits_{n\to\infty}\left(\ln{n}\cdot\left(n\left(\frac{a_n}{a_{n+1}}\right)-1\right)-1\right)=q<1\]
        то ряд ($*$) расходится.
        \item (Признак Гаусса, без доказательства, знать формулировку)\\
        Выводится из признака Бертрана.\\
        \[\frac{a_n}{a_{n+1}}=\lambda+\frac{\mu}{n}+\frac{\theta_n}{n^{1+\epsilon}}\]
        где $\theta_n$ - ограниченная последовательность, $\epsilon>0$ - произвольное. Тогда при
        \begin{enumerate}
            \item $\lambda>1 \Rightarrow (*)$ сходится, $\lambda<1 \Rightarrow (*)$ - расходится.
            \item $\lambda=1: \mu>1 \Rightarrow (*)$ сходится, $\mu<1 \Rightarrow (*)$ расходится.
            \item $\lambda=1,\ \mu=1 \Rightarrow (*)$ расходится.  
        \end{enumerate}
    \end{enumerate}
\end{examples}
