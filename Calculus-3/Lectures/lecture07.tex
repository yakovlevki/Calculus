\subsection{Ряды Тейлора}
\begin{definition}
    Пусть $f(x) \in C^\infty(B(x_0))$. Ряд вида
    \[\sum_{n=0}^{\infty}\frac{f^{(n)}(x_0)}{n!}(x-x_0)^n\]
    называется рядом Тейлора функции $f(x)$ с центром ряда $x_0$.
\end{definition}
\begin{comm}
    Далее везде центр разложения в 0.
\end{comm}
\begin{theorem}
    Если в некоторой окрестности нуля выполнено равенство
    \[\sum_{n=0}^{\infty}a_n x^n = f(x)\]
    то
    \[a_n = \frac{f^{(n)}(0)}{n!}\]
\end{theorem}
\begin{proof}
    Подставим $x=0$, тогда $a_0=f(0)$. Продифференцируем и подставим $x=0$, тогда $a_1=f'(0)$. Снова продифференцируем и подставим $x=0$, тогда $a_2=\frac{f''(x)}{2}$. Продолжая выполнять эту операцию, для каждого n получим $a_n=\frac{f^{(n)}(x)}{n!}$.
\end{proof}
\begin{theorem}
    Пусть $f(x) \in C^\infty(-a, a)$ и $|f^{(n)}(x)| \leq A^n, \ A > 0$.\\
    Тогда
    \[\sum_{n=0}^{\infty}\frac{f^{(n)}(x_0)}{n!}(x-x_0)^n = f(x) \text{ на } (-a, a)\]
\end{theorem}
\begin{proof}
    По формуле Тейлора с остаточным членом в форме Лагранжа $\forall x \in (-a, a)$:
    \[f(x) = \sum_{n=0}^{N}\frac{f^{(n)}(0)}{n!}x^n + \frac{f^{(N+1)}(\xi)}{(N+1)!}x^{N+1}\]
    Тогда
    \[\left| f(x) - \sum_{n=0}^{N}\frac{f^{(n)}(0)}{n!}x^n \right| = \left| \frac{f^{(N+1)}(\xi)}{(N+1)!}x^{N+1}\right| \leq \frac{A^{N+1}\cdot a^{N+1}}{(N+1)!} \rightarrow 0\]
\end{proof}
\begin{example}
    Рассмотрим функцию 
    \[f(x) = \begin{cases}
        e^{-\frac{1}{x^2}}, \ x \neq 0\\
        0, \ x = 0
    \end{cases}\]
    Несложно проверить, что $f(x) \in C^\infty(B(0)),\ f^{(n)}(0) = 0$.\\
    Значит,
    \[\sum_{n=0}^{\infty}\frac{f^{(n)}(0)}{n!}x^n \equiv 0 \neq f(x)\] 
\end{example}
\begin{example} (Контрпример к обратному утверждению второй теоремы Абеля)\\
    Рассмотрим ряд
    \[\sum_{n=0}^{\infty}(-1)^n x^n\]
    Этот ряд сходится на $(-1, 1)$, причём
    \[\sum_{n=0}^{\infty}(-1)^n x^n = \frac{1}{1+x} \underset{x\rightarrow1-0}{\longrightarrow} \frac{1}{2}\]
    но ряд
    \[\sum_{n=0}^{\infty}(-1)^n\]
    не сходится.
\end{example}
\begin{lemma}
    Если $a_n \rightarrow A$ при $n \rightarrow \infty$, то
    \[\frac{1}{N}\sum_{n=1}^{N}a_n \rightarrow A\]
\end{lemma}
\begin{proof}
    \begin{multline*}
        \left|\frac{1}{N}\sum_{n=1}^{N}a_n - A\right| = \frac{1}{N}\left|\sum\limits_{n=1}^{N}a_n - AN\right|\leq \\
        \leq \frac{1}{N}\left|\sum\limits_{n=1}^{N_1}a_n - AN_1\right|+\frac{1}{N}\sum\limits_{n=N_1+1}^{N}|a_n - A| < \epsilon\cdot \frac{N-N_1}{N} + \epsilon < 2\epsilon
    \end{multline*}
\end{proof}
\begin{theorem}(Теорема Таубера)\\
    Пусть степенной ряд
    \[\sum_{n=0}^{\infty}a_n x^n\]
    сходится на $(-1, 1)$, существует
    \[\lim \limits_{x\rightarrow 1-0} \sum_{n=0}^{\infty}a_n x^n = A\]
    и пусть $a_n = \overline{\overline{o}}(\frac{1}{n}),\ n \to \infty$.\\
    Тогда существует
    \[\sum_{n=0}^{\infty}a_n = A\]
\end{theorem}
\begin{proof}
    Обозначим $\forall x \in (0, 1)$
    \[N_x = \left[\frac{1}{1-x}\right]\]
    и введём функцию
    \[\phi(x) = \sum_{n=0}^{N_x}a_n - \sum_{n=0}^{\infty}a_n x^n\]
    Если $\phi(x) \rightarrow 0$ при $x \rightarrow 1-0$, то теорема доказана.
    \[\phi(x) = \sum_{n=0}^{N_x}a_n(1-x^n) + \sum_{n=N+1}^{\infty}a_n x^n\]
    \begin{multline*}
        \left|\sum_{n=0}^{N_x}a_n(1-x^n)\right| = \left|\sum_{n=0}^{N_x}a_n(1-x)(1+x+...x+x^{n-1})\right| \leq \\
        \leq \frac{1}{\frac{1}{1-x}}\sum_{n=0}^{N_x}|a_n|n \leq \frac{1}{N_x}\sum_{n=0}^{N_x}|a_nn| \rightarrow 0,\ x \rightarrow 1-0
    \end{multline*}
    \begin{multline*}
        \left|\sum_{n=N_x+1}^{\infty}a_n x^n\right| = \left|\sum_{n=N_x+1}^{\infty}\frac{a_n n}{n} x^n \right| < \frac{1}{N_x + 1}\sum_{n=N_x+1}^{\infty}\epsilon x^n <\\
        <\frac{\epsilon}{N_x + 1}\sum_{n=0}^{\infty}x^n = \frac{\epsilon}{N_x + 1} \cdot \frac{1}{1-x} < \epsilon
    \end{multline*}
    Отсюда $|\phi(x)| < 2\epsilon \ \forall n > N_x$.
\end{proof}
\subsection{Бесконечные произведения}
\begin{definition}
    Рассмотрим $\{u_n\}_{n=1}^{\infty}, \{\Pi_n = \prod \limits_{k=1}^{n}u_k\}_{n=1}^{\infty}$.\\
    Пара последовательностей $\{\{u_n\}_{n=1}^{\infty}, \{\Pi_n\}_{n=1}^{\infty}\}$ называется бесконечным произведением чисел $\{u_n\}$, обозначается 
    \[\prod \limits_{n=1}^{\infty}u_n\]
    $u_n$ - общий член, $\Pi_n$ - частичное произведение.
\end{definition}
\begin{definition}
    $\prod \limits_{n=1}^{\infty}u_n$ называется сходящимся, если $\exists \lim \limits_{n\rightarrow \infty} \Pi_n = a \neq 0$.\\
    $\prod \limits_{n=1}^{\infty}u_n$ называется расходящимся, если $\nexists \lim \limits_{n\rightarrow \infty} \Pi_n$ или $\lim \limits_{n\rightarrow \infty} \Pi_n = 0$.
    Если $\exists n: u_n = 0$, то $\prod \limits_{n=1}^{\infty}u_n$ называется расходящимся к 0.
\end{definition}
\begin{theorem}(Необходимое условие сходимости)
    Если $\prod \limits_{n=1}^{\infty}u_n$ сходится, то $u_n \rightarrow 1$.
\end{theorem}
\begin{proof}
    \[\lim \limits_{n \rightarrow \infty} u_n = \lim \limits_{n \rightarrow \infty} \frac{\Pi_n}{\Pi_{n-1}} = 1\]
\end{proof}
