\subsection{Степенные ряды}
\begin{definition}
    Ряды вида 
    \[\sum_{n=0}^{\infty}a_n(x-x_0)^n\]
    называются степенными рядами. $a_n$ - коэффициэнты степенного ряда, $x_0$ - центр разложения.
\end{definition}
\begin{comm}
    В центре разложения ряд сходится.
\end{comm}
\begin{comm}
    Сдвиг $x-x_0\mapsto x$ не ограничивает общность ряда, поэтому будем рассматривать ряды вида
    \[\sum_{n=0}^{\infty}a_n x^n\]
\end{comm}
\begin{theorem} (Первая теорема Абеля)\\
    Рассмотрим стеепнной ряд 
    \[\sum_{n=0}^{\infty}a_nx^n \eqno{(*)}\]
    \begin{enumerate}
        \item Если существует $x_1\in \R$, что ($*$) сходится в точке $x_1$, то для любых $x$, таких, что $|x|<|x_1|$ ряд $(*)$ сходится.
        \item Если существует $x_2\in \R$, что ряд $(*)$ расходится в точке $x_2$, то для любых $x$, таких что $|x|>|x_2|$ ряд $(*)$ расходится.
    \end{enumerate}
\end{theorem}
\begin{proof}\tab
    \begin{enumerate}
        \item Пусть $x: |x|<|x_1|$.
        \[\sum_{n=0}^{\infty}|a_n x^n|=\sum_{n=0}^{\infty}\left(|a_n x_1^n|\cdot \left|\frac{x}{x_1}\right|^n\right)\overset{(1)}{\leq} M\cdot \sum_{n=0}^{\infty}\left|\frac{x}{x_1}\right|^n\]
        (1): Поскольку сходится ряд
        \[\sum\limits_{n=1}^{\infty}a_n x_1^n\]
        то $a_n x_1^n\to 0 \Rightarrow |a_n x_1^n|<M$\\
        Таким образом, ряд сходится по признаку сравнения с рядом геометрической прогрессии.
        \item Теперь пусть $x: |x|>|x_2|$. От противного: пусть есть точка $y: |y|>|x_2|$, в которой ряд сходится. Тогда, по первой части теоремы, ряд сходится во всех точках $x: |x|<|y|$, а значит и в $x_2$ - противоречие.
    \end{enumerate}
\end{proof}
\begin{comm}
    Рассмотрим ряд
    \[\sum_{n=0}^{\infty}a_n x^n \eqno{(*)}\]
    Пусть $A\subset \R$ - все точки, в которых ряд ($*$) сходится, $A\ne \emptyset$, так как $\{0\}\in A$.\\
    Пусть $B\subset \R$ - все точки, в которых ряд ($*$) расходится и пусть $B\ne \emptyset$.\\
    Тогда из первой теоремы Абеля следует, что $A=\{0\}$ или $\exists\ a>0: (-a,a)\subset A$, а также $\exists\ b>0: (-\infty, -b)\cap(b,+\infty)\subset B$.
\end{comm}
\begin{statement}
    $a=b$
\end{statement}
\begin{proof}
    Предположим что это не так, тогда $a<b$ или $b<a$. Пусть для определенности $a<b$, по аксиоме полноты $\exists\ c\in\R: a\leq c\leq b$. Значит $c$ не лежит ни в $A$ ни в $B$, а такого быть не может, противоречие.
\end{proof}
\begin{definition}
    Число $a=b$ называется радиусом сходимости степенного ряда и обозначается $R$. Формально полагают, что если $R=0$, то $A=\{0\}$, если $B=\emptyset$, то $R:=+\infty$.
\end{definition}
\begin{theorem} (Формула Коши-Адамара)\\
    Пусть дан ряд
    \[\sum_{n=0}^{\infty}a_n x^n\]
    Тогда
    \[\frac{1}{R}=\uplim\limits_{n\to \infty}\sqrt[n]{|a_n|}\]
    Если предел равен нулю, то $R=+\infty$.\\
    Если предел равен бесконечности, то $R=0$.
\end{theorem}
\begin{proof}
    Применим признак Коши к ряду
    \[\uplim\limits_{n\to \infty}\sqrt[n]{|a_n x^n|}=|x|\cdot \frac{1}{R}=\begin{cases}
        l<1\ -\ \text{сходится, и}\ |x|<R,\\
        l>1\ -\ \text{расходится, и}\ |x|>R  
    \end{cases}\]
\end{proof}
\begin{theorem}
    Рассмотрим ряд
    \[\sum_{n=0}^{\infty}a_n x^n=S(x) \eqno{(*)}\]
    и пусть $R>0$.
    \begin{enumerate}
        \item $\forall \epsilon\in (0,R)$ на отрезке $[-R+\epsilon, R-\epsilon]$ ряд равномерно сходится.
        \item $S(x)\in \mathcal{C}(-R,R)$.
        \item $\forall x\in (-R,R)$:
        \[\sum_{n=0}^{\infty}\frac{a_n}{n+1} x^{n+1}=\int\limits_{0}^{x}S(t)\ dt\]
        \item $S(x)\in \mathcal{C}^{\infty}(-R,R)$.
    \end{enumerate}
\end{theorem}
\begin{proof}\tab
    \begin{enumerate}
        \item $|a_n x^n|\leq|a_n|\cdot (R-\epsilon)^n$ по признаку Вейерштрасса.
        \item из пункта 1: $\forall \epsilon>0: S(x)\in \mathcal{C}[-R+\epsilon,R-\epsilon] \Rightarrow S(x)\in \mathcal{C}(-R,R)$.
        \item Поскольку ряд $(*)$ равномерно сходится на любом отрезке внутри $(-R,R)$, то можно почленно интегрировать
        \[\int\limits_{0}^{x}S(t)\ dt=\int\limits_{0}^{x}\left(\sum_{n=0}^{\infty}a_n t^n\right)dt=\sum_{n=0}^{\infty}\left(\int\limits_{0}^{x}a_n t^n\ dt\right)=\sum_{n=0}^{\infty}\frac{a_n}{n+1}x^{n+1}\]
        \item Заметим, что
        \[\lim\limits_{n\to\infty}\sqrt[n]{n^{\alpha}}=1\]
        значит $\forall \alpha\in \R$ ряд
        \[\sum_{n=1}^{\infty}a_n n^\alpha x^n\]
        имеет, по формуле Коши-Адамара, тот же радиус сходимости $R$. Тогда по теореме о почленном дифференцировании функциональных рядов, получим утверждение пункта.
    \end{enumerate}
\end{proof}
\begin{theorem} (Вторая теорема Абеля)\\
    Рассмотрим ряд 
    \[\sum_{n=0}^{\infty}a_n x^n=S(x),\ R>0\]
    Пусть 
    \[\sum_{n=0}^{\infty}a_n R^n\]
    сходится. Тогда $S(x)\in \mathcal{C}[0,R]$ то есть
    \[\exists\ \lim\limits_{x\to R-0}\left(\sum_{n=0}^{\infty}a_n x^n\right)=S(R)\]
\end{theorem}
\begin{proof}
    \[\sum_{n=0}^{\infty}a_n x^n=\sum_{n=0}^{\infty}a_n R^n\cdot \left(\frac{x}{R}\right)^n\]
    а этот ряд, по признаку Абеля, равномерно сходится на $[0,R]$.
\end{proof}