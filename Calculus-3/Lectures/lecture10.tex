\begin{theorem}
    Пусть $\phi(x),\ \psi(x)\in \mathcal{C}[a,b],\ f(x,y)\in \mathcal{C}(G)$. Тогда
    \[F(y)=\int\limits_{\phi(y)}^{\psi(y)}f(x,y)\ dx\in \mathcal{C}[a,b]\]
\end{theorem}
\begin{proof}
    \begin{multline*}
        F(y+\Delta y)-F(y)|=\Bigg|\ \int\limits_{\phi(y+\Delta y)}^{\psi(y+\Delta y)}f(x,y+\Delta y)\ dx-\\-
        \int\limits_{\phi(y)}^{\psi(y)}f(x,y+\Delta y)\ dx+\int\limits_{\phi(y)}^{\psi(y)}f(x,y+\Delta y)\ dx\ -\int\limits_{\phi(y)}^{\psi(y)}f(x,y)\ dx\ \Bigg|\leq\\
        \leq \Bigg|\int\limits_{\phi(y)}^{\psi(y+\Delta y)}f(x,y+\Delta y)\ dx\ \Bigg|+\Bigg|\ \int\limits_{\phi(y+\Delta y)}^{\psi(y)}f(x,y+\Delta y)\ dx\ \Bigg|+\\
        +\int\limits_{\phi(y)}^{\psi(y)}|f(x,y+\Delta y)-f(x,y)|\ dx\leq M\cdot \epsilon+M\cdot \epsilon+\epsilon\cdot \max\limits_{y}|\psi(y)-\phi(y)|
    \end{multline*}
    
\end{proof}
\begin{theorem}
    Пусть $\phi(y),\ \psi(y)\in \mathcal{C}^1[a,b],\ f(x,y)\in \mathcal{C}(G),\ \exists\ f'_y(x,y)\in \mathcal{C}(G)$. Тогда $F(y)\in \mathcal{C}^1(G)$ и
    \[F'(y)=\left(\ \int\limits_{\phi(y)}^{\psi(y)}f(x,y)\ dx\right)'=\int\limits_{\phi(y)}^{\psi(y)}f'_y(x,y)\ dx+f(\psi(y),y)\cdot \psi'(y)-f(\phi(y),y)\cdot \phi'(y)\]
\end{theorem}
\begin{proof}
    \begin{multline*}
        \frac{1}{\Delta y}(F(y+\Delta y)-F(y))=\\
        =\frac{1}{\Delta y}\Bigg(\ \int\limits_{\phi(y+\Delta y)}^{\psi(y+\Delta y)}f(x,y+\Delta y)\ dx-\int\limits_{\phi(y)}^{\psi(y)}f(x,y+\Delta y)\ dx+\\
        + \int\limits_{\phi(y)}^{\psi(y)}f(x,y+\Delta y)\ dx-\int\limits_{\phi(y)}^{\psi(y)}f(x,y)\ dx\ \Bigg)=\frac{1}{\Delta y}\int\limits_{\phi(y)}^{\psi(y+\Delta y)}f(x,y+\Delta y)\ dx-\\
        -\frac{1}{\Delta y}\int\limits_{\phi(y)}^{\psi(y+\Delta y)}f(x,y+\Delta y)\ dx +\frac{1}{\Delta y} \int\limits_{\phi(y)}^{\psi(y)}(f(x,y+\Delta y)-f(x,y))\ dx=\\
        =\frac{1}{\Delta y}(\psi(y+\Delta y)-\psi(y))\cdot f(\psi(y+\theta_1\Delta y), y+\Delta y)-\\
        -\frac{1}{\Delta y}(\phi(y+\Delta y)-\phi(y))f(\phi(y+\theta_2\Delta y), y+\Delta y)+\int\limits_{\phi(y)}^{\psi(y)}f'_y(x,y+\theta_3\Delta y)\ dx
    \end{multline*}
    Значит существует предел
    \[\lim\limits_{\Delta y\to 0}\frac{F(y+\Delta y)-F(y)}{\Delta y}=\psi'(y)f(\psi(y),y)-\phi'(y)f(\phi(y),y)+\int\limits_{\phi(y)}^{\psi(y)}f'_y(x,y)\ dx\]
\end{proof}
\begin{theorem}
    Пусть $f(x,y)\in \mathcal{C}([a,b]\times [c,d])$. Тогда 
    \[\int\limits_{c}^{d}\left(\int\limits_{a}^{b}f(x,y)\ dx\right) dy=\int\limits_{a}^{b}\left(\int\limits_{c}^{d}f(x,y)\ dy\right) dx\]
\end{theorem}
\begin{proof}
    Рассмотрим
    \[F(t)=\int\limits_{c}^{t}\left(\int\limits_{a}^{b}f(x,y)\ dx\right) dy,\ G(t)=\int\limits_{a}^{b}\left(\int\limits_{c}^{t}f(x,y)\ dy\right)dx\]
    Заметим, что $F(C)=G(C)=0$.
    \[F'(t)=\int\limits_{a}^{b}f(x,y)\ dx,\ G'(t)=\int\limits_{a}^{b}f(x,t)\ dx\]
    Значит $F(t)=G(t)$, в частности для $t=d$.
\end{proof}
\section{Несобственные интегралы с параметром}
% коментарий по то кто такой омега
\begin{definition}
    Рассмотрим Несобственный интеграл 
    \[\int\limits_{a}^{\omega}f(x,y)\ dx\]
    Пусть на $Y\subset \R$:
    \[\int\limits_{a}^{\omega}f(x,y)\ dx=F(y)\]
    Если $\forall \epsilon>0\ \exists\ A\in [a,\omega): \forall a'>A,\ \forall y\in Y:$
    \[\left|\int\limits_{a}^{a'}f(x,y)\ dx-F(y)\right|<\epsilon\]
    то
    \[\int\limits_{a}^{\omega}f(x,y)\ dx\overset{Y}{\rightrightarrows} F(y)\]
\end{definition}
Далее, без ограничения общности будем писать $\omega=+\infty$.
\begin{theorem} (Критерий Коши)\\
    \[\int\limits_{a}^{a'}f(x,y)\ dx\overset{Y}{\rightrightarrows} F(t)\]
    тогда и только тогда, когда $\forall \epsilon>0\ \exists\ A>a,\ \forall a_1,a_2>A$:
    \[\left|\int\limits_{a_1}^{a_2}f(x,y)\ dx\right|<\epsilon\]
\end{theorem}
\begin{proof}\tab
    \begin{itemize}
        \item[$(\Rightarrow)$:]
        \[\left|\int\limits_{a_1}^{a_2}f(x,y)\ dx\right|=\left|\int\limits_{a_1}^{+\infty}f(x,y)\ dx-\int\limits_{a_2}^{+\infty}f(x,y)\ dx\right|< 2\epsilon\]
        \item[$(\Leftarrow)$:]
        \[\left|\int\limits_{a_1}^{a_2}f(x,y)\ dx\right|<\epsilon\]
        Тогда при $a_2\to +\infty$:
        \[\left|\int\limits_{a_1}^{+\infty}f(x,y)\ dx\right|\leq \epsilon\]
    \end{itemize}
\end{proof}
\begin{theorem} (Признак Вейерштрасса равномерной сходимости)\\
    Рассмотрим
    \[\int\limits_{a}^{+\infty}f(x,y)\ dx,\ y\in Y\]
    Если $\exists\ g(x)$ такая, что $|f(x,y)|<g(x)$ и сходится интеграл
    \[\int\limits_{a}^{+\infty}g(x)\ dx\]
    то интеграл 
    \[\int\limits_{a}^{+\infty}f(x,y)\ dx\]
    сходится равномерно на $Y$.
\end{theorem}
\begin{proof}
    По критерию Коши:
    \[\left|\int\limits_{a_1}^{a_2}f(x,y)\ dx\right|\leq \int\limits_{a_1}^{a_2}g(x)\ dx<\epsilon\]
\end{proof}
\begin{theorem} (Признаки Абеля и Дирихле равномерной сходимости)\\
    Рассмотрим интеграл
    \[\int\limits_{a}^{+\infty}f(x,y)\cdot g(x,y)\ dx,\ y\in Y\]
    Пусть $|f(x,y)|,\ |g(x,y)|<M$
    \begin{itemize}
        \item[($\mathcal{A}$):] 
        \[\int\limits_{a}^{+\infty}f(x,y)\ dx \overset{Y}{\rightrightarrows}\]
        $g(x,y)$ монотонна $\forall y$. Тогда интеграл
        \[\int\limits_{a}^{+\infty}f(x,y)\cdot g(x,y)\ dx\overset{Y}{\rightrightarrows}\]
        \item[($\mathcal{D}$):]
        \[\left|\int\limits_{a}^{A}f(x,y)\ dx\right|<M\]
        $g(x,y)$ монотонна и $g(x)\overset{Y}{\rightrightarrows}0$. Тогда
        \[\int\limits_{a}^{+\infty}f(x,y)\cdot g(x,y)\ dx\overset{Y}{\rightrightarrows}\]
    \end{itemize}
\end{theorem}
\begin{proof} % по первой теореме о среднем.
    \begin{multline*}
        \left|\int\limits_{a_1}^{a_2}f(x,y)g(x,y)\ dx\right|=\left|g(a_1,y)\cdot \int\limits_{a_1}^{c}f(x,y)\ dx+ g(a_2,y)\cdot \int\limits_{c}^{a_2}f(x,y)\ dx\right|\leq \\
        \leq |g(a_1,y)|\cdot \left|\int\limits_{a_1}^{c}f(x,y)\ dx \right|+|g(a_2,y)|\cdot \left|\int\limits_{c}^{a_2}f(x,y)\ dx\right|=(*)
    \end{multline*}
    $(\mathcal{A}):\ (*)\leq \epsilon\cdot M+\epsilon\cdot M$\\
    $(\mathcal{D}):\ (*)\leq \epsilon\cdot 2M+\epsilon\cdot 2M$
\end{proof}