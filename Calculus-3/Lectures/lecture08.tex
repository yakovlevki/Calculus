\begin{consequense}
    Если произведение сходится, то общий член имеет вид 
    \[u_n=1+a_n,\ a_n\to 0\]
\end{consequense}
\subsection{Связь бесконечных произведений и рядов}
\begin{theorem}
    Произведение 
    \[\prod\limits_{n=1}^{\infty}(1+a_n)\]
    сходится тогда и только тогда, когда ряд
    \[\sum\limits_{n=1}^{\infty}\ln{(1+a_n)}\]
    сходится.
\end{theorem}
\begin{proof}\tab
    \begin{itemize}
        \item[($\Rightarrow$):] 
        Пусть произведение сходится, тогда $\Pi_N\to \Pi$. Поскольку $\ln{x}\in \mathcal{C}(0,+\infty)$, то 
        \[\sum\limits_{n=1}^{\infty}\ln{(1+a_n)}=\lim\limits_{N\to \infty} \ln{(\Pi_N)}=\ln{(\Pi)}\]
        значит, ряд сходится.
        \item[($\Leftarrow$):]
        Пусть ряд сходится, тогда $S_N\to S$. Поскольку $\exp{x}\in \mathcal{C}(0,+\infty)$, то
        \[\prod\limits_{n=1}^{\infty}(1+a_n)=\lim\limits_{N\to\infty} \exp{(S_N)}=\exp(S)\]
        значит, произведение сходится.
    \end{itemize}
\end{proof}
\begin{definition}
    Произведение называется абсолютно сходящимся или условно сходящимся, если таковым является соответствующий ряд из логарифмов. 
\end{definition}
\begin{theorem} Произведение
    \[\prod\limits_{n=1}^{\infty}(1+a_n)\]
    сходится абсолютно тогда и только тогда, когда ряд
    \[\sum\limits_{n=1}^{\infty}|a_n|\]
    сходится.
\end{theorem}
\begin{proof}
    \[\lim\limits_{n\to\infty}\frac{|\ln(1+a_n)|}{|a_n|}=1>0\]
    отсюда, по признаку сравнения в предельной форме, получаем утверждение теоремы.
\end{proof}
\begin{consequense}
    Если $a_n$ знакопостоянны, то произведение и соответствующий ряд из логарифмов сходятся или расходятся одновременно.
\end{consequense}
\begin{theorem}
    Пусть ряд
    \[\sum\limits_{n=1}^{\infty}a_n\]
    сходится. Тогда произведение 
    \[\prod\limits_{n=1}^{\infty}(1+a_n)\]
    сходится или расходится одновременно с рядом
    \[\sum\limits_{n=1}^{\infty}a_n^2\]
\end{theorem}
\begin{proof}
    Разложим по формуле Тейлора:
    \[\ln(1+a_n)-a_n=-\frac{a_n^2}{2}+\bar{\bar{o}}(a_n^2)\]
    \[\lim\limits_{n\to\infty}\frac{\ln(1+a_n)-a_n}{a_n^2}=-\frac{1}{2}\]
    Значит, по признаку сравнения в предельной форме, ряды 
    \[\sum\limits_{n=1}^{\infty}(\ln{(1+a_n)-a_n}),\ \ \ \sum\limits_{n=1}^{\infty}a_n^2\]
    сходятся или расходятся одновременно. При этом
    \[\sum\limits_{n=1}^{\infty}\ln{(1+a_n)}=\sum\limits_{n=1}^{\infty}(\ln{(1+a_n)-a_n})+\sum\limits_{n=1}^{\infty}a_n\]
    то есть ряды 
    \[\sum\limits_{n=1}^{\infty}\ln{(1+a_n)}, \ \ \ \sum\limits_{n=1}^{\infty}(\ln{(1+a_n)}-a_n)\]
    сходятся или расходятся одновременно. Таким образом, комбинируя эти факты, получаем утверждение теоремы.
\end{proof}
\begin{comm}
    Можно рассматривать функциональные произведения
    \[\prod\limits_{n=1}^{\infty}(1+a_n(x))\]
    анализируя соответствующий функциональный ряд из логарифмов.
\end{comm}
\begin{example} (Пример Эйлера)\\
    Пусть $p_k$ обозначает $k$-е простое число. Тогда при $s>1$ 
    \[\zeta(s)=\sum\limits_{n=1}^{\infty}\frac{1}{n^s}=\prod\limits_{k=1}^{\infty}\left(1-\frac{1}{p_k^s}\right)^{-1}\]
    \[\prod\limits_{k=1}^{N}\left(1-\frac{1}{p_k^s}\right)^{-1}=\prod\limits_{k=1}^{N}\left(\sum\limits_{l=0}^{\infty}\frac{1}{p^{ls}}\right)=\sum\frac{1}{(p_{k_1}^{l_1}\cdot p_{k_2}^{l_2}\cdot \dots\cdot p_{k_m}^{l_m})^s}\]
    \[\sum\limits_{n=1}^{N}\frac{1}{n^s}<\sum\frac{1}{(p_{k_1}^{l_1}\cdot p_{k_2}^{l_2}\cdot \dots\cdot p_{k_m}^{l_m})^s}<\sum\limits_{n=1}^{\infty}\frac{1}{n^s}\]
    Тогда
    \[\sum\frac{1}{(p_{k_1}^{l_1}\cdot p_{k_2}^{l_2}\cdot \dots\cdot p_{k_m}^{l_m})^s} \to \sum\limits_{n=1}^{\infty}\frac{1}{n^s}=\zeta(s),\ N\to \infty\]
    %все ряды сходятся абсолютно, 
\end{example}
\subsection{Разложение синуса в бесконечное произведение}
\begin{lemma}
    $\forall n\in \N$:
    \[\sin((2n+1)x)=(2n+1)\sin{x}\cdot P_n(\sin^2{x})\]
    где $P_n$ - многочлен $n$-й степени.
\end{lemma}
\begin{proof}
    Индукция по $n$. База $n=1$:
    \[\sin{3x}=3\sin{x}-4\sin^3{x}\]
    Шаг: Пусть верно для $m<2n+1$ (нам понадобятся только $2n-3$ и $2n-1$)
    \[\sin{(2n+1)x}+\sin{(2n-3)x}=2\sin{(2n-1)x}\cdot \cos{2x}\]
    Отсюда выразим $\sin{(2n+1)x}$ и подставим выражения для $2n-3$ и $2n-1$:
    \begin{multline*}
        \sin{(2n+1)x}=2\sin{(2n-1)x}\cdot \cos{2x}-\sin{(2n-3)x}=\\
        =(2n-1)\sin{x}\cdot P_{n-1}(\sin^2{x})\cdot (1-2\sin^2{x})-(2n-3)\sin{x}\cdot P_{n-2}(\sin^2{x})=\\
        =(2n+1)\sin{x}\cdot \Bigg(-\frac{2(2n-1)}{2n+1}\cdot P_n(\sin^2{x})+\frac{2n-1}{2n+1}\cdot P_{n-1}(\sin^2{x})-\frac{2n-3}{2n+1}\cdot P_{n-2}(\sin^2{x})\Bigg)
    \end{multline*}
\end{proof}
\begin{lemma}
    $\forall \{a_k\}_{k=1}^n\subset \R:$
    \[\left|\prod\limits_{k=1}^{n}(1+a_k)-1\right|\leq \prod\limits_{k=1}^{n}(1+|a_k|)-1\]
\end{lemma}
\begin{proof}
    Индукция по $n$. База $n=1$:
    \[|(1+a_1)-1|\leq (1+|a_1|-1)\]
    Шаг: Пусть верно для $n_1$:
    \[\left|\ \prod\limits_{k=1}^{n-1}(1+a_k)-1\ \right|\leq \prod\limits_{k=1}^{n-1}(1+|a_k|)-1\]
    \begin{multline*}
        \left|\ \prod\limits_{k=1}^{n}(1+a_k)-1\ \right|=\left|\ (1+a_n)\prod\limits_{k=1}^{n-1}(1+a_k)-1\ \right|\overset{(1)}{=}\\
        \overset{(1)}{=}\left|\ (1+a_n)\prod\limits_{k=1}^{n-1}(1+a_k)-(1+a_n)+a_n\ \right|=\tab[2cm]\\
        =\left|\ (1+a_n)\left(\prod\limits_{k=1}^{n-1}(1+a_k)-1\right)+a_n\ \right|\leq\\
        \tab[2cm]\leq |1+a_n|\cdot \left|\prod\limits_{k=1}^{n-1}(1+a_k)-1\right|+|a_n|\overset{(2)}{\leq}\\
        \overset{(2)}{\leq}|1+a_n|\cdot \prod\limits_{k=1}^{n-1}(1+|a_k|)-1+|a_n|=\prod\limits_{k=1}^{n}(1+|a_k|)-1
    \end{multline*}
    (1): Добавили и вычли $a_n$.\\
    (2): По предположению индукции.
\end{proof}
\begin{lemma}
    Если $x\in [0,\frac{\pi}{2}]$, то $\frac{2x}{\pi}\leq \sin{x}\leq x$
\end{lemma}
\begin{theorem} (Разложение синуса в бесконечное произведение)\\
    $\forall x\ne \pi k$:
    \[\sin{x}=x\cdot \prod\limits_{n=1}^{\infty}\left(1-\frac{x^2}{\pi^2 n^2}\right)\]
\end{theorem}
\begin{proof}
    $\forall n\in \N$:
    \[\sin((2n+1)x)=(2n+1)\sin{x}\cdot P_n(\sin^2{x})\]
    \[\sin{((2n+1)x)}=0 \Leftrightarrow x=\frac{\pi k}{2n+1},\ k\in \Z\]
    при этом
    \[\sin{\frac{\pi k}{2n+1}}\neq 0\]
    значит, обязан обнуляться многочлен:
    \[P_n(\sin^2{x})=0,\ x=\frac{\pi k}{2n+1},\ k=1,\dots,n\]
    Таким образом, найдены все корни многочлена. Обозначим $\omega = \sin^2{x}$, тогда
    \[P_n(\omega)=0 \Leftrightarrow \omega=\sin^2{\frac{\pi k}{2n+1}},\ k=1,\dots, n\]
    Разложим многочлен на множители:
    \[P_n(\omega)=B\cdot \prod\limits_{k=1}^{n}\left(\omega-\sin^2\frac{\pi k}{2n+1}\right)\]
    где $B$ - коэффциент при старшем члене. Преобразуем:
    \begin{multline*}
        P_n(\sin^2{x})=B\cdot \prod\limits_{k=1}^{n}\left(\sin^2{x}-\sin^2{\frac{\pi k}{2n+1}}\right)=\\
        =B\cdot \prod\limits_{k=1}^{n}\left(-\sin^2{\frac{\pi k}{2n+1}}\right)\cdot \prod\limits_{k=1}^{n}\left(1-\frac{\sin^2{x}}{\sin^2{\frac{\pi k}{2n+1}}}\right)=A\cdot \prod\limits_{k=1}^{n}\left(1-\frac{\sin^2{x}}{\sin^2{\frac{\pi k}{2n+1}}}\right)
    \end{multline*}
    Таким образом, из начального соотношения, получим
    \[\frac{\sin((2n+1)x)}{(2n+1)\sin{x}}=A\cdot \prod\limits_{k=1}^{n}\left(1-\frac{\sin^2{x}}{\sin^2{\frac{\pi k}{2n+1}}}\right)\]
    Перейдя к пределу при $x\to 0$ в обеих частях, получим
    \[\frac{\sin((2n+1)x)}{(2n+1)\sin{x}}\to 1,\ \ \ \prod\limits_{k=1}^{n}\left(1-\frac{\sin^2{x}}{\sin^2{\frac{\pi k}{2n+1}}}\right)\to 1\]
    значит, $A=1$ и выражение имеет вид:
    \[\frac{\sin((2n+1)x)}{(2n+1)\sin{x}}=\prod\limits_{k=1}^{n}\left(1-\frac{\sin^2{x}}{\sin^2{\frac{\pi k}{2n+1}}}\right)\]
    сделаем замену $t=(2n+1)x$. Тогда
    \[\frac{\sin{t}}{(2n+1)\cdot \sin{\frac{t}{2n+1}}}=\prod\limits_{k=1}^{n}\left(1-\frac{\sin^2{\frac{t}{2n+1}}}{\sin^2{\frac{\pi k}{2n+1}}}\right)\]
    Возьмем $t\in \R,\ |t|<n, m<n$ и предаствим произведение в виде
    \[\frac{\sin{t}}{(2n+1)\cdot \sin{\frac{t}{2n+1}}}=\prod\limits_{k=1}^{m}\left(1-\frac{\sin^2\frac{t}{2n+1}}{\sin^2\frac{\pi k}{2n+1}}\right)\cdot R_{n,m}(t)\]
    где 
    \[R_{n,m}(t)=\prod\limits_{k=m+1}^{n}\left(1-\frac{\sin^2{\frac{t}{2n+1}}}{\sin^2{\frac{\pi k}{2n+1}}}\right)\]
    Посмотрим на пределы при $n\to \infty$:
    \[\frac{\sin{t}}{(2n+1)\cdot \sin{\frac{t}{2n+1}}}\to \frac{\sin{t}}{t},\ \ \ \prod\limits_{k=1}^{m}\left(1-\frac{\sin^2\frac{t}{2n+1}}{\sin^2\frac{\pi k}{2n+1}}\right)\to \prod\limits_{k=1}^{m}\left(1-\frac{t^2}{\pi^2 k^2}\right)\]
    причем $R_{n,m}(t)$ выражается через них, значит существует предел
    \[\lim\limits_{n\to\infty}R_{n,m}(t)=R_m(t)\]
    Итак, перейдем к пределу при $n\to \infty$:
    \[\frac{\sin{t}}{t}=\prod\limits_{k=1}^{m}\left(1-\frac{t^2}{\pi^2 k^2}\right)\cdot R_m(t)\]
    Оценим к $R_{n,m}$, воспользовавшись леммами:
    \begin{multline*}
        \left|\prod\limits_{k=m+1}^{n}\left(1-\frac{\sin^2\frac{t}{2n+1}}{\sin^2\frac{\pi k}{2n+1}}\right)-1\right|\leq \prod\limits_{k=m+1}^{n}\left(1+\frac{\sin^2\frac{t}{2n+1}}{\sin^2\frac{\pi k}{2n+1}}\right)-1\leq\\
        \leq \prod\limits_{k=m+1}^{n}\left(1+\frac{t^2}{(\frac{2}{\pi}\cdot \pi k)^2}\right)-1=\prod\limits_{k=m+1}^{n}\left(1+\frac{t^2}{4k^2}\right)-1<\\
        <\prod\limits_{k=m+1}^{\infty}\left(1+\frac{t^2}{4k^2}\right)-1
    \end{multline*}
    Рассмотрим произведение
    \[\prod\limits_{k=1}^{\infty}\left(1+\frac{t^2}{4k^2}\right)=C\]
    оно сходится, поскольку сходится ряд 
    \[\sum\limits_{k=1}^{\infty}\left|\frac{t^2}{4k^2}\right|\]
    при этом
    \[\prod\limits_{k=m+1}^{\infty}\left(1+\frac{t^2}{4k^2}\right)=\frac{\prod\limits_{k=1}^{\infty}\left(1+\frac{t^2}{4k^2}\right)}{\prod\limits_{k=1}^{m}\left(1+\frac{t^2}{4k^2}\right)}\to\frac{C}{C}=1,\ m\to \infty\]
    Значит, $R_m(t)\to 1$ при $m\to \infty$.
    Таким образом, получаем утверждение теоремы:
    \[\sin{t}=t\cdot \prod\limits_{k=1}^{\infty}\left(1-\frac{t^2}{\pi^2 k^2}\right)\]
\end{proof}