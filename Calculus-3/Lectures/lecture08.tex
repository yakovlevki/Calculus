\begin{consequense}
    Если произведение сходится, то общий член имеет вид 
    \[u_n=1+a_n,\ a_n\to 0\]
\end{consequense}
\begin{theorem}
    Произведение 
    \[\prod\limits_{n=1}^{\infty}(1+a_n)\]
    сходится тогда и только тогда, когда ряд
    \[\sum\limits_{n=1}^{\infty}\ln{(1+a_n)}\]
    сходится.
\end{theorem}
\begin{proof}\tab
    \begin{itemize}
        \item[($\Rightarrow$):] 
        \[\Pi_{N}\text{ - сходится} \Rightarrow \ln(\Pi_N) \text{ - сходится},\ \ln(x)\in \mathcal{C}(0, +\infty) \Rightarrow \sum_{n=1}^N ln(1+a_n) \text{ - сходится}\]
        \item[($\Leftarrow$):]
        \[\sum_{n=1}^N \ln(1+a_n) \text{ - сходится} \Rightarrow e^{\left(\sum\limits_{n=1}^N \ln(1+a_n)\right)} \text{ - сходится} \Rightarrow \Pi_N \text{ - сходится}\]
    \end{itemize}
\end{proof}
\begin{definition}
    Произведение называется абсолютно сходящимся или условно сходящимся, если таковым является соответсвующий ряд из логарифмов. 
\end{definition}
\begin{theorem} Произведение
    \[\prod\limits_{n=1}^{\infty}u_n\]
    сходится абсолютно тогда и только тогда, когда ряд
    \[\sum\limits_{n=1}^{\infty}|a_n|\]
    сходится.
\end{theorem}
\begin{proof}
    \[\lim\limits_{n\to\infty}\frac{|\ln(1+a_n)|}{|a_n|}=1\]
    из этого следует утверждение в обе стороны.
\end{proof}
\begin{consequense}
    Если $a_n$ знакопостоянны, то произведение и соответсвующий ряд из логарифмов сходятся или расходится одновременно.
\end{consequense}
\begin{theorem}
    Пусть ряд
    \[\sum\limits_{n=1}^{\infty}a_n\]
    сходится. Тогда произведение 
    \[\prod\limits_{n=1}^{\infty}(1+a_n)\]
    сходится или расходится одновременно с рядом
    \[\sum\limits_{n=1}^{\infty}a_n^2\]
\end{theorem}
\begin{proof}
    \[\ln(1+a_n)-a_n=-\frac{a_n^2}{2}+\bar{\bar{o}}(a_n^2)\]
    \[\lim\limits_{n\to\infty}\frac{\ln(1+a_n)-a_n}{a_n^2}=-\frac{1}{2}\]
\end{proof}
\begin{comm}
    Можно рассматривать функциональные произведения
    \[\prod\limits_{n=1}^{\infty}(1+a_n(x))\]
    анализируя соответсвующий функциональный ряд из логарифмов.
\end{comm}
\begin{example} (Пример Эйлера)\\
    Пусть $p_k$ обозначает $k$-е простое число. Тогда при $s>1$ 
    \[\zeta(s)=\sum\limits_{n=1}^{\infty}\frac{1}{n^s}=\prod\limits_{k=1}^{\infty}(1-\frac{1}{p_k^s})^{-1}\]
    \[\prod\limits_{k=1}^{N}\left(1-\frac{1}{p_k^s}\right)^{-1}=\prod\limits_{k=1}^{N}\left(\sum\limits_{l=0}^{\infty}\frac{1}{p^{ls}}\right)=\sum\frac{1}{(p_{k_1}^{l_1}\cdot p_{k_2}^{l_2}\cdot \dots\cdot p_{k_m}^{l_m})^s}\]
    \[\sum\limits_{n=1}^{N}\frac{1}{n^s}<\sum\frac{1}{(p_{k_1}^{l_1}\cdot p_{k_2}^{l_2}\cdot \dots\cdot p_{k_m}^{l_m})^s}<\sum\limits_{n=1}^{\infty}\frac{1}{n^s}\]
    Тогда
    \[\sum\frac{1}{(p_{k_1}^{l_1}\cdot p_{k_2}^{l_2}\cdot \dots\cdot p_{k_m}^{l_m})^s} \to \sum\limits_{n=1}^{\infty}\frac{1}{n^s}=\zeta(s),\ N\to \infty\]
    %все ряды сходятся абсолютно, 
\end{example}
\begin{theorem}
    $\forall x\ne \pi k$:
    \[\sin{x}=x\cdot \prod\limits_{n=1}^{\infty}\left(1-\frac{x^2}{\pi^2 n^2}\right)\]
\end{theorem}
\begin{proof}
    $\forall n: \sin((2n+1)x)=(2n+1)\sin{x}\cdot P_n(\sin^2{x})$\\
    это утверждение можно показать по индукции, с базой $\sin{3x}=3\sin{x}-4\sin^3{x}$\\
    \[\sin{((2n+1)x)}=0 \Leftrightarrow x=\frac{\pi k}{2n+1},\ k\in \Z\]
    при этом
    \[P_n(\sin^2{x})=0,\ x=\frac{\pi k}{2n+1},\ k=1,\dots,n\]
    \[P_n(\omega)=0 \Leftrightarrow \omega=\sin^2{\frac{\pi k}{2n+1}},\ k=1,\dots, n\]
    Заметим, что
    \[P_n(\sin^2{x})=\frac{\sin((2n+1)x)}{(2n+1)\sin{x}}\to 1,\ x\to 0\]
    Тогда
    \[P_n(\omega)=\prod\limits_{k=1}^{n}(1-\frac{\omega^2}{\sin^2{\frac{\pi k}{2n+1}}})\]
    \[P(\sin^2{x})=\prod\limits_{k=1}^{n}\left(1-\frac{\sin^2{x}}{\sin^2{\frac{\pi k}{2n+1}}}\right)=\frac{\sin((2n+1)x)}{(2n+1)\sin{x}}\]
    сделаем замену: $(2n+1)x=t$
    \[\frac{\sin{t}}{(2n+1)\cdot \sin{\frac{t}{2n+1}}}=\prod\limits_{k=1}^{n}\left(1-\frac{\sin^2{\frac{t}{2n+1}}}{\sin^2{\frac{\pi k}{2n+1}}}\right)\]
    \begin{lemma}
        $\forall \{a_k\}_{k=1}^n\subset \R:$
        \[\left|\prod\limits_{k=1}^{n}(1+a_k)-1\right|\leq \prod\limits_{k=1}^{n}(1+|a_k|)-1\]
    \end{lemma}
    \begin{proof}
        Очев по индукции.
    \end{proof}
    Возьмем $\forall t\in \R,\ |t|<n, m<n$
    \[\frac{\sin{t}}{(2n+1)\cdot \sin{\frac{t}{2n+1}}}=\prod\limits_{k=1}^{m}\left(1-\frac{\sin^2\frac{t}{2n+1}}{\sin^2\frac{\pi k}{2n+1}}\right)\cdot R_{n,m}(t) \Rightarrow \exists\ \lim\limits_{n\to\infty}R_{n,m}(t)=R_n(t)\]
    Перейдем к пределу при $n\to \infty$:
    \[\frac{\sin{t}}{t}=\prod\limits_{k=1}^{m}\left(1-\frac{t^2}{\pi^2 k^2}\right)\cdot R_m(t)\]
    Вернемся к $R_{m,n}$:
    \begin{multline*}
        \left|\prod\limits_{k=m+1}^{n}\left(1-\frac{\sin^2\frac{t}{2n+1}}{\sin^2\frac{\pi k}{2n+1}}\right)\right|\leq \prod\limits_{k=m+1}^{n}\left(1+\frac{\sin^2\frac{t}{2n+1}}{\sin^2\frac{\pi k}{2n+1}}\right)-1\leq\\
        \leq \prod\limits_{k=m+1}^{n}\left(1+\frac{t^2}{(\frac{2}{\pi}\cdot \pi k)^2}\right)-1=\prod\limits_{k=m+1}^{n}\left(1+\frac{t^2}{4k^2}\right)-1<\\
        <\prod\limits_{k=m+1}^{\infty}\left(1+\frac{t^2}{4k^2}\right)-1\to 0
    \end{multline*}
    при $m\to \infty \Rightarrow R_m(t)\to 1,\ m\to \infty$. Значит
    \[\sin{t}=t\cdot \prod\limits_{k=1}^{\infty}\left(1-\frac{t^2}{\pi^2 k^2}\right)\]
\end{proof}