\begin{theorem} (Теорема Фейера)\\
    Если $f(x)\in \mathcal{C}(\R)$ и $2\pi$ периодическая, то $\Phi_N(x)\overset{\R}{\rightrightarrows} f(x)$
\end{theorem}
\begin{proof}
    \begin{multline*}
        F_N(x) = \frac{a_0}{2}+\sum\limits_{n=1}^{N}(a_n\cdot \cos{(nx)}+b_n\cdot \sin{(nx)}) = \\
        \hspace{-14cm} = \frac{1}{2\pi} \int\limits_{-\pi}^{\pi}f(t) \ dt + \\ % поправишь сам 
        + \sum\limits_{n=1}^{N} \left( \frac{1}{\pi}\int\limits_{\pi}^{\pi}f(t) \cos{(nt)} \ dt \cdot \cos{(nx)} + \frac{1}{\pi}\int\limits_{-\pi}^{\pi}f(t) \sin{(nt)} \ dt \cdot \sin{(nx)} \right) = \\
        = \frac{1}{\pi}\int\limits_{-\pi}^{\pi} f(t) \left(\frac{1}{2} + \sum\limits_{n=1}^N \cos{(n(t-x))} dt \right) = \frac{1}{\pi}\cdot \int\limits_{-\pi}^{\pi} f(t+x) \cdot D_N(t) \ dt
    \end{multline*}
    Получили:
    \[F_N(x)=\frac{1}{\pi}\cdot \int\limits_{-\pi}^{\pi}D_N(t)\cdot f(t+x)\ dt\]

    отсюда
    \[\Phi_N(x)=\frac{1}{\pi} \int\limits_{-\pi}^{\pi}\Psi_N(t)f(t+x)\ dt\] % это тоже надо пояснить 
    \begin{multline*}
        |\Phi_N(x)-f(x)|=\frac{1}{\pi}\cdot \left|\ \int\limits_{-\pi}^{\pi}\Psi_N(x)\cdot (f(t+x)-f(x))\ dt\ \right|\leq \\
        \leq \frac{1}{\pi}\cdot \int\limits_{-\pi}^{-\delta}\Psi_N(t)\cdot |f(t+x)-f(x)|\ dt+\frac{1}{\pi}\cdot \int\limits_{-\delta}^{\delta}\Psi_N(t)\cdot |f(t+x)-f(x)|\ dt+\\
        +\frac{1}{\pi}\cdot \int\limits_{\delta}^{\pi}\Psi_N(t)\cdot |f(t+x)-f(x)|\ dt\leq \frac{\epsilon}{\pi}+\frac{\epsilon}{\pi}\cdot 2M\cdot 2
    \end{multline*}
    \textit{(будет пояснено)} % там по равномерной непрерывности разность оценивается
\end{proof}
\begin{theorem} (Теорема Вейерштрасса о равномерном приближении непрерывной функции на отрезке многочленами)\\
    Пусть $f(x)\in \mathcal{C}[a,b]$, а также $\forall \epsilon>0\ \exists P_{\epsilon}(x)$ такой, что
    \[|f(x)-P_{\epsilon}(x)|<\epsilon,\ \forall x\in [a,b]\]
\end{theorem}
\begin{proof}
    Линейной заменой получим, что достаточно доказать для отрезка $[0,\pi]$, затем непрерывно продолжаем ее до отрезка $[-\pi,\pi]$. Тогда по теореме Фейера существует тригонометрический многочлен, приближающий функцию. Затем раскладываем синусы и косинусы по формулам Тейлора и получаем утверждение теоремы.\\
    \textit{(была доказана устно, тоже будет пояснено)}
\end{proof}
\begin{theorem} (Равенство Парсеваля)\\
    \[\frac{a_0^2}{2}+\sum\limits_{n=1}^{\infty}(a_n^2+b_n^2)=\frac{1}{\pi}\int\limits_{-\pi}^{\pi}f^2(x)\ dx\]
\end{theorem}
\begin{proof}
    \begin{multline*}
        \frac{1}{\pi}\int\limits_{-\pi}^{\pi}(f(x)-T_N(x))^2\ dx=\frac{1}{\pi}\int\limits_{-\pi}^{\pi}f^2(x)\ dx+\\
        +2\left(\frac{a_0}{2}-\frac{A_0}{2}\right)+\sum\limits_{n=1}^{N}((a_n-A_n)^2+(b_n-B_n)^2)-\frac{a_0^2}{2}-\sum\limits_{n=1}^{N}(a_n^2+b_n^2)
    \end{multline*}
    $\forall \epsilon>0\ \exists\ T_N(x)$ такой, что
    \[\epsilon>\frac{1}{\pi}\int\limits_{-\pi}^{\pi}(f(x)-T_N(x))^2\ dx\geq \frac{1}{\pi}\int\limits_{-\pi}^{\pi}f^2(x)\ dx-\frac{a_0^2}{2}-\sum\limits_{n=1}^{N}(a_n^2+b_n^2)\]
    отсюда
    \[\epsilon>\frac{1}{\pi}\int\limits_{-\pi}^{\pi}f^2(x)\ dx-\frac{a_0^2}{2}-\sum\limits_{n=1}^{\infty}(a_n^2+b_n^2)\]
    значит
    \[\frac{a_0^2}{2}+\sum\limits_{n=1}^{\infty}(a_n^2+b_n^2)=\frac{1}{\pi}\int\limits_{-\pi}^{\pi}f^2(x)\ dx\]
\end{proof}