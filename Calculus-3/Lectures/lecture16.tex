\begin{definition}
    Пусть существуют интегралы
    \[\int\limits_{-\pi}^{\pi}f(x)\ dx,\ \ \int\limits_{-\pi}^{\pi}|f(x)|\ dx\]
    Тогда числа
    \[a_n=\frac{1}{\pi}\cdot \int\limits_{\pi}^{\pi}f(x)\cdot \cos{(nx)}\ dx,\ \ b_n=\int\limits_{\pi}^{\pi}f(x)\cdot \sin{(nx)}\ dx\]
    называются коэффициэнтами Фурье функции $f(x)$, а ряд
    \[\frac{a_0}{2}+\sum\limits_{n=1}^{\infty}(a_n\cdot \cos{(nx)}+b_n\cdot \sin{(nx)})\]
    называется рядом Фурье функции $f(x)$.
\end{definition}
\begin{theorem}
    Пусть существуют интегралы
    \[\int\limits_{\pi}^{\pi}f(x)\ dx,\ \ \int\limits_{\pi}^{\pi}f^2(x)\ dx\]
    а также
    \[F_N(x)=\frac{a_0}{2}+\sum\limits_{n=1}^{N}(a_n\cdot \cos{(nx)}+b_n\cdot \sin{(nx)})\]
    Тогда
    \[\int\limits_{-\pi}^{\pi}(f(x)-F_N(x))^2\ dx=\min\limits_T \int\limits_{-\pi}^{\pi}(f(x)-T_N(x))^2\ dx\]
    где $T_N$ - тригонометрический полином степени не выше N.
\end{theorem}
\begin{proof}
    \begin{multline*}
        \int\limits_{-\pi}^{\pi}\left(f(x)-\frac{A_0}{2}-\sum\limits_{n=1}^{N}(A_n\cdot \cos{(nx)}+B_n\cdot \sin{(nx)})^2\right)dx=\\
        =\int\limits_{-\pi}^{\pi}f^2(x)\ dx-2\pi\cdot \frac{A_0^2}{4}+\pi\cdot \sum\limits_{n=1}^{N}(A_n^2+B_n^2)-2\cdot \frac{A_0}{2}\cdot \int\limits_{-\pi}^{\pi}f(x)\ dx-\\
        -2\cdot \sum\limits_{n=1}^{N}\left(A_n\cdot \int\limits_{-\pi}^{\pi}f(x)\cdot \cos{(nx)}\ dx+B_n\cdot \int\limits_{-\pi}^{\pi}f(x)\cdot \sin{(nx)}\ dx\right)=\\
        =\int\limits_{-\pi}^{\pi}f^2(x)\ dx+2\pi\cdot \frac{A_0^2}{4}+\pi\cdot \sum\limits_{n=1}^{N}(A_n^2+B_n^2)-4\pi\cdot \frac{A_0}{2}\cdot \frac{a_0}{2}-2\pi\cdot \sum\limits_{n=1}^{N}(A_n a_n+B_n b_n)+\\
        +2\pi\cdot\frac{a_0^2}{4}+\pi\cdot \sum\limits_{n=1}^{N}(a_n^2+b_n^2)-2\pi\cdot \frac{a_0^2}{4}-\pi\cdot \sum\limits_{n=1}^{N}(a_n^2+b_n^2)=\\
        =\int\limits_{-\pi}^{\pi}f^2(x)\ dx+2\pi\left(\frac{A_0}{2}-\frac{a_0}{2}\right)^2+\pi\cdot \sum\limits_{n=1}^{N}((A_n-a_n)^2+(B_n-b_n)^2)-\\
        -2\pi\cdot \frac{a_0^2}{4}-\pi\cdot \sum\limits_{n-1}^{N}(a_n^2+b_n^2)
    \end{multline*}
    Первое слагаемое - констанста, она не от чего не зависит. Последние два слагаемых не зависят от $T$. Таким образом, у этой величины есть минимум, тогда и только тогда, когда $A_n=a_n$ и $B_n=b_n$, что и означает $T_N=F_N$.
\end{proof}
\begin{consequense} (Неравенство Бесселя)\\
    \[\frac{a^2_0}{2}+\sum\limits_{n=1}^{N}(a_n^2+b_n^2)\leq \int\limits_{-\pi}^{\pi}f^2(x)\ dx\]
\end{consequense}
\begin{proof}
    Из доказательства теоремы, заметим, что
    \[\int\limits_{-\pi}^{\pi}f^2(x)\ dx-\pi\cdot \sum\limits_{n=1}^{N}(a_n^2+b_n^2)-\pi\cdot \frac{a_0^2}{2}\geq 0\]
    это так, поскольку это часть слагаемых вычисленного интеграла от квадрата функции. Отсюда
    \[\frac{a_0^2}{2}+\sum\limits_{n=1}^{N}(a_n^2+b_n^2) \leq\frac{1}{\pi}\cdot\int\limits_{-\pi}^{\pi}f^2(x)\ dx\]
\end{proof}
\begin{consequense}
    При $n\to \infty$ выполнено, что $a_n\to 0,\ b_n\to 0$
\end{consequense}
\begin{proof}
    При $n\to \infty$ ряд сходится, значит, по необходимому условию сходимости, выполнено $a_n\to 0, b_n\to 0$.
\end{proof}
\begin{definition}
    $N$-м ядром Дирихле называется
    \[D_N(x)=\frac{1}{2}+\sum\limits_{n=1}^{N}\cos{(nx)}\]
\end{definition}
\begin{theorem} (Свойства ядер Дирихле)\\
    \begin{enumerate}
        \item $D_n$ непрерывная, четная и $2\pi$-периодичная функция.
        \item $D_n(0)=N+\frac{1}{2}$
        \item \[\frac{1}{\pi}\int\limits_{-\pi}^{\pi}D_N(x)\ dx=1\]
        \item $\forall x\ne 2\pi k$.
        \[D_n(x)=\frac{\sin{((N+\frac{1}{2})x)}}{2\cdot \sin{(\frac{x}{2})}}\]
    \end{enumerate}
\end{theorem}
\begin{proof}
    \begin{enumerate}
        \item Очев
        \item $D_N(0)=\frac{1}{2}+\sum\limits_{n=1}^{N}\cos{0}=N+\frac{1}{2}$
        \item \[\int\limits_{-\pi}^{\pi}D_N(x)\ dx=\int\limits_{-\pi}^{\pi}\ \left(\frac{1}{2}+\sum\limits_{n=1}^{N}\cos{(nx)}\right)=\int\limits_{-\pi}^{\pi}\frac{1}{2}\ dx+\sum\limits_{n=1}^{N}\left(\ \int\limits_{-\pi}^{\pi}\cos{(nx)}\ dx\right)=\pi\]
        \item Домножим и поделим на $\sin{\frac{x}{2}}$, дальше очев.
    \end{enumerate}
\end{proof}
\begin{comm}
    \begin{multline*}
        F_N(x)=\frac{a_0}{2}+\sum\limits_{n=1}^{N}(a_n\cdot \cos{(nx)}+b_n\cdot \sin{(nx)})=\\
        =\frac{1}{2\pi}\cdot \int\limits_{-\pi}^{\pi}f(t)\ dt+\frac{1}{\pi}\cdot \sum\limits_{n=1}^{N}\left(\cos{(nx)}\cdot \int\limits_{-\pi}^{\pi}f(t)\cos{(nt)}\ dt+\sin{(nx)}\cdot \int\limits_{-\pi}^{\pi}f(t) \sin{(nt)}\ dt\right)=\\
        =\frac{1}{2\pi}\cdot \int\limits_{-\pi}^{\pi}f(t)\ dt+\frac{1}{\pi} \int\limits_{-\pi}^{\pi}f(t)\cdot \sum\limits_{n=1}^{N}(\cos{(nx)}\cos{(nt)}+\sin{(nx)}\sin{(nt)})\ dt=\\
        =\frac{1}{2\pi}\cdot \int\limits_{-\pi}^{\pi}f(t)\ dt+\frac{1}{\pi}\cdot \int\limits_{-\pi}^{\pi}f(t)\cdot \sum\limits_{n=1}^{N}\cos{(n(x-t))}\ dt=\frac{1}{\pi}\cdot \int\limits_{-\pi}^{\pi}f(t)\cdot D_N(x-t)\ dt\overset{(1)}{=}\\
        \overset{(1)}{=}\frac{1}{\pi}\cdot \int\limits_{x-\pi}^{x+\pi}f(x-\tau)D_N(\tau)\ d\tau=\frac{1}{\pi}\cdot \int\limits_{-\pi}^{\pi}f(x-\tau)D_N(\tau)\ d\tau=\frac{1}{\pi}\cdot \int\limits_{-\pi}^{\pi}f(x+\tau)D_N(\tau)\ d\tau
    \end{multline*}
    (1): Сделаем замену $t=x-\tau$.
    Отсюда
    \[F_N(x)=\frac{1}{2\pi}\cdot \int\limits_{-\pi}^{\pi}(f(x+\tau)+f(x-\tau))D_N(\tau)\ d\tau=\frac{1}{\pi}\cdot \int\limits_{0}^{\pi}(f(x+\tau)+f(x-\tau))D_N(\tau)\ d\tau\]
\end{comm}