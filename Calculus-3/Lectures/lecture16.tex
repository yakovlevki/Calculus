\begin{definition}
    Пусть существуют интегралы
    \[\int\limits_{-\pi}^{\pi}f(x)\ dx,\ \ \int\limits_{-\pi}^{\pi}|f(x)|\ dx\]
    Тогда числа
    \[a_n=\frac{1}{\pi} \int\limits_{-\pi}^{\pi}f(x)\cdot \cos{(nx)}\ dx,\ \ b_n=\frac{1}{\pi} \int\limits_{-\pi}^{\pi}f(x)\cdot \sin{(nx)}\ dx\]
    называются коэффициентами Фурье функции $f(x)$, а ряд
    \[\frac{a_0}{2}+\sum\limits_{n=1}^{\infty}(a_n\cdot \cos{(nx)}+b_n\cdot \sin{(nx)})\]
    называется рядом Фурье функции $f(x)$.
\end{definition}
\begin{comm}Из неравенства
    \[|f(x)|\leq \frac{1+f^2(x)}{2}\]
    следует, что если существует интеграл от квадрата, то существует и от модуля.
\end{comm}
\begin{theorem}
    Пусть существуют интегралы
    \[\int\limits_{-\pi}^{\pi}f(x)\ dx,\ \ \int\limits_{-\pi}^{\pi}f^2(x)\ dx\]
    а также
    \[F_N(x)=\frac{a_0}{2}+\sum\limits_{n=1}^{N}(a_n\cdot \cos{(nx)}+b_n\cdot \sin{(nx)})\]
    Тогда
    \[\int\limits_{-\pi}^{\pi}(f(x)-F_N(x))^2\ dx=\min\limits_{T_N} \int\limits_{-\pi}^{\pi}(f(x)-T_N(x))^2\ dx\]
    где $T_N$ --- тригонометрический полином степени не выше $N$.
\end{theorem}
\begin{proof}
    Пусть \[T_N = \frac{A_0}{2}+\sum\limits_{n=1}^{N} \left( A_n\cdot \cos{(nx)}+B_n\cdot \sin{(nx)} \right) \]
    произвольный тригонометрический полином степени $N$.\\
    Хотим минимизировать
    \[ I = \int\limits_{-\pi}^{\pi}(f(x)-T_N(x))^2\ dx = \int\limits_{-\pi}^{\pi} f^2(x) \ dx + 2\int\limits_{-\pi}^{\pi} f(x) \cdot T_N(x) \ dx + \int\limits_{-\pi}^{\pi} T_N^2(x) \ dx \]
    Вычислим второй и третий интегралы через коэффициенты Фурье:
        \[ \int\limits_{-\pi}^{\pi} f(x) \ dx = \pi \cdot a_0, \ \ \int\limits_{-\pi}^{\pi} f(x) \cdot \cos{(nx)} \ dx = \pi \cdot a_n, \ \ \int\limits_{-\pi}^{\pi} f(x) \cdot \sin{(nx)} \ dx = \pi \cdot b_n\]
    \begin{enumerate}
        \item
        \begin{multline*}
            \int\limits_{-\pi}^{\pi} f(x) \cdot T_N(x) \ dx =\\
            = \frac{A_0}{2} \int\limits_{-\pi}^{\pi} f(x) \ dx\ +\ \sum\limits_{n=1}^{N}\left(A_n \int\limits_{-\pi}^{\pi}f(x)\cdot \cos{(nx)} \ dx + B_n \int\limits_{-\pi}^{\pi}f(x) \cdot \sin{(nx)}\ dx \right)=\\
            = \pi \left( \frac{A_0 \cdot a_0}{2} + \sum\limits_{n=1}^{N}(A_n a_n + B_n b_n)\right)
        \end{multline*}
        \item
        \begin{multline*}
            \int\limits_{-\pi}^{\pi}  T_N^2(x) \ dx =\\
            = \int\limits_{-\pi}^{\pi} \left(\frac{A_0}{2}\right)^2 \ dx + \sum\limits_{n=1}^{N} \int\limits_{-\pi}^{\pi} \left( A_n\cdot \cos{(nx)}+B_n\cdot \sin{(nx)} \right)^2 = \\
            = \pi \cdot \frac{A_0^2}{2} + \pi \cdot \sum\limits_{n=1}^{N} \left(A_n^2 + B_n^2\right)
        \end{multline*}
    \end{enumerate}
    Таким образом
    \[ I = \int\limits_{-\pi}^{\pi} f^2(x) \ dx -2 \pi \left( \frac{A_0 a_0}{2} + \sum\limits_{n=1}^{N}(A_n a_n + B_n b_n)\right) + \pi \cdot \frac{A_0^2}{2} + \pi \cdot \sum\limits_{n=1}^{N} \left(A_n^2 + B_n^2\right) \]
    Преобразуем члены с $A_0, A_n$ и $B_n$:
    \[\pi \cdot \frac{A_0^2}{2} - 2 \pi \cdot \frac{A_0 a_0}{2} = \pi \left( \frac{A_0^2}{2} - A_0 a_0\right) = \pi \left( \frac{1}{2}\left( A_0 - a_0\right)^2 - \frac{1}{2} a_0^2 \right)\]
    \[ \pi A_n^2 - 2\pi A_n a_n = \pi((A_n-a_n)^2 - a_n^2)\]
    \[ \pi B_n^2 - 2\pi B_n b_n = \pi((B_n-b_n)^2 - b_n^2)\]
    Значит
    \begin{multline*}
        I = \int\limits_{-\pi}^{\pi} f^2(x) \ dx + \frac{\pi}{2}\left( A_0 - a_0\right)^2 +\\+\pi \sum\limits_{n=1}^{N}\left( (A_n-a_n)^2 + (B_n-b_n)^2\right) - \frac{\pi}{2} a_0^2 - \pi \sum\limits_{n=1}^{N} (a_n^2 + b_n^2)
    \end{multline*}
    Члены не зависящие от $A_0, A_n$ и $B_n$, обозначим:
    \[C = \int\limits_{-\pi}^{\pi} f^2(x) \ dx - \frac{\pi}{2} a_0^2 - \pi \sum\limits_{n=1}^{N} (a_n^2 + b_n^2)\]
    тогда
    \[ I = C + \frac{\pi}{2}\left(A_0 - a_0\right)^2 + \pi \sum\limits_{n=1}^{N}\left( (A_n-a_n)^2 + (B_n-b_n)^2\right)\]
    Поскольку $\pi > 0$, ясно, что $I$ минимально, когда все квадраты равны нулю. Значит $\forall n$:
    \[A_0 = a_0, \ A_n = a_n, \ B_n = b_n\]
    а это и означает, что $T_N(x) = F_N(x)$.
\end{proof}
\begin{consequense} (Неравенство Бесселя)\\
    \[\frac{a^2_0}{2}+\sum\limits_{n=1}^{N}(a_n^2+b_n^2)\leq \frac{1}{\pi}\int\limits_{-\pi}^{\pi}f^2(x)\ dx\]
\end{consequense}
\begin{proof}
    Из доказательства теоремы, заметим, что
    \[\int\limits_{-\pi}^{\pi}f^2(x)\ dx-\pi\cdot \sum\limits_{n=1}^{N}(a_n^2+b_n^2)-\pi\cdot \frac{a_0^2}{2}\geq 0\]
    это так, поскольку это часть слагаемых вычисленного интеграла от квадрата функции. Отсюда
    \[\frac{a_0^2}{2}+\sum\limits_{n=1}^{N}(a_n^2+b_n^2) \leq\frac{1}{\pi}\cdot\int\limits_{-\pi}^{\pi}f^2(x)\ dx\]
\end{proof}
\begin{consequense}
    При $n\to \infty$ выполнено, что $a_n\to 0,\ b_n\to 0$
\end{consequense}
\begin{proof}
    При $n\to \infty$ ряд сходится, значит, по необходимому условию сходимости, выполнено $a_n\to 0, b_n\to 0$.
\end{proof}
\subsection{Ядро Дирихле} % (fold)
% \label{sub:Ядро Дирихле}

% subsection Ядро Дирихле (end)
\begin{definition}
    $N$-м ядром Дирихле называется
    \[D_N(x)=\frac{1}{2}+\sum\limits_{n=1}^{N}\cos{(nx)}\]
\end{definition}
\begin{theorem} (Свойства ядер Дирихле)\\
    \begin{enumerate}
        \item $D_N$ непрерывная, четная и $2\pi$-периодичная функция.
        \item $D_N(0)=N+\frac{1}{2}$
        \item \[\frac{1}{\pi}\int\limits_{-\pi}^{\pi}D_N(x)\ dx=1\]
        \item $\forall x\ne 2\pi k$.
        \[D_N(x)=\frac{\sin{((N+\frac{1}{2})x)}}{2\cdot \sin{(\frac{x}{2})}}\]
    \end{enumerate}
\end{theorem}
\begin{proof} \tab
    \begin{enumerate} 
        \item Очев
        \item $D_N(0)=\frac{1}{2}+\sum\limits_{n=1}^{N}\cos{0}=N+\frac{1}{2}$
        \item \[\int\limits_{-\pi}^{\pi}D_N(x)\ dx=\int\limits_{-\pi}^{\pi}\ \left(\frac{1}{2}+\sum\limits_{n=1}^{N}\cos{(nx)}\right)=\int\limits_{-\pi}^{\pi}\frac{1}{2}\ dx+\sum\limits_{n=1}^{N}\left(\ \int\limits_{-\pi}^{\pi}\cos{(nx)}\ dx\right)=\pi\]
        % \item Домножим и поделим на $\sin{\frac{x}{2}}$, дальше очев.
        \item Умножим обе части на \(2 \sin\left(\frac{x}{2}\right)\):
        \[2 \sin\left(\frac{x}{2}\right) D_N(x) = \sin\left(\frac{x}{2}\right) + \sum_{n=1}^N 2 \cos(nx) \sin\left(\frac{x}{2}\right)\]
        \[2 \cos(nx) \sin\left(\frac{x}{2}\right) = \sin\left(nx + \frac{x}{2}\right) - \sin\left(nx - \frac{x}{2}\right)\]
        Подставляем в сумму:
        \[2 \sin\left(\frac{x}{2}\right) D_N(x) = \sin\left(\frac{x}{2}\right) + \sum_{n=1}^N \left( \sin\left((n+\frac{1}{2})x\right) - \sin\left((n-\frac{1}{2})x\right)\right)\]
        Все промежуточные слагаемые сокращаются, остаётся:
        \[2 \sin\left(\frac{x}{2}\right) D_N(x) = \sin\left((N+\frac{1}{2})x\right)\]
        Отсюда, при \(x \neq 2\pi k\):
        \[D_N(x) = \frac{\sin\left((N+\frac{1}{2})x\right)}{2 \sin\left(\frac{x}{2}\right)}\]
    \end{enumerate}
\end{proof}
\begin{comm}
    \begin{multline*}
        F_N(x)=\frac{a_0}{2}+\sum\limits_{n=1}^{N}(a_n\cdot \cos{(nx)}+b_n\cdot \sin{(nx)})=\\
        =\frac{1}{2\pi}\cdot \int\limits_{-\pi}^{\pi}f(t)\ dt+\frac{1}{\pi}\cdot \sum\limits_{n=1}^{N}\left(\cos{(nx)}\cdot \int\limits_{-\pi}^{\pi}f(t)\cos{(nt)}\ dt+\sin{(nx)}\cdot \int\limits_{-\pi}^{\pi}f(t) \sin{(nt)}\ dt\right)=\\
        =\frac{1}{2\pi}\cdot \int\limits_{-\pi}^{\pi}f(t)\ dt+\frac{1}{\pi} \int\limits_{-\pi}^{\pi}\left(f(t)\cdot \sum\limits_{n=1}^{N}(\cos{(nx)}\cos{(nt)}+\sin{(nx)}\sin{(nt)})\right) dt=\\
        =\frac{1}{2\pi}\cdot \int\limits_{-\pi}^{\pi}f(t)\ dt+\frac{1}{\pi}\cdot \int\limits_{-\pi}^{\pi}\left(f(t)\cdot \sum\limits_{n=1}^{N}\cos{(n(x-t))}\right) dt=\frac{1}{\pi}\cdot \int\limits_{-\pi}^{\pi}f(t)\cdot D_N(x-t)\ dt\overset{(1)}{=}
    \end{multline*}
    \begin{multline*}
        \overset{(1)}{=}\frac{1}{\pi}\cdot \int\limits_{x-\pi}^{x+\pi}f(x-\tau)D_N(\tau)\ d\tau=\frac{1}{\pi}\cdot \int\limits_{-\pi}^{\pi}f(x-\tau)D_N(\tau)\ d\tau=\frac{1}{\pi}\cdot \int\limits_{-\pi}^{\pi}f(x+\tau)D_N(\tau)\ d\tau
    \end{multline*}
    (1): Сделаем замену $t=x-\tau$.
    Отсюда
    \[F_N(x)=\frac{1}{2\pi}\cdot \int\limits_{-\pi}^{\pi}(f(x+\tau)+f(x-\tau))D_N(\tau)\ d\tau=\frac{1}{\pi}\cdot \int\limits_{0}^{\pi}(f(x+\tau)+f(x-\tau))D_N(\tau)\ d\tau\]
\end{comm}