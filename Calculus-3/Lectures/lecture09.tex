\begin{consequense} (Формула Валлиса)\\
    При $t=\frac{\pi}{2}$
    \[1=\frac{\pi}{2}\cdot \prod\limits_{k=1}^{\infty}\left(1-\frac{1}{4k^2}\right) \Rightarrow \frac{2}{\pi}=\prod\limits_{k=1}^{\infty}\left(\frac{4k^2-1}{4k^2}\right)\]
\end{consequense}
\newpage
\section{Интегралы с параметром}
\subsection{Собственные интегралы с параметром}
    Пусть $\phi(y),\ \psi(y)\in \mathcal{C}[a,b],\ \phi(y)\leq \psi(y)$. Рассмотрим $G\subset \R^2: $
    \[G=\{(x,y): y\in [a,b],\ \phi(y)\leq x\leq \psi(y)\}\]
\[\begin{tikzpicture}[scale=1.7]
\draw[->] (-0.5, 0) -- (4.5,0) node[right] {$x$};
\draw[->] (0,-0.5) -- (0,3.5) node[above] {$y$};

\pgfmathdeclarefunction{phi}{1}{%
    \pgfmathparse{0.8 + 0.4*sin(deg(1.2*#1)) + 0.2*cos(deg(3*#1))}%
}
\pgfmathdeclarefunction{psi}{1}{%
    \pgfmathparse{3.2 + 0.5*sin(deg(0.8*#1)) + 0.3*cos(deg(2.5*#1)) + 0.1*sin(deg(5*#1))}%
}

% Первая кривая φ(y)
\draw[thick] plot[domain=0.3:3.2,samples=100] ({phi(\x)}, \x);
\node[left] at ({phi(1.7)}, 1.7) {$\phi(y)$};

% Вторая кривая ψ(y)
\draw[thick] plot[domain=0.3:3.2,samples=100] ({psi(\x)}, \x);
\node[right] at ({psi(1.7)}, 1.7) {$\psi(y)$};

% Прямые, ограничивающие сверху и снизу
\draw[dashed] (0,0.3) -- (4,0.3);
\draw[dashed] (0,3.2) -- (4,3.2);

\node[left] at (0,0.3) {$a$};
\node[left] at (0,3.2) {$b$};

% Подпись области
\node at (2.2,1.7) {$G$};
\end{tikzpicture}\]
\begin{definition} Пусть $f(x,y)\in \mathcal{R}[\phi(y),\psi(y)]$. Интеграл
    \[F(y)=\int\limits_{\phi(y)}^{\psi(y)}f(x,y)\ dx\]
    называется собственным интегралом с параметром.
\end{definition}