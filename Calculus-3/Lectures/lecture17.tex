\begin{theorem} (Принцип локализации Римана)\\
    Пусть существуют интегралы
    \[\int\limits_{-\pi}^{\pi}f(x)\ dx,\ \ \int\limits_{-\pi}^{\pi}f^2(x)\ dx\]
    Тогда существование предела
    \[\lim\limits_{N\to\infty}\left(\frac{a_0}{2}+\sum\limits_{n=1}^{N}(a_n\cdot \cos{(nx_0)}+b_n\cdot \sin{(nx_0)})\right)=A\]
    равносильно тому, что $\forall \delta>0$ существует 
    \[\lim\limits_{N\to\infty}\frac{1}{\pi}\int\limits_{0}^{\delta}(f(x_0+t)+f(x_0-t))D_N(t)\ dt=A\]
\end{theorem}
\begin{proof}
    \[F_N(x_0)=\frac{1}{\pi}\cdot \int\limits_{0}^{\pi}(f(x_0+t)+f(x_0-t))D_N(t)\ dt\]
    значит, формулировка теоремы равносильна утверждению, что $\forall \delta>0$
    \[\lim\limits_{N\to\infty}\int\limits_{\delta}^{\pi}(f(x_0+t)+f(x_0-t))D_N(t)\ dt=0\]
    Выразим ядро Дирихле в явной форме:
    \begin{multline*}
        D_N(t) =\frac{\sin{((N+\frac{1}{2})t)}}{2\cdot \sin{(\frac{t}{2})}} = \frac{\sin{(Nt)}\cos{(\frac{t}{2})} + \cos{(Nt)}\sin{(\frac{t}{2})}}{2\cdot \sin{(\frac{t}{2})}} = \\
        = \frac{1}{2}\left(\cos{(Nt)} + \ctg{(\frac{t}{2})} \sin{(Nt)} \right)
    \end{multline*}
    \begin{multline*}
        \frac{1}{\pi}\cdot \int\limits_{\delta}^{\pi}\cdot\frac{f(x_0+t)+f(x_0-t)}{2}\cdot \cos{(Nt)}\ dt+\\
        +\frac{1}{\pi}\cdot \int\limits_{\delta}^{\pi}\frac{f(x_0+t)+f(x_0-t)}{2}\cdot \ctg{(\frac{t}{2})}\sin{(Nt)}\ dt = I_1 + I_2
    \end{multline*}
    Определим две вспомогательные функции на $[-\pi, \pi]$:
    \[h(t)=\begin{cases}
        \frac{f(x_0+t)+f(x_0-t)}{2},\ t\in [\delta, \pi]\\
        0,\ t\in [-\pi, \delta]
    \end{cases}\]
    \[g(t)=\begin{cases}
        \frac{f(x_0+t)+f(x_0-t)}{2}\cdot \ctg{\frac{t}{2}},\ t\in [\delta, \pi]\\
        0,\ t\in [-\pi,\delta)
    \end{cases}\]
    Тогда $I_1$ и $I_2$ --- в точности коэффициэнты фурье этих функций, а они стремятся к нулю.
\end{proof}
\begin{theorem} (Признак Дини поточечной сходимости ряда Фурье)\\
    Пусть $f(x)$ - $2\pi$ периодическая, существуют интегралы 
    \[\int\limits_{-\pi}^{\pi}f(x)\ dx,\ \ \int\limits_{-\pi}^{\pi}f^2(x)\ dx\]
    для $x_0\in [-\pi,\pi]$ существуют пределы
    \[\lim\limits_{x\to x_0-0}f(x)=f_-(x_0),\ \ \lim\limits_{x\to x_0+0}f(x)=f_+(x_0)\]
    Если $\exists\ \delta>0$ такой, что существует интеграл
    \[\int\limits_{0}^{\delta}\frac{|f(x_0+t)+f(x_0-t)-f_+(x_0)-f_-(x_0)|}{t}\ dt\]
    то ряд Фурье функции $f(x)$ сходится в точке $x_0$ и равен $\frac{f_+(x_0)+f_-(x_0)}{2}$
\end{theorem}
\begin{proof}
    \begin{multline*}
    \frac{1}{\pi}\int\limits_{0}^{\pi}(f(x_0+t)+f(x_0-t))D_N(t)\ dt-\frac{f_+(x_0)+f_-(x_0)}{2}\cdot \frac{1}{\pi}\cdot \int\limits_{-\pi}^{\pi}D_N(t)\ dt =\\
    =\frac{1}{\pi}\cdot \int\limits_{0}^{\pi}(f(x_0+t)+f(x_0-t)-f_+(x_0)-f_-(x_0))D_N(t)\ dt=\\
    =\frac{1}{\pi}\cdot \int\limits_{0}^{\delta}(f(x_0+t)+f(x_0-t)-f_+(x_0)-f_-(x_0))\cdot \frac{\sin{((N+\frac{1}{2})t)}}{2\sin{\frac{t}{2}}}\ dt+\\
    \frac{1}{\pi}\cdot \int\limits_{\delta}^{\pi}(f(x_0+t)+f(x_0-t)-f_+(x_0)-f_-(x_0))\cdot \frac{\sin{((N+\frac{1}{2})t)}}{2\sin{\frac{t}{2}}}\ dt=I_1+I_2
    \end{multline*}
    \[|I_1|\leq \frac{1}{\pi}\cdot \int\limits_{0}^{\delta}\frac{|f(x_0+t)+f(x_0-t)-f_+(x_0)-f_-(x_0)|}{2\sin{\frac{t}{2}}}\ dt< \epsilon\]
    $|I_2|<\epsilon$ при большом $N$ (из принципа локализации Римана).
\end{proof}
% \begin{consequense}
%     Если $f(x)$ в точке $x_0$ односторонне Гёльдерова, то теорема выполнена. Где Гёльдерова функция - это функция, для которой
%     \[|f(x)-f(y)|\leq C\cdot |x-y|^{\alpha}\]
% \end{consequense}
\begin{definition}
    Функция $f(x)$ называется Гёльдеровой, если $\exists\ C>0$:
    \[|f(x)-f(y)|\leq C\cdot |x-y|^{\alpha}\]
\end{definition}
\begin{definition}
    Для функции $f(x)$ выполнено условие Гёльдера в точке $x_0$  для односторонних пределов, если
    $\exists\ C>0, \alpha>0, \delta > 0: \forall t \in (0,\delta)$:
    \[|f(x_0 + t) - f_{+}(x_0)| \leq Ct^\alpha, \ |f(x_0 - t) - f_{-}(x_0)| \leq Ct^\alpha\]
\end{definition} 
\begin{consequense}
    Если функция удовлетворяет условию Гёльдера в точке $x_0$ (хотя бы односторонне), то выполнен признак Дини, значит ряд Фурье в этой точке сходится к среднему арифметическому односторонних пределов.
\end{consequense} 