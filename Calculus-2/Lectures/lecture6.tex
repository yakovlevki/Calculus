\subsection{Формула Ньютона-Лейбница}
\begin{theorem} (Формула Ньютона-Лейбница)\\
    Пусть $f(x)\in \mathcal{R}[a,b],\ f(x)\in \mathcal{C}([a,b]\setminus \{x_i\}_{i=1}^{n})$.
    \[\exists\ F(x): \ F(x)\in \mathcal{D}([a,b]\setminus \{x_i\}_{i=1}^{n}),\ F'(x)=f(x),\ F(x)\in \mathcal{C}[a,b]\] 
    Тогда: 
    \[\int\limits_{a}^{b}f(x)\ dx=F(b)-F(a)\]
\end{theorem} 
\begin{proof}
    Пусть сначала $f(x)\in \mathcal{C}(a,b),\ F'(x)=f(x)$ на $(a,b)$. Но интеграл
    \[\int\limits_{a}^{x}f(t)\ dt\]
    тоже первообразная $f(x)$ на $(a,b) \Rightarrow \exists\ C:$
    \[F(x)+C=\int\limits_{a}^{x}f(t)\ dt\]
    $\Rightarrow F(a)+C=0$. Тогда 
    \[F(b)-F(a)=\int\limits_{a}^{b} f(t)\ dt\]
    Общий случай:
        \[F(b)-F(a)=\sum\limits_{i=1}^{n-1}(F(x_{i+1})-F(x_{i}))=\sum\limits_{i=1}^{n-1} \int\limits_{x_{i-1}}^{x_i} f(t)\ dt=\int\limits_{a}^{b} f(t)\ dt\]
\end{proof} 
\subsection{Замена переменной и интегрирование по частям}
\begin{theorem}
    Пусть $f(x)\in \mathcal{C}(a,b),\ \phi(t)\in \mathcal{C}^1(\alpha, \beta),\ \phi((\alpha, \beta))\subset (a,b).\\
    \forall \alpha_0, \beta_0\in (\alpha, \beta)$ и $a_0=\phi(\alpha_0),\ b_0=\phi(\beta_0)$. Тогда
    \[\int\limits_{a_0}^{b_0}f(x)\ dx=\int\limits_{\alpha_0}^{\beta_0}f(\phi(t))\cdot \phi'(t)\ dt\]
\end{theorem} 
\begin{proof}
    $f\in \mathcal{C}(a,b) \Rightarrow \exists\ F'(x)=f(x)$
    \[\int\limits_{a_0}^{b_0}f(x)\ dx= F(b_0)-F(a_0)\]
    Но $(F(\phi(t)))'=F'(\phi(t))\cdot \phi'(t)$, а значит
    \[\int\limits_{\alpha_0}^{\beta_0}f(\phi(t))\cdot\phi'(t)\ dt=F(\phi(\beta_0))-F(\phi(\alpha_0))\]
\end{proof} 
\begin{theorem} (Интегрирование по частям)\\
    Пусть $f(x),\ g(x)\in \mathcal{C}^1[a,b]$
    \[\int\limits_{a}^{b}f(x)\cdot g'(x)\ dx=f(x)\cdot g(x)|_a^b-\int\limits_{a}^{b}f'(x)g(x)\ dx\]
\end{theorem} 
\begin{proof}
    \[f(x)\cdot g(x)|_a^b=\int\limits_{a}^{b}(f(x)\cdot g(x))'\ dx=\int\limits_{a}^{b}f'(x)g(x)\ dx+\int\limits_{a}^{b}f(x)\cdot g'(x)\ dx\]
\end{proof} 
\newpage
\section{Спрямляемые кривые и квадрируемые фигуры}
\subsection{Кривая в \texorpdfstring{$\R^n$}{Rn}}

\begin{definition}
    Кривой в $\R^n$ называется непрерывное отображение:
    \[\bar{\gamma}: [a,b]\to \R^n\]
\end{definition} 
\begin{comm}
    \[\bar{\gamma}=\begin{pmatrix}
       \gamma_1(t)\\
       \vdots\\
       \gamma_n(t) 
    \end{pmatrix}\]
\end{comm} 
% Пример того, что окружность можно задать параметрически разными способами
\begin{definition}
    Рассмотрим $\bar{\gamma}: [a,b]\to \R^n$. Если $\exists\ t_1\ne t_2: \bar{\gamma}(t_1)=\bar{\gamma}(t_2)$, то $\bar{\gamma}(t_1)$ называется точкой самопересечения. Мощность подмножеста $[a,b]$, точки которого переходят в $\bar{\gamma}(t_1)$ называется кратностью точки самопересечения. Если кривая не имеет точек самопересечения, то она называется простой.
\end{definition} 
\begin{definition}
    Если $\bar{\gamma}(t)$ имеет единственную точку самопересечения\\
    $\bar{\gamma}(a)=\bar{\gamma}(b)$, то кривая называется простой замкнутой.
\end{definition} 
\begin{definition}
    Множество точек $\{\bar{\gamma}(t_i)\}_{i=0}^N$ называется разбиением кривой, если $\{t_i\}_{i=0}^N$ является разбиением отрезка $[a,b]$. Обозначается $T_{\gamma}$.
\end{definition} 
\begin{definition}
    $L(T_{\bar{\gamma}})$ - множество отрезков $\{[\bar{\gamma}(t_{i-1}),\bar{\gamma}(t_i)]\}_{i=1}^N$ называется вписанной в $\bar{\gamma}(t)$ ломаной, а число $|L(T_{\bar{\gamma}})|$ - длиной ломаной.
\end{definition} 
\begin{statement}
    Если $T'_{\bar{\gamma}}$ - измельчение $T_{\bar{\gamma}}$, то 
    \[|L(T_{\bar{\gamma}})|\leq |L(T'_{\bar{\gamma}})|\]
\end{statement} 
\begin{proof}
    Очевидно.
\end{proof} 
\begin{definition}
    Если множество $\{|L(T_{\bar{\gamma}})|\}_{T_{\bar{\gamma}}}$ ограничено, то кривая $\bar{\gamma}(t)$ называется спрямляемой, а
    \[\sup\limits_{T_{\bar{\gamma}}}\{|L(T_{\bar{\gamma}})|\}=|\bar{\gamma}|\]
    называется длиной кривой.
\end{definition} 