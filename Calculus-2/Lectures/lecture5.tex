\begin{numtheorem}
    $f(x) \in \mathcal{R}[a, b]$ и $f(x) \geq \delta > 0$. Тогда $\frac{1}{f(x)} \in \mathcal{R}[a, b]$
\end{numtheorem}
\begin{proof}
    $\forall x', x'' \in [a, b]$:
    \begin{align*}
        \vline\ \frac{1}{f(x')} - \frac{1}{f(x'')}\ \vline =\ \vline \frac{f(x'') - f(x')}{f(x')f(x'')}\ \vline \leq \frac{1}{\delta^2}\cdot\ \vline f(x'')-f(x')\ \vline
    \end{align*}
    Дальнейшее доказательство аналогично предыдущему (\textit{на всякий случай приведём аналогичную выкладку, необходимую для доказательства})
    \begin{multline*}
        M_i(\frac{1}{f(x)})-m_i(\frac{1}{f(x)})=\sup(\frac{1}{f(x')}-\frac{1}{f(x'')})\leq\\
        \leq\frac{1}{\delta^2}\sup|f(x'')-f(x')|=\frac{1}{\delta^2}(M_{if}-m_{if})
    \end{multline*}
\end{proof}
\begin{consequense}
    Из пунктов 6 и 7 следует интегрируемость дроби $\frac{f(x)}{g(x)}$.
\end{consequense}
\begin{numtheorem}
    $f(x) \in \mathcal{R}[a, b]$. Тогда $|f(x)| \in \mathcal{R}[a, b]$
\end{numtheorem}
\begin{proof}
    $\forall x', x'' \in [a, b]$:
    \begin{align*}
        \vline\ |f(x')| - |f(x'')|\ \vline \leq |f(x')-f(x'')|
    \end{align*}
    Далее совпадает с предыдущим доказательством.
\end{proof}
\begin{comm}
    Обратное утверждение неверно:
    \[f(x) = \begin{cases}
        \tab[0.35cm] 1, \ x\in\mathbb{Q}\subset [0, 1]\\
        -1,\ x\notin\mathbb{Q}
    \end{cases}\]
    $\Rightarrow |f(x)| \equiv 1$ на отрезке $[0, 1]$.
\end{comm}
\begin{numtheorem}
    $f(x) \in \mathcal{R}[a, b]$. Тогда 
    \[\vline\ \int\limits_{a}^{b}f(x)\ dx\ \vline\leq \int\limits_{a}^{b} |f(x)|\ dx\]
\end{numtheorem}
\begin{proof}
    \[|\sigma_f| \leq \sigma_{|f|}\]
\end{proof}
\begin{comm}
    \[\int\limits_{a}^{b}|f(x)|\ dx \leq \sup \limits_{[a, b]} |f(x)|\cdot\int\limits_{a}^{b}1\ dx\]
\end{comm}
\setcounter{thmcount}{0}
\subsection{Первая теорема о среднем}

\begin{theorem} (Первая теорема о среднем)\\
    Пусть $f(x), g(x) \in \mathcal{R}[a, b],\ g(x) \geq 0,\ M = \sup f(x),\ m = \inf f(x)$. Тогда\\
    $\exists\ \mu\in [m,M]$:
    \[\int\limits_{a}^{b}f(x)\cdot g(x)\ dx = \mu\cdot\int\limits_{a}^{b}g(x)\ dx\]
\end{theorem}
\begin{proof}
    \[m\cdot \sigma_g(T) \leq \sigma_{f\cdot g}(T) \leq M\cdot\sigma_g(T)\]
    Тогда
    \[m\cdot \int\limits_{a}^{b} g(x)\ dx \leq \int\limits_{a}^{b} f(x)\cdot g(x)\ dx \leq M\cdot\int\limits_{a}^{b} g(x)\ dx\]
    Рассмотрим случаи:
    \begin{enumerate}
        \item 
        \[\int\limits_{a}^{b} g(x)\ dx = 0 \Rightarrow \int\limits_{a}^{b} f(x)\cdot g(x)\ dx = 0\]
        В этом случае равенство верно для любого $\mu$.
        \item
        \[\int\limits_{a}^{b} g(x)\ dx \ne 0 \Rightarrow m\leq \cfrac{\int\limits_{a}^{b} f(x)\cdot g(x)\ dx}{\int\limits_{a}^{b} g(x)\ dx}\leq M\]
        Значит, подойдет $\mu$, равное значению этой дроби
    \end{enumerate}
\end{proof}
\subsection{Интеграл с переменным верхним пределом}
\begin{definition}
    Интегралом с переменным верхним пределом называется интеграл вида:
    \[\int\limits_{a}^{x} f(t)\ dt\]
\end{definition} 
\begin{theorem}
    Пусть $f(t)\in \mathcal{R}[a, b]$. Тогда функция
    \[\phi(x) = \int\limits_{a}^{x} f(t)\ dt\]
    непрерывна на $[a, b]$.
\end{theorem}
\begin{proof} 
    $\forall x_0\in [a,b]$ и $\Delta x\to 0:$
    \[|\phi(x_0 + \Delta x) - \phi(x_0)| =\ \vline\int\limits_{x_0}^{x_0 + \Delta x} f(t)\ dt\ \vline \leq M_{f([a, b])}\cdot |\Delta x| \rightarrow 0\]
\end{proof}
\begin{theorem}
    Пусть $f(x)\in \mathcal{R}[a, b]$ и $f$ непрерывна в $x_0 \in [a, b]$. Тогда функция
    \[\phi(x) = \int_{a}^{x} f(t)\ dt\]
    имеет производную в $x_0$ и $\phi'(x_0) = f(x_0)$.
\end{theorem}
\begin{proof}
    \begin{multline*}
        \vline\ \frac{\phi(x_0 + \Delta x) - \phi(x_0)}{\Delta x} - f(x_0)\ \vline =\\=\ \vline\frac{1}{\Delta x}\cdot \int\limits_{x_0}^{x_0 + \Delta x} f(x)\ dx - \frac{f(x_0)}{\Delta x} \cdot \int\limits_{x_0}^{x_0 + \Delta x} 1\ dx\ \vline =\tab[3cm]\\=\ \vline\frac{1}{\Delta x}\cdot \int\limits_{x_0}^{x_0 + \Delta x} (f(x) - f(x_0))\ dx\ \vline \leq \sup \limits_{[x_0, x_0+\Delta x]}|f(x) - f(x_0)|\cdot 1 \longrightarrow 0.
    \end{multline*}
\end{proof}
\begin{consequense}
    Пусть $f(x) \in \mathcal{C}(a, b)$. Тогда $\forall c \in (a, b)$:
    \[\exists \ \left(\int\limits_{c}^{x} f(t)dt\right)' = f(x), \text{ то есть } \int\limits_{c}^{x} f(t)dt\ \text{-}\ \text{первообразная} \ f(x)\]
\end{consequense}
\begin{proof}
    Очевидно.
\end{proof}
\begin{comm}
    Интервал в формулировке следствия взят для применимости теоремы к неограниченным на интервале функциям (например $\tg(x)$ на $[0, \pi]$), для которых тем не менее применима предыдущая теорема по аналогичным рассуждениям.
\end{comm}
