\begin{numtheorem}
    $f(x) \in \mathcal{R}[a, b]$ и $f(x) \geq \delta > 0$. Тогда $\frac{1}{f(x)} \in \mathcal{R}[a, b]$
\end{numtheorem}
\begin{proof}
    $\forall x', x'' \in [a, b]$:
    \begin{align*}
        |\frac{1}{f(x')} - \frac{1}{f(x'')}| = |\frac{f(x'') - f(x')}{f(x')f(x'')}| \leq \frac{1}{\delta^2}|f(x'')-f(x')|
    \end{align*}
    Дальнейшее доказательство аналогично предыдущему (\textit{возможно, распишем подробнее}).
\end{proof}
\begin{comm}
    Из пунктов 6 и 7 следует интегрируемость дроби $\frac{f(x)}{g(x)}$.
\end{comm}
\begin{numtheorem}
    $f(x) \in \mathcal{R}[a, b]$. Тогда $|f(x)| \in \mathcal{R}[a, b]$
\end{numtheorem}
\begin{proof}
    $\forall x', x'' \in [a, b]$:
    \begin{align*}
        ||f(x')| - |f(x'')|| = \leq |f(x')-f(x'')|
    \end{align*}
    Далее совпадает с предыдущим доказательством.
\end{proof}
\begin{comm}
    Обратное утверждение неверно:\\
    $f = \begin{cases}
        1, \ x\in\mathbb{Q}\subset [0, 1]\\
        -1, x\notin\mathbb{Q}
    \end{cases} \Rightarrow |f(x)| \equiv 1$ на отрезке $[0, 1]$
\end{comm}
\begin{numtheorem}
    $f(x) \in \mathcal{R}[a, b]$. Тогда $|\int_{a}^{b}f dx|\leq \int_{a}^{b}|f| dx$
\end{numtheorem}
\begin{proof}
    $|\sigma_f| \leq \sigma_{|f|}$
\end{proof}
\begin{comm}
    $\int_{a}^{b}|f| dx \leq \sup \limits_{[a, b]} |f|\cdot\int_{a}^{b}1 dx$
\end{comm}
\setcounter{thmcount}{0}
\begin{theorem}[1-я теорема о среднем]
    Пусть $f(x), g(x) \in \mathcal{R}[a, b], g \geq 0$.\\
    $M = \sup f, m = \inf f$. Тогда $\exists \mu: m\leq\mu\leq M: \int_{a}^{b}f\cdot g dx = \mu\cdot\int_{a}^{b}g dx$
\end{theorem}
\begin{proof}
    $m\cdot \sigma_g(T) \leq \sigma_{fg}(T) \leq M\cdot\sigma_g(T)$. Тогда:
    \[m\cdot \int_{a}^{b} g dx \leq \int_{a}^{b} fg dx \leq M\cdot\int_{a}^{b} g dx\]
    Рассмотрим случаи:
    \[1. \int_{a}^{b} g dx = 0 \Rightarrow \int_{a}^{b} fg dx = 0 \ \ \text{с любым} \ \mu;\]
    \[2. \int_{a}^{b} g dx = 0 \Rightarrow m\leq \frac{\int_{a}^{b} fg dx}{\int_{a}^{b} g dx}\leq M \Rightarrow \ \mu, \text{ равное дроби, подойдёт.}\]
\end{proof}
\subsection{Интеграл с переменным верхним пределом}
\[\int_{a}^{x} f(x) dx\]
\begin{theorem}
    Пусть $f(x)\in \mathcal{R}[a, b]$. Тогда $\int_{a}^{x} f(t) dt = F(x) \in \mathcal{C}[a, b]$.
\end{theorem}
\begin{proof}
    \[\forall x_0 \ \ |F(x_0 + \Delta x) - F(x_0)| = |\int_{x_0}^{x_0 + \Delta x} f(t) dt| \leq M_{f([a, b])}\cdot |\Delta x| \rightarrow 0\]
\end{proof}
\begin{theorem}
    Пусть $f(x)\in \mathcal{R}[a, b]$ и $f$ непрерывна в $x_0 \in [a, b]$. Тогда $\int_{a}^{x} f(t) dt = F(x)$ имеет производную в $x_0$ и $F'(x_0) = f(x_0)$.
\end{theorem}
\begin{proof}
    \begin{multline*}
        |\frac{F(x_0 + \Delta x) - F(x_0)}{\Delta x} - f(x_0)| =\\= |\frac{1}{\Delta x}\int_{x_0}^{x_0 + \Delta x} f(x) dx - f(x_0)\cdot\frac{1}{\Delta x}\int_{x_0}^{x_0 + \Delta x} 1\ dx| =\\= |\frac{1}{\Delta x}\int_{x_0}^{x_0 + \Delta x} (f(x) - f(x_0)) dx| \leq \sup \limits_{[a, b]}|f(x) - f(x_0)|\cdot 1 \longrightarrow 0.
    \end{multline*}
\end{proof}
\begin{consequense}
    Пусть $f(x) \in \mathcal{C}(a, b)$. Тогда $\forall c \in (a, b)$
    \[\exists \ (\int_{c}^{x} f(t)dt)' = f(x), \text{ то есть } \int_{c}^{x} f(t)dt - \ \text{первообразная} \ f(x)\].
\end{consequense}
\begin{proof}
    Очевидно.
\end{proof}
\begin{comm}
    Интервал в формулировке следствия взят для применимости теоремы к неограниченным на интервале функциям (напр. $\tg(x)$ на $[0, \pi]$), для которых тем не менее применима предыдущая теорема по аналогичным рассуждениям.
\end{comm}