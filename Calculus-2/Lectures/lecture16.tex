\begin{theorem}
    $f(\bar{x})$ дифференцируема в точке $\bar{x}_0 \Rightarrow f(\bar{x})$ непрерывна в точке $\bar{x}_0$.
\end{theorem} 
\begin{proof}
    \begin{multline*}
        |f(\bar{x})-f(\bar{x}_0)|=\left|\sum\limits_{i=1}^{n}A_i \Delta x_i +\bar{\bar{o}}{((\rho))}\right|\leq\\
        \leq \sum\limits_{i=1}^{n}|A_i|\cdot|\Delta x_i|+|\bar{\bar{o}}{(\rho)}|\leq \epsilon\cdot \sum\limits_{i=1}^{n}|A_i|+\epsilon
    \end{multline*}
\end{proof} 
\begin{theorem}
    Если $f(\bar{x})$ дифференцируема в точке $\bar{x}_0$, то в точке $\bar{x}_0$:
    \[\forall i=1, \dots, n:\ \exists\ \frac{\partial {f}}{\partial {x_i}}=A_i\]
\end{theorem} 
\begin{proof}
    \begin{multline*}
        \lim\limits_{\Delta x_i\to 0}\frac{f(x_{01},\dots,x_{0i}+\Delta x_i,\dots,x_{0n})-f(\bar{x}_0)}{\Delta x_i}=\\
        =\lim\limits_{\Delta x_i\to 0}\frac{A_i \Delta x_i+\bar{\bar{o}}{(|\Delta x_i|)}}{\Delta x_i}=A_i=\frac{\partial {f}}{\partial {x_i}}(x_0)
    \end{multline*}
    Если $f(\bar{x})=x_i$, то $f(\bar{x})-f(\bar{x}_0)=\Delta x_i=df=dx_i$
    отсюда
    \[df=\sum\limits_{i=1}^{n} \frac{\partial {f}}{\partial {x_i}}(\bar{x}_0)\ dx_i\]
\end{proof} 
\begin{theorem} (Достаточное условие дифференцируемости)\\
    Пусть $\forall i=1,\dots, n$ в некоторой $B(\bar{x}_0)$ существуют $\frac{\partial {f}}{\partial {x_i}}$ и они непрерывны в $B(\bar{x}_0)$. Тогда $f(\bar{x})$ дифференцируема в точке $\bar{x}_0$.
\end{theorem} 
\begin{proof}
    Докажем в $\R^2$.
    \begin{multline*}
        f(x_0+\Delta x, y_0+\Delta y)-f(x_0, y_0)=\\
        =f(x_0+\Delta x, y_0+\Delta y)-f(x_0, y_0+\Delta y)+f(x_0, y_0+\Delta y)-f(x_0, y_0)= (1)\\
        =f'_x(x_0+\theta_1 \Delta x, y_0+\Delta y)\Delta x+f'_y(x_0,y_0+\theta_2 \Delta y) \Delta y= (2)+(3)\\
        =f'_x(x_0,y_0)\Delta x+f'_y(x_0,y_0)\Delta y+\bar{\bar{o}}{(1)}\Delta x+\bar{\bar{o}}{(1)} \Delta y= (4)\\
        =f'_x(x_0,y_0)\Delta x+f'_y(x_0,y_0)\Delta y+\bar{\bar{o}}{(\rho)}
    \end{multline*}
    где $0 < \theta_1, \theta_2 < 1$. Пояснения:\\
    (1):\ По теореме Лагранжа\\
    (2):\ Из непрерывности частных производных:
    \[f'_x(x_0+\theta_1 \Delta x, y_0+\Delta y)=f'_x(x_0,y_0)+\bar{\bar{o}}{(1)},\ \rho \to 0\]
    (3):\ Из непрерывности частных производных:
    \[f'_y(x_0, y_0+\theta_2 \Delta y)=f'_y(x_0,y_0)+\bar{\bar{o}}{(1)},\ \rho \to 0\]
    (4):\ \[\frac{\bar{\bar{o}}{(1)}\Delta x+\bar{\bar{o}}{(1)} \Delta y}{\rho}=\bar{\bar{o}}{(1)}\]
\end{proof} 
\begin{comm}
    Аналогичное доказательство можно провернуть для $\R^n$.
\end{comm}
\begin{definition}
    Пусть $f(\bar{x})$ определена в $B(\bar{x}_0)$. Обозначим $y_0=f(\bar{x}_0)$, плоскость
    \[y-y_0=\sum\limits_{i=1}^{n}A_i \Delta x_i\]
    называется касательной к графику $y=f(\bar{x})$ в точке $\bar{x}_0$, если для любой прямой, проходящей через точки $(\bar{x}_0, f(\bar{x}_0))$ и $(\bar{x}, f(\bar{x}))$, ее угол с нормалью к плоскости стремится к $\frac{\pi}{2}$ при $\bar{x}\to \bar{x}_0$.
\end{definition} 
\begin{theorem} (Геометрический смысл дифференциала и уравнение касательной)\\
    Если $f(\bar{x})$ дифференцируема в точке $\bar{x}_0$, то 
    \[y-y_0=\sum\limits_{i=1}^{n}\frac{\partial {f}}{\partial {x_i}}(\bar{x}_0)(x_i-x_{0i})\]
\end{theorem} 
\begin{proof}
    Проведем через $(\bar{x}_0, f(\bar{x}_0))$ и произвольную $(\bar{x}, f(\bar{x}))$ секущую, ее параметрическое уравнение имеет вид:
    \[\begin{cases}
        X_i=x_{0i}+(x_i-x_{0i})t,\\
        Y=f(\bar{x}_0)+(f(\bar{x})-f(\bar{x}_0))t.
    \end{cases}
    \]
    где $i=1,\dots,n$. Тогда можем выписать ее направляющий вектор:
    \[\begin{pmatrix}
        x_1-x_{01},\ \dots,\ x_n-x_{0n},\ f(\bar{x})-f(\bar{x}_0)
    \end{pmatrix}
    \]
    теперь выпишем вектор нормали к плоскости:
    \[\begin{pmatrix}
        \frac{\partial {f}}{\partial {x_1}},\ \dots,\ \frac{\partial {f}}{\partial {x_n}},\ -1
    \end{pmatrix}
    \]
    значит
    \[\cos{\alpha}=\frac{\sum\limits_{i=1}^{n}\frac{\partial {f}}{\partial {x_i}}(x_i-x_{0i})-(f(\bar{x})-f(\bar{x}_0))}{\sqrt{1+\sum\limits_{i=1}^{n}(\frac{\partial {f}}{\partial {x_i}})^2}\cdot \sqrt{(f(\bar{x})-f(\bar{x}_0))^2+\sum\limits_{i=1}^{n}(x_i-x_{i0})^2}}\leq \frac{|\bar{\bar{o}}{(\rho)}|}{c\rho}\to 0\]
    следовательно $\alpha \to \frac{\pi}{2}$.
\end{proof} 
\begin{theorem} (Дифференциал сложной функции и инвариантность формы первого дифференциала)\\
    Пусть $f(\bar{x})$ дифференцируема в $\Omega\subset \R^n,\ \bar{x}(\bar{t}): \R^k\to \R^n,\ \bar{x}(\bar{t})$ дифференцируемы в $\Omega_1\subset \R^k$ и $\forall \bar{t}\in \Omega_1: \bar{x}(\bar{t})\in \Omega$. Тогда $f(\bar{x}(\bar{t}))$ дифференцируема в $\Omega_1$.
\end{theorem} 
\begin{proof}
    $\forall \bar{x}_0\in \Omega$:
    \[f(\bar{x}_0+\Delta \bar{x})-f(\bar{x}_0)=\sum\limits_{i=1}^{n}\frac{\partial {f}}{\partial {x_i}}(x_i-x_{0i})+\bar{\bar{o}}{\left(\sqrt{\sum\limits_{i=1}^{n}(x_i-x_{0i})^2}\ \right)}\]
    \begin{multline*}
        f(\bar{x}(\bar{t}_0+\Delta \bar{t}))-f(\bar{x}(\bar{t}_0))=\sum\limits_{i=1}^{n}\frac{\partial {f}}{\partial {x_i}}(x_i(\bar{t})-x_i(\bar{t}_0))+\bar{\bar{o}}{\left(\sqrt{\sum\limits_{i=1}^{n}(x_i(\bar{t})-x_i(t_0))^2}\ \right)}=\\
        =\sum\limits_{i=1}^{n}\frac{\partial {f}}{\partial {x_i}}\left(\sum\limits_{j=1}^{k}\frac{\partial {x_i}}{\partial {t_j}}(t_j-t_{0j})+\bar{\bar{o}}{(\rho)}\right)+\bar{\bar{o}}{\left(\sqrt{\sum\limits_{i=1}^{n}\left(\sum\limits_{j=1}^{k}\frac{\partial {x_i}}{\partial {t_j}}(t_j-t_{0j})+\bar{\bar{o}}{(\rho)}\right)^2}\right)}=\\
        =\sum\limits_{j=1}^{k}\sum\limits_{i=1}^{n}\frac{\partial {f}}{\partial {x_i}}\cdot \frac{\partial {x_i}}{\partial {t_j}}(t_j-t_{0j})+\bar{\bar{o}}{(\rho)}+\bar{\bar{o}}{(O(\rho))}=\sum\limits_{j=1}^{k}\left(\sum\limits_{i=1}^{n}\frac{\partial {f}}{\partial {x_i}}\cdot \frac{\partial {x_i}}{\partial {t_j}}(t_j-t_{0j})\right)+\bar{\bar{o}}{(\rho)}.
    \end{multline*}
    Значит
    \[df=\sum\limits_{j=1}^{k}\left(\sum\limits_{i=1}^{n}\frac{\partial {f}}{\partial {x_i}}\cdot \frac{\partial {x_i}}{\partial {t_j}}\right) dt_j\]
    Заметим, что если подставить дифференциал функции $x_i(\bar{t})$:
    \[dx_i=\sum\limits_{j=1}^{k}\frac{\partial {f}}{\partial {x_i}}\ dt_j\]
    В дифференциал функции $f(\bar{x})$, то получим
    \[df = \sum\limits_{i=1}^{n}\frac{\partial {f}}{\partial {x_i}}\ dx_i=\sum\limits_{i=1}^{n}\frac{\partial {f}}{\partial {x_i}}\left(\sum\limits_{j=1}^{k}\frac{\partial {x_i}}{\partial {t_j}}\ dt_j\right)=\sum\limits_{j=1}^{k}\left(\sum\limits_{i=1}^{n}\frac{\partial {f}}{\partial {x_i}}\cdot \frac{\partial {x_i}}{\partial {t_j}}\right) dt_j\]
    что совпадает с выражением дифференциала, полученным в доказательстве теоремы. Значит, первый дифференциал инвариантен относительно выбора системы координат. 
\end{proof} 
\begin{theorem}
    Пусть $f(\bar{x})\in \mathcal{C}^2(\Omega)$. Тогда $\forall i,j:$
    \[\frac{\partial^2 {f}}{\partial {x_i}\partial{x_j}}=\frac{\partial^2 {f}}{\partial {x_j}\partial{x_i}}\]
\end{theorem} 
\begin{proof} 
    Пусть $\Delta f(x)=f(x+\Delta x)-f(x)$. Докажем в $\R^2$
    \begin{multline*}
        \Delta_x(\Delta_y f(x,y))=\Delta_x(f(x,y+\Delta_y)-f(x,y))=\\
        =f(x+\Delta x, y+\Delta y)-f(x+\Delta x, y)-f(x,y+\Delta y)+f(x,y)
    \end{multline*}
    \begin{multline*}
        \Delta_y(\Delta_x f(x,y))=\Delta_y(f(x+\Delta x,y)-f(x,y))=\\
        =f(x+\Delta x, y+\Delta y)-f(x,y+\Delta y)-f(x+\Delta x,y)+f(x,y)
    \end{multline*}
    Значит
    \[\Delta_x(\Delta_y f(x,y))=\Delta_y(\Delta_x f(x,y))\]
    Тогда по теореме Лагранжа:
    \begin{multline*}
        \Delta_x(\Delta_y f(x,y))=\Delta_x (f(x,y+\Delta y)-f(x,y))=\\
        =\Delta_x(f'_y(x,y+\theta_2\Delta y)\Delta y)=f''_{yx}(x+\theta_1\Delta x,y+\theta_2\Delta y)\Delta y \Delta x
    \end{multline*}
    \begin{multline*}
        \Delta_y(\Delta_x f(x,y))=\Delta_y (f(x+\Delta x,y)-f(x,y))=\\
        =\Delta_y(f'_x(x+\theta_3\Delta x,y)\Delta x)=f''_{xy}(x+\theta_3 \Delta x, y+\theta_4 \Delta y)\Delta x\Delta y
    \end{multline*}
    Устремив $\sqrt{\Delta x^2+\Delta y^2}\to 0$, получим
    \[f''_{xy}(x,y)=f''_{yx}(x,y)\]
\end{proof} 