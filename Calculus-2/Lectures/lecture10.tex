\newpage
\section{Интеграл Римана-Стилтьеса}
\subsection{Функции ограниченной вариации}
\begin{definition}
    Пусть $f(x)$ определена на $[a,b],\ T_{[a,b]}$ - разбиение отрезка $[a,b]$. Сумма вида 
    \[V(f,t)=\sum\limits_{i=1}^{n}|f(x_i)-f(x_{i-1})|\]
    называется вариацией функции на данном разбиении $T$
\end{definition} 
\begin{definition}
    Если $\exists\ M>0$ такое, что $\forall T_{[a,b]}: V(f,T)\leq M$, то функция называется функцией ограниченной вариации на $[a,b]$, а величина
    \[\variation\limits_{a}^{b} f(x)=\text{var}_{[a,b]}f(x)=\sup\limits_{T}V(f,T)\] % нужно написать declaremathoperator
    называется полной вариацией функции на отрезке $[a,b]$
\end{definition} 
% \begin{examples}
    % В разработке
    % Пример 1 (полная вариация за исключением конечного числа точек)\\
    % Пример 2 (когда это невозможно)
% \end{examples}
\subsection{Свойства функций ограниченной вариации}
\begin{numtheorem}
    Если $f\in \mathcal{V}[a,b]$, то $f$ - ограничена на $[a,b]$.
\end{numtheorem}
\begin{proof}
    Временно очев.
\end{proof}
\begin{numtheorem}
    Пусть $T_{[a,b]}'$ - измельчение $T_{[a,b]}$. Тогда $V(f, T)\leq V(f, T')$
\end{numtheorem} 
\begin{proof}
    Временно очев (модуль суммы меньше или равен суммы модулей).
\end{proof} 
\begin{numtheorem}
    Если $f\in \mathcal{V} [a,b]$, то $\forall \alpha\in \R$:
    \[\variation\limits_{a}^{b}(\alpha\cdot f(x))=|\alpha|\cdot \variation\limits_{a}^{b} f(x)\] 
\end{numtheorem} 
\begin{numtheorem}
    Если $f,g\in \mathcal{V}[a,b]$, то $f+g\in \mathcal{V}[a,b]$
\end{numtheorem} 
\begin{proof}
    \[\variation\limits_{a}^{b} (f(x)+g(x))\leq \variation\limits_{a}^{b} f(x)+\variation\limits_{a}^{b} g(x)\]
\end{proof} 
\begin{numtheorem}
    Если $f,g\in \mathcal{V}[a,b]$, то $f(x)\cdot g(x)\in \mathcal{V}[a,b]$
\end{numtheorem} 
\begin{proof}
    \begin{multline*}
        |f(x_i)g(x_i)-f(x_{i-1})g(x_i)+f(x_{i-1})g(x_{i})-f(x_{i-1})g(x_{i-1})\leq\\
        \leq M_1\cdot |f(x_i)-f(x_{i-1})|+M_2\cdot |g(x_i)-g(x_{i-1})|
    \end{multline*}
\end{proof} 
\begin{numtheorem}
    Если $f,g\in \mathcal{V}[a,b],\ g\geq \epsilon>0$, то $\frac{f}{g}\in \mathcal{V}[a,b]$
\end{numtheorem} 
\begin{numtheorem}
    Если $f(x)$ монотонна на $[a,b]$, то
    \[\variation\limits_{a}^{b} f(x)=|f(b)- f(a)|\]
\end{numtheorem}
\begin{numtheorem} (Аддитивность)\\
    Если $c\in (a,b),\ f\in \mathcal{V}[a,c]$ и $f\in \mathcal{V}[b,c]$, то $f\in \mathcal{V}[a,b]$ и 
    \[\variation\limits_{a}^{b}f(x)=\variation\limits_{a}^{c}f(x)+\variation\limits_{b}^{c}f(x)\]
\end{numtheorem} 
\begin{proof}
    Рассмотрим 
    \[V(f,T_{[a,b]})\leq V(f, T_{[a,b]\cup \{c\}})=V(f, T_{[a,c]})+V(f,T_{[c,b]})\]
    значит
    \[\variation\limits_{a}^{b} f(x)\leq \variation\limits_{a}^{c} f(x)+\variation\limits_{c}^{b} f(x)\]
    С другой стороны, рассмотрим 
    \[V(f, T_{[a,c]})+V(f, T_{[c,b]})=V(f, T_{a,b}\ni c)\]
    отсюда получим
    \[\variation\limits_{a}^{c}f(x)+\variation\limits_{c}^{b}f(x)=\sup V(f,T_{[a,b]}\ni c)\leq \variation\limits_{a}^{b} f(x)\] 
\end{proof} 
\setcounter{thmcount}{0}
\begin{theorem}
    Если $f(x)\in \mathcal{V}[a,b]$, то $\exists\ h(x),\ v(x)$ - монотонно неубывющие на $[a,b]$ такие, что $f(x)=v(x)-h(x)$.
\end{theorem} 
\begin{proof}
    Пусть
    \[v(x)=\variation\limits_{a}^{x} f(x)\]
    Рассмотрим функцию $h(x)=v(x)-f(x)$. Пусть $x_1,\ x_2\in [a,b],\ x_2>x_1$:
    \[h(x_2)-h(x_1)=\variation\limits_{x_1}^{x_2}f(x)-(f(x_2)-f(x_1))\geq 0\]
    так как 
    \[\variation\limits_{x_1}^{x_2}\geq|f(x_2)-f(x_1)|\]
    $\Rightarrow h(x)$ - неубывает.
\end{proof} 
\subsection{Липшицевы функции}
\begin{definition}
Функция $f(x)$, опеределенная на $(a,b)$, называется липшицевой, если $\exists\ M>0$ такое, что $\forall x_1,x_2\in [a,b]:$
\[|f(x_2)-f(x_1)|\leq M\cdot |x_2-x_1|\]
Часто обозначают $f(x)\in \text{Lip}[a,b]$ или $f\in \text{Lip}_1[a,b]$
\end{definition} 
\begin{theorem}
    Если $f(x)\in \text{Lip}_1[a,b]$, то $f\in \mathcal{V}[a,b]$
\end{theorem} 
\begin{proof}
    \[V(f,T)=\sum\limits_{i=1}^{n}|f(x_i)-f(x_{i-1})|\leq M\cdot\sum\limits_{i=1}^{n}|x_i-x_{i-1}|=M\cdot (b-a)\]
\end{proof} 
\begin{theorem}
    Если $f(x)\in C^1[a,b]$, то $f(x)\in \mathcal{V}[a,b]$ и 
    \[\variation\limits_{a}^{b} f(x)=\int\limits_{a}^{b}|f'(x)|\ dx\]
\end{theorem} 
\begin{proof}
    Если $f\in C^1[a,b]$, то $f\in \text{Lip}_1[a,b]$, так как по формуле Лагранжа:
    \[|f(x_2)-f(x_1)|=|f'(\xi)(x_2-x_1)|\leq \max\limits_{x\in [a,b]}|f'(x)|\cdot|(x_2-x_1)|\]
    \begin{multline*}
        V(f, T)=\sum\limits_{i=1}^{n}|f(x_i)-f(x_{i-1})|=\sum\limits_{i=1}^{n} |f'(\xi)(x_i-x_{i-1})|=\\
        =\sum\limits_{i=1}^{n} |f'(\xi)| (x_i-x_{i-1})\to \int\limits_{a}^{b}|f'(x)|\ dx
    \end{multline*}
    $\Rightarrow \forall \epsilon>0\ \exists\ \delta>0,\ \forall T,\ d(T)<\delta$:
    \[\left|V(f,T)-\int\limits_{a}^{b}|f'(x)|\ dx\right|<\epsilon\]
    По определению точной верхней грани: $\forall \epsilon>0\ \exists\ V(f,T_{\epsilon})$ такое, что
    \[\variation\limits_{a}^{b} f(x)-V(f,T_{\epsilon})<\epsilon\]
    измельчаем $T_{\epsilon}$, значит $T_{\epsilon}^*$ с $d(T_{\epsilon}^*)<\delta$:
    \begin{multline*}
        \left| \variation\limits_{a}^{b} f(x)-\int\limits_{a}^{b}|f'(x)|\ dx \right|=\\=\left|\variation\limits_{a}^{b} f(x)-\int\limits_{a}^{b}|f'(x)|\ dx+V(f,T_{\epsilon}^*)-V(f, T_{\epsilon}^*)\right|\leq\\
        \leq \left|\variation\limits_{a}^{b} f(x)-V(f, T_{\epsilon}*)\right|+\left|\int\limits_{a}^{b}|f(x)|\ dx-V(f,T_{\epsilon}*)\right|<2\epsilon
    \end{multline*}
\end{proof}
% Тут должна быть вставка про лестницу кантора
\subsection{Определение интеграла Римана-Стилтьеса}
\begin{definition}
    Пусть $f(x),\ g(x)$ определены на $[a,b]$. $\forall T(\xi)$ сумма 
    \[\sum\limits_{i=1}^{n} f(\xi_i)(g(x_i)-g(x_{i-1}))=\sigma_g(f,T)\]
    называется интегральной суммой Римана-Стилтьеса.
\end{definition} 
\begin{definition}
    Если существует предел
    \[\lim\limits_{d\to 0}\sigma_g(f, T(\xi))=\int\limits_{a}^{b}f(x)\ d(g(x))\]
    то он называется интегралом Римана-Стилтьеса.
\end{definition}
\subsection{Свойства интеграла Римана-Стилтьеса}
\begin{numtheorem}
    $\forall \alpha,\ \beta\in \R$:
    \[\int\limits_{a}^{b}(\alpha\cdot f(x))\ d(\beta\cdot g(x))=\alpha \cdot \beta \int\limits_{a}^{b} f(x)\ d(g(x))\]
\end{numtheorem} 
\begin{numtheorem}
    Если существуют интегралы
    \[\int\limits_{a}^{b}f_1(x)\ d(g(x)),\ \int\limits_{a}^{b}f_2(x)\ d(g(x))\]
    то существует интеграл 
    \[\int\limits_{a}^{b}(f_1+f_2)\ d(g(x))=\int\limits_{a}^{b}f_1(x)\ d(g(x))+\int\limits_{a}^{b}f_2(x)\ d(g(x))\]
\end{numtheorem} 
\begin{numtheorem}
    Если существуют интегралы
    \[\int\limits_{a}^{b}f(x)\ d(g_1(x)),\ \int\limits_{a}^{b}f(x)\ d(g_2(x))\]
    то существует интегал 
    \[\int\limits_{a}^{b}f(x)\ d(g_1(x)+g_2(x))=\int\limits_{a}^{b}f(x)\ d(g_1)+\int\limits_{a}^{b}f(x)\ d(g_2)\]
\end{numtheorem} 
