\begin{comm}
    \[\tg^2z+1=\frac{\sin^2z+\cos^2z}{\cos^2z}=\frac{1}{\cos^2z}\]
    \[\int \frac{dt}{(t^2+q^2)^k}=\begin{vmatrix} t=q\tg{z}\\ dt=\cfrac{q}{\cos^2z}dz\end{vmatrix}=\int \frac{qdz}{\cos^2z(q^2\tg^2z+q^2)^k}=\int \frac{\cos^{2k-2}z}{q^{2k-1}}dz\]
\end{comm}
\subsection{Метод Остроградского}
\begin{multline*}
    \int \frac{P(x)}{Q(x)}\ dx=\int \frac{P(x)}{\prod\limits_{i=1}^{n}(x-a_i)^{\alpha_i}\cdot \prod\limits_{j=1}^{k}(x^2+b_j x+c_j)^{\beta_j}}\ dx=\\
    =\frac{P_1(x)}{\prod\limits_{i=1}^{n}(x-a_i)^{\alpha_i-1}\cdot \prod\limits_{j=1}^{k}(x^2+b_j x+c_j)^{\beta_j-1}}+\\
    +\int \frac{P_2(x)}{\prod\limits_{i=1}^{n}(x-a_i)\cdot \prod\limits_{j=1}^{k}(x^2+b_j x+c_j)}\ dx
\end{multline*}
\newpage
\section{Интеграл Римана}
\subsection{Интегрируемость по Риману}
\begin{definition}
    $\{x_i\}_{i=0}^n\subset [a,b]$ называется разбиением отрезка, если \\
    $a=x_0<\dots<x_n=b$. Обозначается $T_{[a,b]}^+$. Если $b=x_0>\dots>x_n=a$, то обозначают $T_{[a,b]}^-$.\\
    Отрезки $[x_{i-1},x_i]$ или $[x_i, x_{i-1}]$ называются отрезками разбиения, их обычно обозначают $\Delta_i$.\\
    Длина отрезка $\Delta_i$ обозначается $\Delta x_i := x_i-x_{i-1}$. \\
    Длина наибольшего из отрезков называется диаметром разбиения \\
    $d(T)=\max|x_i-x_{i-1}| = \max{\Delta x_i}$. 
\end{definition} 
\begin{definition}
    Пусть $T_{[a,b]}$ - разбиение отрезка $[a,b]$. Разметкой для $T_{[a,b]}$ называется множество точек $\{\xi_i\}_{i=1}^n$ такое, что $\forall i: \xi_i \in \Delta_i$.\\
    Если $\{\xi_i\}_{i=1}^n$ является разметкой для $\{x_i\}_{i=0}^n$, то пара $\left(\{x_i\}_{i=0}^n, \{\xi_i\}_{i=1}^n\right)$ называется размеченым разбиением и обозначается $T(\xi)$.
\end{definition} 
\begin{definition}
    Сумма 
    \[\sigma_{[a,b]}=\sum\limits_{i=1}^{N}f(\xi_i)(x_i-x_{i-1})\]
    называется интегральной суммой. Иногда ее обозначают $\sigma_T(\xi)$ или $\sigma(T_{[a,b]}(\xi))$
\end{definition} 
\begin{definition}
    Пусть $f(x)$ определена на $[a,b]$. Рассмотрим $T_{[a,b]}(\xi)$. Если
    \[\exists\ I\in \R: \forall \epsilon>0\ \exists\ \delta>0,\ \forall\ T(\xi)\subset \{T:d(T)<\delta\}:\ \vline\ \sum\limits_{i=1}^{N}f(\xi_i)(x_i-x_{i-1})-I\ \vline <\epsilon\]
    то говорят, что $f(x)$ интегрируема по Риману на $[a,b]$, а число $I$ называют интегралом Римана на размеченных разбиениях на отрезке $[a,b]$. Интеграл Римана обозначают
    \[I=\int\limits_{a}^{b}f(x)\ dx\ \ \text{или}\ \ I=\int\limits_{b}^{a}f(x)\ dx\]
    для $T^+$ и $T^-$ соответственно. 
\end{definition} 
\begin{comm}
    Можно считать определение интеграла определением предела интегральных сумм и писать
    \[\lim\limits_{d\to 0}\left(\sum\limits_{i=1}^{N}f(\xi_i)(x_i-x_{i-1})\right)=I\]
    где $d$ - диаметр разбиения.
\end{comm} 
\begin{statement}
    \[\text{Если}\ \exists\ \int\limits_{a}^{b}f(x)\ dx,\ \text{то}\ \ \exists\ \int\limits_{b}^{a}f(x)\ dx\ \ \text{и}\ \int\limits_{a}^{b}f(x)\ dx=-\int\limits_{b}^{a}f(x)\ dx\]
\end{statement} 
\begin{definition}
    Класс функций, интегрируемых на $[a,b]$ по Риману, обозначается $\mathcal{R}[a,b]$.
\end{definition} 
\begin{theorem}
    Если $f(x)\in \mathcal{R}[a,b]$, то $f(x)$ - ограничена на $[a,b]$.
\end{theorem} 
\begin{proof}
    Предположим, что $\exists \{x_n\}_{n=1}^{\infty}\subset [a,b],\ \exists \lim\limits_{n\to \infty}x_n=\widetilde{x}$, что\\
    $|f(x_n)|>n$ и пусть
    \[\exists\ \lim\limits_{d\to 0}\left( \sum\limits_{i=0}^{N}f(\xi_i)(x_i-x_{i-1}) \right)=I\] 
    Возьмем $\epsilon=1$. Тогда  
    \[\vline\ \sum\limits_{i=0}^{N}f(\xi_i)(x_i-x_{i-1})-I\ \vline<1\]
    Возмем $\Delta_k$ такой, что $\widetilde{x}\in \Delta_k \Rightarrow f(x)$ - неограничена на $\Delta_k$. Тогда, зафиксировав точки в остальных отрезках разбиения, получим
    \[I-\sum\limits_{i=1,i\ne k}^{N} f(\xi_i)(x_i-x_{i-1})-1<f(\xi_k)(x_k-x_{k-1})<I-\sum\limits_{i=1,i\ne k}^{N} f(\xi_i)(x_i-x_{i-1})+1\]
    противоречие с тем, что $f(x)$ принимает сколь угодно большие  на $\Delta_k$.
\end{proof} 

