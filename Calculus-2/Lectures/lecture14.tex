\begin{theorem}
    Пусть $A\subset \R^n$ - замкнуто, $\bar{x}\not\in A$. Тогда $\exists\ \bar{y}\in A: \rho(\bar{x}, \bar{y})=\rho(\bar{x}, A)$.
\end{theorem} 
\begin{proof} По свойству точной нижней грани:
    \[\forall \epsilon>0\ \exists\ \bar{y}_{\epsilon}\in A: \rho(\bar{x}, \bar{y}_{\epsilon})<\rho(\bar{x}, A)+\epsilon\] 
    Возьмем 
    \[\forall m\in \N\ \exists\ \bar{y}_m: \rho(\bar{x}, \bar{y}_m)<\rho(\bar{x}, A)+\frac{1}{m} \Rightarrow \{\bar{y}_m\} - \text{ограничено}\] \
    Значит
    \[\exists\ \{\bar{y}_{m_k}\}_{k=1}^{\infty}: \bar{y}_{m_k} \to \bar{y}\in A,\ \rho(\bar{x}, \bar{y}_{m_k})<\rho(\bar{x}, A)+\frac{1}{m_k}\] 
    после предельного перехода получим: 
    \[\rho(\bar{x}, \bar{y})\leq \rho(\bar{x}, A),\ \rho(\bar{x}, \bar{y})\geq \rho(\bar{x}, A) \Rightarrow \rho(\bar{x}, \bar{y})= \rho(\bar{x}, A)\] 
\end{proof} 
\begin{definition}
    Множество $A\subset \R^n$ называется линейно связным, если любые две его точки можно соединить кривой, то есть
    \[\forall x,y\in A\ \exists\ \bar{\gamma}: [0,1] \to A: \bar{\gamma}(0)=x,\ \bar{\gamma}(1)=y\]
\end{definition} 
\begin{theorem}
    Пусть $A\subset \R^n$ - линейно связное, $B\subset \R^n,\ A\cap B \ne \emptyset,\\
    A\cap (\R^n\setminus B)\ne \emptyset$. Тогда $A\cap \partial B \ne \emptyset$.
\end{theorem}
\begin{proof}
    Пусть $x_1\in A\cap B,\ x_2\in A\cap (\R^n\setminus B),\ \bar{\gamma}(t)\subset A,\ t\in [0,1],\\
    \bar{\gamma}(0)=x_1,\ \bar{\gamma}(1)=x_2$. Пусть  
    \[\tau=\sup\limits_{\bar{\gamma}(t)\in B}\{t\} \Rightarrow \forall \epsilon>0\ \exists\ t_{\epsilon}\in (\tau-\epsilon, \tau),\ \bar{\gamma}(t_{\epsilon})\in B\] 
    Тогда при $t>\tau$ получим
    \[\bar{\gamma}(t)\in \R^n\subset A \Rightarrow \bar{\gamma}(\tau)\in A\cap \partial B\]
\end{proof}  
\begin{definition}
    Открытое линейно связное множество называется областью.\\
    Замыкание области называют замкнутой областью. Область часто будем обозначать $\Omega$.
\end{definition} 
\subsection{Функции нескольких переменных и их предел}
Будут рассматриваться функции $f: \R^n\to \R,\ f(\bar{x})=f(x_1, \dots, x_n)$
\begin{definition}
    Множество $\{\bar{x}: f(\bar{x})=C,\ C=\text{const}\}$ называется множеством (линией) уровня $C$.
\end{definition} 
\begin{definition} (Предел по Гейне)\\
    Если $\forall \{\bar{x}_k\}_{k=1}^{\infty}\subset D(f):\ \bar{x}_k\to \bar{x}_0,\ \bar{x}_k\ne \bar{x}_0$ существует предел
    \[\lim\limits_{k\to \infty}f(\bar{x}_k)=A\]
    то $A$ называется пределом функции $f$ в точке $\bar{x}_0$.
\end{definition} 
\begin{definition} (Предел по Коши)\\
    Если $\exists\ A\in \R:$
    \[\forall \epsilon>0\ \exists\ \delta>0,\ \forall \bar{x}\in \mathring{B}_{\delta_\epsilon}(\bar{x}_0)\cap D(f): |f(\bar{x})-A|<\epsilon\]
    то сушествует предел 
    \[\lim\limits_{\bar{x}\to x_0}f(\bar{x})=A\]
\end{definition} 
\begin{theorem}
    Эти определения эквивалентны
\end{theorem} 
\begin{proof}
    Без доказательства.
\end{proof} 
\begin{comm}
    Свойства, доказанные для пределов в первом семестре, остаются верными для функций многих переменных и доказываются аналогично.
\end{comm} 