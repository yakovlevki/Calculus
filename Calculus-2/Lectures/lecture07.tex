\subsection{Спрямляемость гладкой кривой и формула ее длины}
\begin{theorem}
    Пусть 
    \[\bar{\gamma}(t) = 
    \begin{pmatrix}
    x_1(t)\\
    x_2(t)\\
    \vdots\\
    x_n(t)
    \end{pmatrix} \in C^1[a,b]\]
    Тогда $\bar{\gamma}(t)$ спрямляема и
    \[|\bar{\gamma}| = \int\limits_{a}^{b} \sqrt{\sum_{j = 1}^{n} x_j^{'2}(t)}\ dt\]
\end{theorem}
\begin{proof}
    \begin{multline*}
        |L(T_{\bar{\gamma}})| = \sum_{i = 1}^{N} \sqrt{ \sum_{j = 1}^{n}(x_j(t_i) - x_j(t_{i - 1}))^2} = (1)\\
        = \sum_{i = 1}^{N} \sqrt{\sum_{j = 1}^{n} x_j^{'2}(\xi_{ij}) \cdot (t_i - t_{i - 1})^2} =\\
        = \sum_{i = 1}^{N} \sqrt{\sum_{j = 1}^{n} x_j^{'2} (\xi_{ij})} (t_i - t_{i - 1}) \leqslant M \cdot \sqrt{n} \cdot (b - a)
    \end{multline*}
    Переход (1) по формуле Лагранжа, а последняя оценка устроена так: каждое из $x'_{j}$ - непрерывно на каждом отрезке разбиения, значит, по второй теореме Вейерштрасса, у нее есть максимум. Возьмем $M$ - максимум из этих максимальных значений на отрезке разбиения, тогда 
    \[\sqrt{\sum\limits_{j=1}^{n}x_j^{'2}}\leq M\cdot \sqrt{n}\]
    остается вынести это за скобку и сумма длин отрезков разбиения схлопнется в $b-a$.
    $\Rightarrow \bar{\gamma}$ спрямляема.
    \newpage
    \begin{multline*}
        \left| |L(T_{\bar{\gamma}})| - \sigma_{\sqrt{\sum_{j = 1}^{n} x_j^{'2}}} \right| =\\
        =\Bigg| \sum_{i = 1}^{N} \sqrt{\sum_{j = 1}^{n} x_j^{'2} (\xi_{ij})}(t_i - t_{i - 1}) - \sum_{i = 1}^{N} \sqrt{\sum_{j = 1}^{n} x_j^{'2} (\nu_{i})}(t_i - t_{i - 1}) \Bigg| =\tab[2cm]\\ 
        \tab[2cm]=\left| \sum_{i=1}^{N} \left( \left( \sqrt{\sum_{j = 1}^{n} x_j^{'2} (\xi_{ij})} - \sqrt{\sum_{j = 1}^{n} x_j^{'2} (\nu_{i})}  \right) (t_i - t_{i - 1}) \right) \right| \leqslant\\
        \sum_{i = 1}^{N} \sum_{j = 1}^{n} |x_j^{'} (\xi_{ij}) - x_j^{'} (\nu_i)| \cdot (t_i - t_{i - 1}) < \epsilon \cdot n \cdot (b - a)
        %\left| \frac{\sum_{i = 1}^{k} (a_i - b_i) (a_i + b_i)}{\sqrt{\epsilon} + \sqrt{\epsilon}} \leqslant \sum_{i = 1}^{k} |a_i - b_i| \right|
    \end{multline*}
    Последняя оценка сделана с применением леммы, которая доказана чуть ниже.
    \[\forall \epsilon > 0 \ \exists\ \delta_{\epsilon} > 0,\ d(T) < \delta_{\epsilon}\]
    \[\Rightarrow \left| |L(T_{\bar{\gamma}})| - \int\limits_{a}^{b} \sqrt{\sum_{j = 1}^{n} x_{j}^{'2}(t)} dt \right| < 2 \epsilon n (b-a)\]
    $\forall \epsilon > 0 \ \exists\ L(T^{*}_{\bar{\gamma}})$, что $|L(T^{*}_{\bar{\gamma}}) | > |\bar{\gamma}| - \epsilon$ (свойство точной верхней грани). \newline
    Измельчаем $T_{\bar{\gamma}}^{*}$ до тех пор, пока $d(T_{\bar{\gamma}}^{*}) < \delta_{\epsilon}$.
\end{proof}
\begin{lemma}
    \[\left| \sqrt{\sum\limits_{i = 1}^{k} a_i^2} - \sqrt{\sum_{i = 1}^{k} b_i^2} \right| \leq \sum\limits_{i = 1}^{k} |a_i - b_i|\]
\end{lemma}
\begin{proof}
    \begin{multline*}
        \left| \sqrt{\sum\limits_{i = 1}^{k} a_i^2} - \sqrt{\sum_{i = 1}^{k} b_i^2} \right| = \left| \frac{\sum\limits_{i = 1}^{k}\left( (a_i - b_i) (a_i + b_i)\right)}{\sqrt{\sum\limits_{i = 1}^{k} a_i^2} + \sqrt{\sum\limits_{i = 1}^{k} b_i^2}} \right| =\\
        =\left| \sum\limits_{i = 1}^{k}\left( (a_i - b_i)\cdot \frac{(a_i + b_i)}{\sqrt{\sum\limits_{i = 1}^{k} a_i^2} + \sqrt{\sum\limits_{i = 1}^{k} b_i^2}}\right) \right| \leq (*)\\ 
        \leq \left| \sum\limits_{i=1}^{n} 1\cdot (a_i-b_i)\right| \leq \sum\limits_{i = 1}^{k} |a_i - b_i|
    \end{multline*}
    \[(*):\ \ \ \  a_i \leq  \sqrt{\sum\limits_{i = 1}^{k} a_i^2},\ b_i \leq \sqrt{\sum\limits_{i = 1}^{k} b_i^2}\ \Rightarrow\ \frac{(a_i + b_i)}{\sqrt{\sum\limits_{i = 1}^{k} a_i^2} + \sqrt{\sum\limits_{i = 1}^{k} b_i^2}} \leq 1\]
\end{proof} 


