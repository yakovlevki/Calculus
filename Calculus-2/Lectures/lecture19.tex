\subsection{Теорема о неявном отображении}
\begin{definition}
    Отображение $\bar{F}:\R^n\to \R^k$ такое, что
    \[\bar{F}=\begin{pmatrix}
        f_1(x_1,\dots,x_n)\\
        \vdots\\
        f_k(x_1,\dots,x_n)
    \end{pmatrix}
    \]
    называется дифференцируемой, если $\forall i=1,\dots k: f_i$ - дифференцируема. 
\end{definition} 
\begin{statement}
    Если $\bar{F}: \R^n\to \R^k$ дифференцируемо, то полное приращение имеет главную линейную часть вида
    \[J\cdot \begin{pmatrix}
        \Delta x_1\\
        \vdots\\
        \Delta x_n
    \end{pmatrix}\]
\end{statement} 
\begin{proof}
    \begin{multline*}
        \begin{pmatrix}
            f_1(\bar{x}_0+\Delta \bar{x})\\
            \vdots\\
            f_k(\bar{x}_0+\Delta \bar{x})
        \end{pmatrix}
        -\begin{pmatrix}
            f_1(\bar{x}_0)\\
            \vdots\\
            f_k(\bar{x}_0)
        \end{pmatrix}
        =\begin{pmatrix}
            df_1+\bar{\bar{o}}{(\|\Delta x\|)}\\
            \vdots\\
            df_k+\bar{\bar{o}}{(\|\Delta x\|)}
        \end{pmatrix}=
        \\=\begin{pmatrix}
            \sum\limits_{i=1}^{n}\frac{\partial {f_1}}{\partial {x_i}}\Delta x_i\\
            \vdots\\
            \sum\limits_{i=1}^{n}\frac{\partial {f_k}}{\partial {x_i}}\Delta x_i
        \end{pmatrix}-\bar{\bar{o}}{(\|\Delta x\|)}=
        \begin{pmatrix}
            \frac{\partial {f_1}}{\partial {x_1}} & \dots & \frac{\partial {f_1}}{\partial {x_n}}\\
            \vdots & \null & \vdots\\
            \frac{\partial {f_k}}{\partial {x_1}} & \dots & \frac{\partial {f_k}}{\partial {x_n}}
        \end{pmatrix}
        \cdot 
        \begin{pmatrix}
            \Delta x_1\\
            \vdots\\
            \Delta x_n
        \end{pmatrix}    
        +\bar{\bar{o}}{(\|\Delta x\|)}    
    \end{multline*} 
\end{proof} 
    \begin{definition}
        Матрица
        \[J=\begin{pmatrix}
            \frac{\partial {f_1}}{\partial {x_1}} & \dots & \frac{\partial {f_1}}{\partial {x_n}}\\
            \vdots & \null & \vdots\\
            \frac{\partial {f_k}}{\partial {x_1}} & \dots & \frac{\partial {f_k}}{\partial {x_n}}
        \end{pmatrix}\]
        называется матрицей Якоби.\\
        Если $k=n$ то 
        \[\det{J}:=\frac{D(f_1,\dots,f_n)}{D(x_1,\dots,x_n)}\]
        называется якобианом.
    \end{definition} 
\begin{theorem} (Теорема о неявном отображении)\\
    Пусть $\bar{F}: \R^{n+m}\to \R^m$ непрерывно дифференцируемо в $B(\bar{x}_0,\bar{y}_0),\\ \bar{F}(\bar{x}_0, \bar{y}_0)=0,\ \frac{D(f_1,\dots, f_m)}{D(y_1,\dots,y_m)}\ne 0$ в $B(\bar{x}_0, \bar{y}_0)$. Тогда существует единственное непрерывное дифференцируемое отображение
    \[\bar{\Phi}=\begin{pmatrix}
        \phi_1(\bar{x})\\
        \vdots\\
        \phi_m(\bar{x})
    \end{pmatrix}
    \]
    такое, что $\bar{\Phi}(\bar{x}_0)=\bar{y}_0,\ \bar{F}(\bar{x}, \bar{\Phi}(\bar{x}))\equiv 0$ в $B(\bar{x}_0, \bar{y}_0)$.
\end{theorem} 
\begin{proof}
    Докажем по индукции. База: при $m=1$ - это теорема о неявной функции.
    \[
        \det{J}=\begin{vmatrix}
            \frac{\partial {f_1}}{\partial {y_1}} & \dots & \frac{\partial {f_1}}{\partial {y_n}} & \frac{\partial {f_1}}{\partial {y_{m+1}}}\\
            \vdots & \null & \vdots\\
            \frac{\partial {f_m}}{\partial {y_1}} & \dots & \frac{\partial {f_m}}{\partial {y_n}} & \frac{\partial {f_m}}{\partial {y_{m+1}}}\\
            \frac{\partial {f_{m+1}}}{\partial {y_1}} & \dots & \frac{\partial {f_{m+1}}}{\partial {y_m}} & \frac{\partial {f_{m+1}}}{\partial {y_{m+1}}}
        \end{vmatrix}
    \]
    имеет ненулевой минор порядка $m$, с точностью до обозначений это будет левый верхний минор. Значит система
    \[\begin{cases}
        f_1 = 0,\\
        \vdots\\
        f_m=0.
    \end{cases}
    \]
    разрешима:
    \[\begin{cases}
        y_1=\psi_1(\bar{x},y_{m+1}),\\
        \vdots\\
        y_m=\psi_m(\bar{x}, y_{m+1}).
    \end{cases}
    \]
    Подставим решение в систему:
    \[\begin{cases}
        f_1(\bar{x}, \psi_1(\bar{x}, y_{m+1}), \dots, \psi_m(\bar{x}, y_{m+1}), y_{m+1})\equiv 0,\\
        \vdots\\
        f_m(\bar{x}, \psi_1(\bar{x}, y_{m+1}), \dots, \psi_m(\bar{x}, y_{m+1}), y_{m+1})\equiv 0,\\
        f_{m+1}(\bar{x}, \psi_1(\bar{x}, y_{m+1}),\dots,\psi_m(\bar{x}, y_{m+1}), y_{m+1})=g(\bar{x}, y_{m+1}).
    \end{cases}
    \]
    возьмем у всех этих уравнений производную по $y_{m+1}$
    \[\begin{cases}
        \frac{\partial {f_1}}{\partial {y_1}}\cdot \frac{\partial {\psi_1}}{\partial {y_{m+1}}}+\frac{\partial {f_1}}{\partial {y_2}}\cdot \frac{\partial {\psi_2}}{\partial {y_{m+1}}}+ \dots +\frac{\partial {f_1}}{\partial {y_m}}\cdot \frac{\partial {\psi_m}}{\partial {y_{m+1}}}+\frac{\partial {f_1}}{\partial {y_{m+1}}}=0,\\
        \vdots\\
        \frac{\partial {f_m}}{\partial {y_1}}\cdot \frac{\partial {\psi_1}}{\partial {y_{m+1}}}+\frac{\partial {f_m}}{\partial {y_2}}\cdot \frac{\partial {\psi_2}}{\partial {y_{m+1}}}+ \dots +\frac{\partial {f_m}}{\partial {y_m}}\cdot \frac{\partial {\psi_m}}{\partial {y_{m+1}}}+\frac{\partial {f_m}}{\partial {y_{m+1}}}=0,\\
        \frac{\partial {f_{m+1}}}{\partial {y_1}}\cdot \frac{\partial {\psi_1}}{\partial {y_{m+1}}}+\frac{\partial {f_{m+1}}}{\partial {y_2}}\cdot \frac{\partial {\psi_2}}{\partial {y_{m+1}}}+ \dots +\frac{\partial {f_{m+1}}}{\partial {y_m}}\cdot \frac{\partial {\psi_m}}{\partial {y_{m+1}}}+\frac{\partial {f_{m+1}}}{\partial {y_{m+1}}}=\frac{\partial {g}}{\partial {y_{m+1}}}.
    \end{cases}
    \]
    Умножим каждую строку на алгебраическое дополнение к элементу последнего стоблца, а затем сложим уравнения. По теореме о фальшивом разложении получим:
    \[\det{J}=\frac{\partial {g}}{\partial {y_{m+1}}}\cdot \begin{vmatrix}
        \frac{\partial {f_1}}{\partial {y_1}} & \dots & \frac{\partial {f_1}}{\partial {y_m}}\\
        \vdots & \null & \vdots\\
        \frac{\partial {f_m}}{\partial {y_1}} & \dots & \frac{\partial {f_m}}{\partial {y_m}}
    \end{vmatrix} \Rightarrow \frac{\partial {g}}{\partial {y_{m+1}}}\ne 0
    \]
    \[g(\bar{x}), y_{m+1}=0 \Rightarrow y_{m+1}=\phi_{m+1}(\bar{x})\]
    отсюда
    \[y_1=\phi(\bar{x})=\psi_1(\bar{x}, \phi_{m+1}(\bar{x}))\]
    Непрерывность и дифференцируемость следует из теоремы о неявной функции.
\end{proof} 
\begin{comm} (О вычислении производных)\\
    \[\begin{cases}
        f_1(\bar{x}, \bar{y})=0,\\
        \vdots\\
        f_m(\bar{x}, \bar{y})=0.
    \end{cases}
    \Rightarrow
    \begin{cases}
        \sum\limits_{i=1}^{n}\frac{\partial {f_1}}{\partial {x_1}}dx_i+\sum\limits_{j=1}^{m}\frac{\partial {f_1}}{\partial {y_j}}dy_j=0,\\
        \vdots\\
        \sum\limits_{i=1}^{n}\frac{\partial {f_m}}{\partial {x_i}}dx_i+\sum\limits_{j=1}^{m}\frac{\partial {f_m}}{\partial {y_j}}dy_j=0.
    \end{cases}
    \Rightarrow
    \begin{cases}
        dy_1=\sum\limits_{i=1}^{n}(\ )_i dx_i,\\
        \vdots\\
        dy_m=\sum\limits_{i=1}^{n}(\ )_i dx_i.
    \end{cases}
    \]
\end{comm} 
\begin{theorem} (Теорема об обратном отображении)\\
    Пусть $\bar{F}: \R_x^n\to \R_y^n$ непрерывно дифференцируемо в $B(\bar{x}_0),\ \det{J}\ne 0$ в $B(\bar{x}_0)$. Тогда существует единственное отображение $\bar{F}^{-1}: \R_y^n\to \R_x^n$ непрерывно дифференцируемое в образе $B(\bar{x}_0)$.
\end{theorem} 
\begin{proof}
    %Рассмотрим отображение из $\R^{2n}$ в $\R^n$:
    Рассмотрим систему
    \[\begin{cases}
        f_1(x_1,\dots,x_n)-y_1=0,\\
        f_2(x_1,\dots,x_n)-y_2=0,\\
        \vdots\\
        f_n(x_1,\dots,x_n)-y_n=0.
    \end{cases}
    \]
    По теореме о неявном отображении получим решение:
    \[\begin{cases}
        x_1=\phi_1(\bar{y}),\\
        \vdots\\
        x_n=\phi_n(\bar{y}).
    \end{cases}
    \]
\end{proof} 