\subsection{Производная по направлению}
\begin{definition}
    Пусть
    \[\bar{l}=\begin{pmatrix}
        \cos{\alpha_1} & \cos{\alpha_2} & \dots & \cos{\alpha_n}
    \end{pmatrix},\
    \sum\limits_{i=1}^{n}\cos^2{\alpha_i}=1
    \]
    Предел
    \[\frac{\partial {f}}{\partial {\bar{l}}}=\lim\limits_{\Delta t\to 0}\frac{f(x_1+\Delta t \cdot \cos{\alpha_1},\dots,x_n+\Delta t \cdot \cos{\alpha_n})-f(\bar{x})}{\Delta t}\]
    называется производной по направлению $\bar{l}$.
\end{definition} 
\begin{theorem}
    Если $f(\bar{x})$ дифференцируема в $B(\bar{x}_0)$, то
    \[\frac{\partial {f}}{\partial {\bar{l}}}(x_0)=\sum\limits_{i=1}^{n}\left(\frac{\partial {f}}{\partial {x_i}}\cdot \cos{\alpha_i}\right)\]
\end{theorem} 
\begin{proof}
    \begin{multline*}
        \lim\limits_{\Delta t\to 0}\frac{f(x_1+\Delta t \cdot \cos{\alpha_1},\dots,x_n+\Delta t \cdot \cos{\alpha_n})-f(\bar{x})}{\Delta t}=\\
        =\lim\limits_{\Delta t\to 0}\frac{\sum\limits_{i=1}^{n}\left(\frac{\partial {f}}{\partial {x_i}}\cdot \Delta t\cdot \cos{\alpha_i}\right)+\bar{\bar{o}}{\left(\sqrt{\sum\limits_{i=1}^{n}\Delta t^2\cdot \cos^2{\alpha_i}}\right)}}{\Delta t}=\\
        =\sum\limits_{i=1}^{n}\left( \frac{\partial {f}}{\partial {x_i}}\cdot \cos{\alpha_i}\right)
    \end{multline*}
\end{proof}
\begin{definition}\
    Ковектор
    \[
    \grad{f}=\begin{pmatrix}
        \frac{\partial {f}}{\partial {x_1}} & \dots & \frac{\partial {f}}{\partial {x_n}}
    \end{pmatrix}
    \]
    называется градиентом функции $f(\bar{x})$.
\end{definition} 
\begin{theorem}
    Производная функции $f(\bar{x})$ по направлению $\grad{f}$ имеет максимальное по модулю значение по сравнению со всеми остальными производными по направлению.
\end{theorem} 
\begin{proof}
    Для любого направления $\bar{l}$:
    \[\left|\frac{\partial {f}}{\partial {\bar{l}}}\right|=\left|\sum\limits_{i=1}^{n}\left(\frac{\partial {f}}{\partial {x_i}}\cdot \cos{\alpha_i}\right)\right|\leq \sqrt{\sum\limits_{i=1}^{n}\left(\frac{\partial {f}}{\partial {x_i}}\right)^2}\cdot 1\]
    \[\left|\frac{\partial {f}}{\partial {(\grad{f})}}\right|=\left|\sum\limits_{i=1}^{n}\frac{\partial {f}}{\partial {x_i}}\cdot \frac{\frac{\partial {f}}{\partial {x_i}}}{\sqrt{\sum\limits_{i=1}^{n}\left(\frac{\partial {f}}{\partial {x_i}}\right)^2}}\right|=\sqrt{\sum\limits_{i=1}^{n}\left(\frac{\partial {f}}{\partial {x_i}}\right)^2}\]
    Отсюда
    \[\left|\frac{\partial {f}}{\partial {(\grad{f})}}\right|=\sqrt{\sum\limits_{i=1}^{n}\left(\frac{\partial {f}}{\partial {x_i}}\right)^2}\geq \left|\frac{\partial {f}}{\partial {\bar{l}}}\right|\]

\end{proof} 
\begin{definition}
    Пусть $f(\bar{x})\in \mathcal{C}^2(\Omega)$.
    \[\delta(\sum\limits_{i=1}^{n}\frac{\partial {f}}{\partial {x_i}}\Delta x_i)=\sum\limits_{i=1}^{n}\left(\sum\limits_{j=1}^{n}\frac{\partial^2 {f}}{\partial {x_i}\partial{x_j}}\delta x_j\right)\Delta x_i\]
    Соответствующая квадратичная форма
    \[d^2f=\sum\limits_{i,j=1}^{n}\frac{\partial^2 {f}}{\partial {x_i}\partial{x_j}} d x_i dx_j\]
    называется вторым дифференциалом функции $f$. Если $f\in \mathcal{C}^k(\Omega)$, то последовательно определяются дифференциалы до $k$ порядка.
\end{definition} 
\subsection{Формула Тейлора}
\begin{theorem} (Формула Тейлора с остаточным членом в форме Лагранжа)\\
    Пусть $f(\bar{x})\in \mathcal{C}^k(B(\bar{x}_0))$. Тогда $\forall \bar{x}\in B(\bar{x}_0)\ \exists\ \theta\in [0,1]$ такое, что
    \[f(\bar{x})=\sum\limits_{m=0}^{k-1}\frac{1}{m!}\cdot d^m f(\bar{x}_0)+\frac{1}{k!}\cdot d^k f(x_0+\theta(\bar{x}-\bar{x}_0))\] 
\end{theorem} 
\begin{proof}
    Пусть 
    \[X_1=x_{01}+(x_1-x_{01})t,\ \dots,\ X_n=x_{0n}+(x_n-x_{0n})t,\ t\in [0,1]\]
    \[F(t)=f(X_1(t),\dots, X_n(t))\]
    \begin{multline*}
        f(\bar{x})=F(1)=\sum\limits_{m=0}^{k-1}\frac{1}{m!}\cdot F^{(m)}(0)\cdot (1-0)+\frac{1}{k!}\cdot F^{(k)}(\theta)(1-0)=\\\
        =\sum\limits_{m=0}^{k-1}\frac{1}{m!}\sum\limits_{i_1,\dots,i_m=1}^{n}\frac{\partial^m {f}}{\partial {x_{i_1}} \dots \partial x_{i_m}}(\bar{x}_0)\cdot \prod\limits_{j=1}^{m}(x_{i_j}-x_{0i_j})+\tab[1.3cm]\tab[2cm]\\
        \tab[2cm]+\frac{1}{k!}\sum\limits_{i_1,\dots,i_k=1}^{n}\frac{\partial^k {f}}{\partial {x_{i_1}}\dots \partial x_{i_k}}(\bar{x}_0+\theta(\bar{x}-\bar{x}_0)) \prod\limits_{j=1}^{m}(x_{i_j}-x_{0i_j})=\\
        =\sum\limits_{m=0}^{k-1}\frac{1}{m!}\cdot d^m (x_0)+\frac{1}{k!}\cdot d^k f(x_0+\theta (\bar{x}-\bar{x}_0))
    \end{multline*}
\end{proof} 
\begin{lemma}\footnote{На лекции считалось очевидным}
    \[F^{(m)}(0)=\sum\limits_{i_1,\dots,i_m=1}^{n}\frac{\partial^m {f}}{\partial {x_{i_1}}\dots\partial x_{i_m}}(\bar{x}_0) \cdot \prod\limits_{j=1}^{m}(x_{i_j}-x_{0i_j}) \eqno(1)\]
    \[F^{(k)}(\theta)=\sum\limits_{i_1,\dots,i_k=1}^{n} \frac{\partial^k {f}}{\partial {x_{i_1}\dots \partial x_{i_k}}}((\bar{x}_0)+\theta(\bar{x}_1-\bar{x}_0))\cdot \prod\limits_{j=1}^{k}(x_{i_j}-x_{0i_j}) \eqno(2)\]
\end{lemma} 
\begin{proof}
    Индукция по $m$. База при $m=1$:
    \[F'(t)=(f(X_1(t),\dots,X_n(t)))'_t=\sum\limits_{i=1}^{n}\frac{\partial {f}}{\partial {x_i}}\cdot \frac{\partial {x_i}}{\partial {t}}=\sum\limits_{i=1}^{n}\frac{\partial {f}}{\partial {x_i}}(\bar{x}_0+t(\bar{x}_1-\bar{x}_0))(x_i-x_{0i})\]
    Пусть для $m$ верно. Тогда 
    \begin{multline*}
        F^{(m+1)}(t)=(F^{(m)}(t))'_t=\\
        =\left(\sum\limits_{i_1,\dots,i_m=1}^{n} \frac{\partial^m {f}}{\partial {x_{i_1}}\dots \partial x_{i_m}}(\bar{x}_0+t(\bar{x}-\bar{x}_0))\cdot \prod\limits_{j=1}^{m}(x_{i_j}-x_{0i_j})\right)'_t=\\
        =\sum\limits_{i_1,\dots,i_m=1}^{n} \left(\frac{\partial^m {f}}{\partial {x_{i_1}}\dots \partial x_{i_m}}(\bar{x}_0+t(\bar{x}-\bar{x}_0))\right)'_t\cdot \prod\limits_{j=1}^{m}(x_{i_j}-x_{0i_j})=\\
        =\sum\limits_{i_1,\dots,i_m,i_{m+1}=1}^{n} \left(\frac{\partial^{(m+1)} {f}}{\partial {x_{i_1}}\dots \partial x_{i_m} \partial x_{i_{m+1}}}(\bar{x}_0+t(\bar{x}-\bar{x}_0))\cdot (x_{i_{m+1}}-x_{0i_{m+1}})\right)\cdot \prod\limits_{j=1}^{m}(x_{i_j}-x_{0i_j})=\\
    =\sum\limits_{i_1,\dots,i_m,i_{m+1}=1}^{n} \left(\frac{\partial^{(m+1)} {f}}{\partial {x_{i_1}}\dots \partial x_{i_m} \partial x_{i_{m+1}}}(\bar{x}_0+t(\bar{x}-\bar{x}_0))\right)\cdot \prod\limits_{j=1}^{m+1}(x_{i_j}-x_{0i_j})
    \end{multline*}
    Отсюда, подставив нужные точки, получаем утверждения леммы.
\end{proof} 