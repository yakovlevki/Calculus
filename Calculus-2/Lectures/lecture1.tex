\section{Неопределенный интеграл}
\begin{definition}
    Пусть $f(x)$ определена на $(a, b)$. Если существует $F(x)$ определённая на $(a, b)$ такая, что $F(x) \in \mathcal{D}(a, b)$ и $F'(x) = f(x)$, то $F(x)$ называется первообразной функцией для $f(x)$.
\end{definition}
\begin{definition}
    Пусть $f(x)$ определена на $(a, b)$. Совокупность всех первообразных функций для $f(x)$ называется неопределённым интегралом $f(x)$ и обозначается 
    \[\int f(x)dx\]
\end{definition}
\begin{theorem}
    Пусть $F(x)$ является первообразной для $f(x)$ на $(a, b)$. Тогда
    \[\int f(x)dx = \{F(x) + C\},\ C = const, \ C\in\R\]
\end{theorem}
\begin{proof}
    $(F(x) + C)' = f(x) + 0 = f(x)$.\\
    Пусть $\phi(x)$ -  первообразная $f(x)$. Тогда:
    \begin{align*}
        (\phi(x) - F(x))' = f(x) - f(x) = 0
    \end{align*}
    т.е. по следствию из теоремы Лагранжа $\phi(x) - F(x) = const$, ч.т.д.
\end{proof}
\begin{statement} (Свойства неопределённого интеграла)
\begin{enumerate}
    \item $\forall c \in \R: $
    \[\int c\cdot f(x)dx = c\cdot \int f(x)dx\]
    (При $c = 0$ множества получаются разными: первое - произвольная константа, а второе - ноль; в рассуждениях этот случай будет опускаться)
    \item \[\int (f(x) \pm g(x))dx = \int f(x) dx \pm \int g(x) dx\]
    \item (Замена переменной)\\
    Пусть $F(x)$ - первообразная для $f(x)$ на $(a, b)$.\\
    Пусть $\phi(t) \in \mathcal{D}(\alpha, \beta)$ и $\phi((\alpha, \beta)) \subset (a, b)$
    Тогда $F(\phi(t))$ является первообразной для $F'(\phi(t))\cdot\phi'(t)$ на $(\alpha, \beta)$.
    \begin{align*}
        \int f(x) dx = \int f(\phi(t))\phi'(t) dt ,\text{где} \ x = \phi(t)
    \end{align*}
    \item (Интегрирование по частям)\ \ Пусть $u,v \in \mathcal{D}(a, b)$.
    \[(u\cdot v)' = u\cdot v' + u'\cdot v\]
    \[\int(uv)'dx = \int uv' dx + \int u'v dx\]
    \[\int uv' dx = uv - \int u'v dx\]
    \[\int udv = uv - \int vdu\]
\end{enumerate}
\end{statement}
\begin{comm}
    Неопределённый интеграл - операция на дифференциалах: 
    \[\int dF(x) = F(x) + C\]
\end{comm}
\subsection{Таблица неопределенных интегралов}
    \begin{tabular}{p{7.5cm}|p{0.5\textwidth}}
        \toprule
        \[\int(x^\alpha)dx = \frac{x^{\alpha + 1}}{\alpha + 1} + C, \ \alpha \neq -1\] & \[\int\frac{dx}{x} = ln|x| + \begin{cases}C_1,\ x>0\\C_2,\ x<0\end{cases}\] \\
        \midrule
        \[\int\frac{dx}{1 + x^2} = \arctg x + C\] & \[\int \cos(x)dx = \sin(x) + C\] \\
        \midrule
        \[\int\frac{dx}{\sqrt{1 - x^2}} = \arcsin x + C\] & \[\int\frac{dx}{1 - x^2} = \frac{1}{2}\ln\ \vline\frac{1+x}{1-x}\ \vline + C\] \\
        \bottomrule
    \end{tabular}
   % Когда-нибудь здесь будет полная табличка в соответствии с учебником - но не сейчас)
\subsection{Интегрирование рациональных дробей}
\[\int\frac{P(x)}{Q(x)}dx,\ P(x),\ Q(x) - \text{многочлены}\]
\[Q(x) = (x-a_1)^{\alpha_1}\dots(x-a_n)^{\alpha_n}(x^2+p_1x+q_1)^{\beta_1}\dots(x^2+p_kx+q_k)^{\beta_k}\]
\begin{multline*}
    \int\frac{P(x)}{Q(x)}dx = \int(\tilde{P} + \sum \limits_{i=1}^{\alpha_1}\frac{\aleph_{1i}}{(x-a_1)^{\alpha_{1i}}} +\dots+ \sum \limits_{i=1}^{\alpha_n}\frac{\aleph_{ni}}{(x-a_n)^{\alpha_{ni}}} + \\
    + \sum \limits_{j=1}^{\beta_1}\frac{\rho_{1j}x + \omega_{1j}}{(x^2+p_1x+q_1)^{\beta_{1i}}} + \dots + \sum \limits_{j=1}^{\beta_k}\frac{\rho_{kj}x + \omega_{kj}}{(x^2+p_1x+q_1)^{\beta_{kj}}}) dx
\end{multline*}
\begin{enumerate}
    \item \[\int\frac{dx}{(x-a)^n} = \int\frac{d(x-a)}{(x-a)^n} = \begin{cases}\ln|x-a|,\ n = 1\\ \cfrac{(x-a)^{1-n}}{1-n}, \ n > 1\end{cases}\]
    \item \[\int\frac{\alpha x+\beta}{(x^2+px+q)^k}dx = \int\frac{\alpha x+\beta}{(x^2+px+q)^k}d(x+\frac{p}{2}) = \int\frac{(\alpha_1 t+\beta_1)dt}{(t^2+q_1^2)^k}\]
    Далее осталось рассмотреть два интеграла:
    \[\int\frac{tdt}{(t^2+q_1^2)^k} = \frac{1}{2}\int\frac{dt^2}{(t^2+q_1^2)^k} = \frac{1}{2}\int\frac{d(t^2+q_1^2)}{(t^2+q_1^2)^k} = \begin{cases}\cfrac{1}{2}\ln(t^2+q_1^2),\ k = 1\\ \cfrac{(t^2+q_1^2)^{1-k}}{2(1-k)}, \ k > 1\end{cases}\]
    А второй на следующей лекции)
\end{enumerate}