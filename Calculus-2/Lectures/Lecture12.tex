\newpage
\section{Несобственный интеграл}
\subsection{Определение несобственного интеграла}
\begin{definition}
    Если $f(x)\in \mathcal{R}[a,a_1],\ \forall a_1\in [a,b)$, то 
    \[\lim\limits_{a_1\to b-0}\int\limits_{a}^{a_1}f(x)\ dx=\int\limits_{a}^{b}f(x)\ dx\]
    называется несобственным интегралом первого рода.
    Если этот предел существует, то интеграл называется сходящимся, если не существует - расходящимся. 
\end{definition} 
\begin{definition}
    Если $f(x)\in \mathcal{R}[a,a_1],\ \forall a_1\in (a, +\infty)$, то
    \[\lim\limits_{a\to +\infty} \int\limits_{a}^{a_1}f(x)\ dx=\int\limits_{a}^{b}f(x)\ dx\]
    называется несобственным интегралом второго рода. Если этот предел существует, то интеграл называется сходящимся, если не существует - расходящимся.
\end{definition} 
\begin{comm}
    В дольнейшем обычно будем рассматривать интегралы
    \[\int\limits_{a}^{\omega}f(x)\ dx\]
    где $\omega$ - число или знак\ $+\infty\ (-\infty)$.
\end{comm} 
\begin{comm}
    Если на отрезке интегрирования несобственного интеграла есть несколько особых точек, то интеграл сходится, если он сходится во всех своих особых точках. Такие интегралы также могут рассматриваться в дальнейших рассуждениях.
\end{comm}
\subsection{Критерий Коши сходимости несобственного интеграла}
\begin{theorem}
    Несобственынй интеграл
    \[\int\limits_{a}^{\omega}f(x)\ dx\]
    сходится тогда и только тогда, когда
    \[\forall \epsilon>0\ \exists\ \delta\in [a,\omega),\ \forall x_1,x_2\in [\delta, \omega): \left|\int\limits_{x_1}^{x}f(t)\ dt\right|<\epsilon\]
\end{theorem} 
\begin{proof}
    Рассмотрим функцию
    \[F(x)=\int\limits_{a}^{x}f(t)\ dt\]
    \[\int\limits_{x_1}^{x}f(t)\ dt=F(x)-F(x_1)\]
    и запишем критерий Коши существования предела $F(x)$.
\end{proof} 
\subsection{Свойства несобственного интеграла}
\setcounter{thmcount}{0}
\begin{numtheorem}
    (Линейность)\\
    $\forall \alpha,\ \beta\in \R$ если существуют
    \[\int\limits_{a}^{\omega}f(x)\ dx,\ \int\limits_{a}^{\omega}g(x)\ dx\]
    то существут интеграл
    \[\int\limits_{a}^{\omega}(\alpha f(x)+\beta g(x))=\alpha \int\limits_{a}^{\omega}f(x)\ dx+\beta \int\limits_{a}^{\omega}g(x)\ dx\]
\end{numtheorem}
\begin{numtheorem}
    Пусть $f(x),\ g(x)\in \mathcal{C}^1[a,b]$. Если существуют два объекта из трех:
    \[\int\limits_{a}^{\omega} f(x)g'(x)\ dx,\ \int\limits_{a}^{\omega}f'(x)g(x)\ dx,\ \lim\limits_{x\to \omega}f(x)g(x)\]
    то существут и третий и верна формула
    \[\int\limits_{a}^{\omega}f(x)g'(x)\ dx=f(x)g(x)|_a^{\omega}-\int\limits_{a}^{\omega}f'(x)g(x)\ dx\]
\end{numtheorem}  
\begin{numtheorem}
    Рассмотрим несобственный интеграл 
    \[\int\limits_{a}^{\omega}f(x)\ dx\]
    и пусть $\phi(t)\in \mathcal{C}^1[a,b),\ a\leq \phi(t)\leq \omega,\ \phi(\alpha)=a,\ \lim\limits_{t\to \beta}\phi(t)=\omega$
    \[\int\limits_{a}^{\omega}f(x)\ dx=\int\limits_{\alpha}^{\beta} f(\phi(t))\phi'(t)\ dt\]
\end{numtheorem} 
\begin{numtheorem}
    Пусть существуют несобственные интегралы
    \[\int\limits_{a}^{\omega}f(x)\ dx,\ \int\limits_{a}^{\omega}g(x)\ dx\]
    Если $f(x)\leq g(x)$, то
    \[\int\limits_{a}^{\omega}f(x)\ dx\leq \int\limits_{a}^{\omega}g(x)\ dx\]
\end{numtheorem} 
\begin{numtheorem}
    Если $\forall a'\in [a, \omega)$ существует
    \[\int\limits_{a}^{a'}f(x)\ dx\]
    и если существут несобственный интеграл
    \[\left|\int\limits_{a}^{\omega}f(x)\ dx\right|\leq \int\limits_{a}^{\omega}|f(x)|\ dx\]
\end{numtheorem} 
\begin{definition}
    Если существует интеграл
    \[\int\limits_{a}^{\omega}|f(x)|\ dx\]
    то интеграл
    \[\int\limits_{a}^{\omega}f(x)\ dx\]\
    называется абсолютно сходящимся.
\end{definition} 
\begin{statement}
    Если $\forall a'\in [a, \omega]$ существует интегал
    \[\int\limits_{a}^{a'}f(x)\ dx\]
    и интеграл
    \[\int\limits_{a}^{\omega}|f(x)|\ dx\]
    сходится, то интерал
    \[\int\limits_{a}^{\omega}f(x)\ dx\]
    сходится
\end{statement} 
\begin{proof}
    По критерию Коши и неравентсву
    \[\left|\int\limits_{a_1}^{a_2}f(x)\ dx\right|\leq \int\limits_{a_1}^{a_2}|f(x)|\ dx<\epsilon\]
\end{proof} 
\begin{definition}
    Если интеграл 
    \[\int\limits_{a}^{\omega}f(x)\ dx\]
    сходится, а интеграл 
    \[\int\limits_{a}^{\omega}|f(x)|\ dx\]
    расходится, то интеграл 
    \[\int\limits_{a}^{\omega}f(x)\ dx\]
    называется условно сходящимся.
\end{definition} 
%ПРИМЕР НЕОГРАНИЧЕНОЙ СХОДЯЩЕЙСЯ ФУНКЦИИ
\subsection{Признаки сходимости несобственных интегралов}
\begin{theorem}
    (Признак Вейерштрасса)\\
    Пусть $0\leq f(x)\leq g(x)$ на $[a],\omega$.\\
    Если существует интеграл
    \[\int\limits_{a}^{\omega}g(x)\ dx\]
    то существует интеграл
    \[\int\limits_{a}^{\omega}f(x)\ dx\]
    Если не существует интеграла
    \[\int\limits_{a}^{\omega}f(x)\ dx\]
    то не существут интеграла
    \[\int\limits_{a}^{\omega}g(x)\ dx\]
\end{theorem} 
\begin{proof}
    По теореме Вейерштрасса.
\end{proof}
\begin{theorem}
    Пусть $0\leq f(x)\leq g(x)$ и существует передел
    \[\lim\limits_{x\to \omega}\frac{f(x)}{g(x)}=1\]
    Тогда интегралы
    \[\int\limits_{a}^{\omega}f(x)\ dx,\ \int\limits_{a}^{\omega}g(x)\ dx\] 
    существуют или не существуют одновременно.
\end{theorem} 
\begin{proof}
    $\forall \epsilon>0\ \exists\ A,\ \forall x\in [A, \omega)$:
    \[|\frac{f(x)}{g(x)}-1|<\epsilon \Rightarrow 1-\epsilon<\frac{f(x)}{g(x)}<1+\epsilon\]
    значит
    \[(1-\epsilon)\cdot g(x)<f(x)<(1+\epsilon)\cdot g(x)\]
    далее воспользуемся признаком Вейерштрасса.
\end{proof} 
\begin{theorem}
    (Признаки Абеля и Дирихле)\\
    Пусть $f(x)\in \mathcal{C}[a,\omega),\ g(x)$ монотонна на $[a,\omega)$, а также $|f(x)|\leq C,\ |g(x)|\leq C$. Тогда 
    \begin{itemize}
        \item[($\mathcal{A}$):] Если существут интеграл 
        \[\int\limits_{a}^{\omega}f(x)\ dx\]
        то существует интеграл
        \[\int\limits_{a}^{\omega}f(x)g(x)\ dx\]
        \item[($\mathcal{D}$):] Если $g(x)\to 0$ при $x\to \omega$ и
        \[\left|\int\limits_{a}^{a'}f(x)\ dx\right|\leq C,\ \forall a'\in [a,\omega]\]
        то существует интеграл
        \[\int\limits_{a}^{\omega}f(x)g(x)\ dx\]
    \end{itemize}
\end{theorem} 
\begin{proof}
    \begin{multline*}
        \left|\int\limits_{x_1}^{x_2}f(x)g(x)\ dx\right|=\left|g(x_1)\cdot \int\limits_{x_1}^{c}f(x)\ dx+g(x_2)\cdot \int\limits_{c}^{x_2}f(x)\ dx\right|\leq\\\
        \leq |g(x_1)|\cdot \left|\int\limits_{x_1}^{c}f(x)\ dx\right|+|g(x_2)|\cdot \left|\int\limits_{c}^{x^2}f(x)\ dx\right|\leq
    \end{multline*} 
    $(\mathcal{A}):\ \leq C\cdot 2\epsilon$\\
    $(\mathcal{D}):\ \leq \epsilon \cdot 4C$
\end{proof} 
\begin{example} Интеграл 
    \[\int\limits_{1}^{+\infty}\frac{\sin{x}}{x}\ dx\]
    сходится, но интеграл
    \[\int\limits_{1}^{+\infty}\left|\frac{\sin{x}}{x}\right|\ dx\]
    расходится, так как
    \[\frac{1}{2x}-\frac{\cos{2x}}{2x}=\frac{1-\cos{2x}}{2x}=\frac{\sin^2{x}}{x}\leq \frac{|\sin{x}|}{x}\]
\end{example}
\subsection{Главное значение интеграла в смысле Коши}
\begin{definition}
    Пусть $f(x)$ определена на $\R$ и $f(x)\in \mathcal{R}[a,b],\ \forall a,b$. Величина
    \[\lim\limits_{A\to +\infty}\int\limits_{-A}^{A}f(x)\ dx=\text{v.p.} \int\limits_{-\infty}^{+\infty}f(x)\ dx\]
    называется главным значением интеграла в смысле Коши. % Интегралом несобств не является
\end{definition} 
\begin{theorem}
    Главное значение интеграла в смысле Коши существует, если сходится несобственный интеграл
    \[\int\limits_{0}^{+\infty}(f(x)+f(-x))\ dx\]
\end{theorem} 
\begin{proof}
    \[f(x)=\frac{f(x)+f(-x)}{2}+\frac{f(x)-f(-x)}{2}\]
    Возьмем интегал от $-A$ до $A$, второе слагаемое всегда ноль.
\end{proof} 
\begin{comm}
    Аналогочно определяется главное значение с особенностью в точке $c\in (a,b)$:
    \[\text{v.p.}\int\limits_{a}^{b}f(x)\ dx=\lim\limits_{\epsilon\to 0+0}\left(\int\limits_{a}^{c-\epsilon}f(x)\ dx+\int\limits_{c+\epsilon}^{b}f(x)\ dx\right)\]
\end{comm} 