\subsection{Критерий Лебега интегрируемости по Риману}
\begin{definition}
    Пусть $A\subset \R$, и если $\forall \epsilon>0\ \exists\ \{(a_i,b_i)\}_{i=1}^{\infty}$ (или конечное) таких, что 
    \[A\subset \bigcup\limits_i(a_i,b_i),\ \sup\limits_n \sum\limits_{i=1}^{n}|b_i-a_i|< \epsilon\]
    Тогда $A$ называется множеством меры 0 по Лебегу. Обозначается $\mu(A)=0$.
\end{definition} 
\begin{theorem} (Свойства множеств с мерой 0 по Лебегу)
    \begin{enumerate}
        \item $B\subset A,\ \mu(A)=0 \Rightarrow \mu(B)=0$
        \item $\{A_i\}_{i=1}^{\infty},\ \mu(A_i)=0 \Rightarrow \mu(\bigcup\limits_{i=1}^{\infty} A_i)=0$
    \end{enumerate}
\end{theorem} 
\begin{proof} \
    \begin{enumerate}
        \item Очевидно
        \item $\forall i\ \exists\ \{(a_{i_l},b_{i_l})\}_{i=1}^{\infty}:$
        \[A_i\subset \bigcup\limits_{l=1}^{\infty} (a_{i_l},b_{i_l}),\ \sum\limits_{l=1} |b_{i_l}-a_{i_l}|<\frac{\epsilon}{2^i}\]
        \[\bigcup\limits_{i=1}^{\infty} A_i \subset \bigcup\limits_{i=1}^{\infty}\left(\bigcup\limits_{l=1}^{\infty} (a_{i_l},b_{i_l})\right),\ \sum\limits_{i=1}^{\infty}\left(\sum\limits_{l=1}^{\infty} |b_{i_l}-a_{i_l}|\right)<\sum\limits_{i=1}^{\infty}\frac{\epsilon}{2^i}=\epsilon\]
    \end{enumerate}
\end{proof} 
\begin{theorem} (Критерий Лебега интегрируемости по Риману)\\
    $f(x)\in \mathcal{R}[a,b] \Leftrightarrow f(x)$ ограничена и для множества $P$ точек разрыва функции $f(x)$ выполнено $\mu(P)=0$.
\end{theorem} 
\begin{proof}
    Без доказательства.
\end{proof} 
\subsection{Свойства интеграла Римана}
\begin{numtheorem} (Интегрируемость на подотрезках)\\
    Если $f(x)\in \mathcal{R}[a,b],\ [c,d]\subset [a,b]$, то $f(x)\in \mathcal{R}[c,d]$.
\end{numtheorem}
\begin{proof}
    Так как $f(x)\in \mathcal{R}[a,b]$, то $\forall\ T_{[a,b]}(\xi): \sigma_f(T_{[a,b]}(\xi))\to I$. Значит, если $\{c,d\}\in T_{[a,b]}$, то $\sigma_f(T_{[a,b]\cup \{c,d\}}(\xi)):$
    \begin{multline*}
        \epsilon> \underline{\underline{S}}_{[a,b]\cup \{c,d\}}-\overline{\overline{S}}_{{[a,b]\cup \{c,d\}}}=\sum\limits_{k=1}^{i}(M_k-m_k)(x_k-x_{k-1})+\sum\limits_{k=i+1}^{j}(M_k-m_k)(x_k-x_{k-1})+\\
        +\sum\limits_{k=j+1}^{N}(M_k-m_k)(x_k-x_{k-1})\geq \sum\limits_{k=i+1}^{j}(M_k-m_k)(x_k-x_{k-1})=\underline{\underline{S}}_{[c,d]}-\overline{\overline{S}}_{[c,d]}
    \end{multline*}
\end{proof}
\begin{numtheorem} (Аддитивность)\\
    Если $f(x)\in \mathcal{R}[a,b],\ c\in [a,b]$, то 
    \[\int\limits_{a}^{b}f(x)\ dx=\int\limits_{a}^{c} f(x)\ dx+\int\limits_{c}^{b}f(x)\ dx\]
\end{numtheorem}
\begin{proof}
    Пусть $c\in T_{[a,b]}(\xi)$. Тогда
    \[\sigma_f(T_{[a,b]})=\sigma_f(T_{[a,c]})+\sigma_f(T_{[c,b]})\]
    \[\sigma_f(T_{[a,c]}) \to \int\limits_{a}^{c}f(x)\ dx,\ \sigma_f(T_{[c,b]})\to \int\limits_{c}^{b}f(x)\ dx\]
    а также
    \[\sigma_f(T_{[a,b]})\to \int\limits_{a}^{b}f(x)\ dx\]
    Теперь пусть $c\not\in T_{[a,b]}$. Рассмотрим $T_{[a,b]\cup c}'=T_{[a,b]}\cup \{c\}$. Тогда при $d\to 0$
        \[\sigma_f(T_{[a,b]})-\sigma_f(T_{[a,b]\cup c}')=f(\xi_j)(x_j-x_{j-1})-f(\xi_j')(c-x_{j-1})-f(\xi_j'')(x_j-c)\to 0\]
\end{proof} 
\begin{comm}
    Если $f(x)\in \mathcal{R}[a,c],\ b<c$, то
    \[\int\limits_{a}^{b}f(x)\ dx=\int\limits_{a}^{c}f(x)\ dx+\int\limits_{c}^{b}f(x)\ dx\]
\end{comm} 
\begin{numtheorem} (Линейность)\\
    Пусть $f(x),g(x)\in \mathcal{R}[a,b]$. Тогда $\forall \alpha,\beta \in \R: \alpha f(x)+\beta g(x) \in \mathcal{R}[a,b]$
    \[\int\limits_{a}^{b}(\alpha f(x)+\beta g(x))\ dx=\alpha\cdot \int\limits_{a}^{b} f(x)\ dx+\beta\cdot \int\limits_{a}^{b}g(x)\ dx\]
\end{numtheorem}
\begin{proof}
    \[\sigma_{\alpha f(x)+\beta g(x)}(T)=\alpha\sigma_f(T)+\beta\sigma_g(T)\]
\end{proof}
\begin{numtheorem}
    Пусть $f(x)\in \mathcal{R}[a,b],\ f(x)\geq 0$. Тогда
    \[\int\limits_{a}^{b}f(x)\ dx\geq 0\]
\end{numtheorem}
\begin{proof}
    \[f(x)\geq 0 \Rightarrow \sigma_f(T)\geq 0 \Rightarrow \int\limits_{a}^{b} f(x)\ dx \geq 0\]
\end{proof} 
\begin{consequense}
    Если $f(x), g(x)\in \mathcal{R}[a,b]$ и $f(x)\geq g(x)$ на $[a,b]$, то 
    \[\int\limits_{a}^{b}f(x)\ dx\geq \int\limits_{a}^{b}g(x)\ dx\]
\end{consequense}
\begin{numtheorem}
    Пусть $f(x)\in \mathcal{R}[a,b],\ f(x)\geq 0,\ \exists\ c\in[a,b]$, что $f(x)$ непрерывна в точке $c$ и $f(c)>0$. Тогда 
    \[\int\limits_{a}^{b}f(x)\ dx>0\]
\end{numtheorem}
\begin{proof}
    По теореме об отделимости\\
    $\exists\ \delta>0: f(x)>\frac{f(c)}{2}$ в $(c-\delta, c+\delta):$
    \[\int\limits_{a}^{b}f(x)\ dx\geq \int\limits_{c-\delta}^{c+\delta}f(x)\ dx>\int\limits_{c-\delta}^{c+\delta}\frac{f(c)}{2}\ dx=\frac{f(c)}{2}\cdot 2\delta=\delta f(c)>0\]
\end{proof}
\begin{numtheorem}
    $f(x), g(x)\in \mathcal{R}[a,b]$. Тогда $f(x)\cdot g(x)\in \mathcal{R}[a,b]$
\end{numtheorem}
\begin{proof} Пусть 
    \[M_1=\sup\limits_{[a,b]}|f(x)|,\ M_2=\sup\limits_{[a,b]}|g(x)|\]
    Ограничим значение $\underline{\underline{S}}_{f\cdot g}-\overline{\overline{S}}_{f\cdot g}$, ограничив разность точных граней на одном отрезке разбиения: (далее супремум рассматривается по всем $x',x''\in[x_i,x_{i-1}]$)
    \begin{multline*}
        M_i(f(x)g(x))-m_i(f(x)g(x))=\sup(f(x')g(x')-f(x'')g(x''))=\\
        =\sup(f(x')g(x')-f(x')g(x'')+f(x')g(x'')-f(x'')g(x''))=\tab[3cm]\\
        =\sup(f(x')(g(x')-g(x''))+g(x'')(f(x')-f(x'')))\leq\tab[1.5cm]\\
        \tab[3cm]\leq \sup|f(x)|\cdot \sup(g(x')-g(x''))+\sup|g(x)|\cdot \sup(f(x')-f(x''))\leq \\
        \leq M_1 (M_{ig}-m_{ig})+M_2(M_{if}-m_{if})
    \end{multline*}
    Отсюда, домножив неравенства на длины соответствующих отрезков и сложив, получим
    \begin{equation*}
        \underline{\underline{S}}_{f\cdot g}-\overline{\overline{S}}_{f\cdot g}\leq M_1(\underline{\underline{S}}_g-\overline{\overline{S}}_g)+M_2(\underline{\underline{S}}_f-\overline{\overline{S}}_f)
    \end{equation*}
    Отсюда из интегрируемости $f$ и $g$ и критерия Дарбу $f(x)g(x) \in \mathcal{R}[a, b]$.  
\end{proof}