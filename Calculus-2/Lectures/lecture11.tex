\begin{numtheorem}
    (Аддитивность)\\
    Если существуют 
    \[\int\limits_{a}^{b}f(x)\ d(g(x)),\ \int\limits_{a}^{c}f(x)\ d(g(x)),\ \int\limits_{c}^{b}f(x)\ d(g(x))\]
    то 
    \[\int\limits_{a}^{b}f(x)\ d(g(x))=\int\limits_{a}^{c}f(x)\ d(g(x)) + \int\limits_{c}^{b}f(x)\ d(g(x))\]
\end{numtheorem} 
\begin{proof}
\begin{multline*}
    \int\limits_{a}^{b}f(x)\ d(g(x))=\lim\limits_{d\to 0, c\in T}\sum\limits_{i=1}^{n}f(\xi_i)(g(x_i)-g(x_{i-1}))=\tab[3cm]\\
    =\lim\limits_{d\to 0}\sum\limits_{i=1}^{n_1}f(\xi_i)(g(x_i)-g(x_{i-1}))+\lim\limits_{d\to 0}\sum\limits_{i=n_1+1}^{n}f(\xi_i)(g(x_i)-g(x_{i-1}))=\\=\int\limits_{a}^{c}f(x)\ d(g(x)) + \int\limits_{c}^{b}f(x)\ d(g(x))
\end{multline*}
\end{proof} 
\begin{comm}
    Если существует интеграл
    \[\int\limits_{a}^{b}f(x)\ d(g(x))\]
    то существуют интегралы
    \[\int\limits_{a}^{c}f(x)\ d(g(x)),\ \int\limits_{b}^{c}f(x)\ d(g(x))\]
\end{comm} 
\begin{comm}
    Если существуют интегралы
    \[\int\limits_{a}^{c}f(x)\ d(g(x)),\ \int\limits_{c}^{b}f(x)\ d(g(x))\]
    то интеграл
    \[\int\limits_{a}^{b}f(x)\ d(g(x))\]
    не обязательно существует.
\end{comm} 
\begin{example}
    \[f(x)=\begin{cases}
        0,\ x\in [-1,0],\\
        1,\ x\in (0, 1].
    \end{cases}\
        g(x)=\begin{cases}
            0,\ x\in [-1,0),\\
            1,\ x\in [0, 1].
    \end{cases}
    \]
    \[\int\limits_{-1}^{0}f(x)\ d(g(x))=\lim\limits_{d\to 0}\sum\limits_{i=1}^{n}f(\xi_i)(g(x_{i})-g(x_{i-1}))=0\]
    \[\int\limits_{0}^{1}f(x)\ d(g(x))=\lim\limits_{d\to 0}\sum\limits_{i=1}^{n}f(\xi_i)(g(x_{i})-g(x_{i-1}))=0\]
    \[\int\limits_{-1}^{1}f(x)\ d(g(x))=\lim\limits_{d\to 0}\sum\limits_{i=1}^{n}f(\xi_i)(g(x_{i})-g(x_{i-1}))=\lim\limits_{d\to 0} f(\xi_j)(g(x_j)-g(x_{j-1}))\]
    При разной разметке будет получаться 1 или 0, значит, предела не существует.
\end{example}
\begin{numtheorem}
    (Интегрирование по частям)\\
    Если существут интеграл
    \[\int\limits_{a}^{b}f(x)\ d(g(x))\]
    то существует интеграл
    \[\int\limits_{a}^{b}g(x)\ d(f(x))\]
    причем
    \[\int\limits_{a}^{b}f(x)\ d(g(x))=f(x)\cdot g(x)|_a^b - \int\limits_{a}^{b}g(x)\ d(f(x))\]
\end{numtheorem} 
\begin{proof}
    \begin{multline*}
        \sum\limits_{i=1}^{n}g(\xi_i)(f(x_i)-f(x_{i-1}))=\\
        =g(\xi_1)(f(x_1)-f(a))+g(\xi_2)(f(x_2)-f(x_1))+\dots+\tab[5cm]\\
        +g(\xi_{n-1})(f(x_{n-1})-f(x_{n-2}))+g(\xi_{n})(f(b)-f(x_{n-1}))=\tab[2cm]\\
        =-g(\xi_1)f(a)-f(x_1)(g(\xi_2)-g(\xi_1))-\dots-f(x_{n-1})(g(\xi_n)-g(\xi_{n-1}))+\\
        +g(\xi_n)f(b)+f(b)g(b)-f(b)g(b)-f(a)g(a)+f(a)g(a)=\\ %добавили и вычли
        =-f(a)(g(\xi_1)-g(a))-\dots-f(b)(g(b)-g(\xi_n))+f(b)g(b)-f(a)g(a)
    \end{multline*}
    Устремим диаметр разбиения к нулю, и получим утверждение теоремы.
\end{proof} 
\subsection{Существование интеграла Римана-Стилтьеса}
В дальнейших рассуждениях будут использоваться обозначения
\[\underline{\underline{S}}(T)=\sum\limits_{i=1}^{n}M_i(g(x_i)-g(x_{i-1})),\ \overline{\overline{S}}(T)=\sum\limits_{i=1}^{n}m_i(g(x_i)-g(x_{i-1}))\] 
\begin{theorem}
    Если $f(x)\in \mathcal{C}[a,b],\ g(x)\in \mathcal{V}[a,b]$, то существует интеграл
    \[\int\limits_{a}^{b}f(x)\ d(g(x))\]
\end{theorem}
\begin{proof}
    $f(x)\in \mathcal{C}[a,b]$, значит, $f(x)$ - равномерно непрерывна на $[a,b]$:
    \[\forall \epsilon>0\ \exists\ \delta>0,\ d(T)<\delta: M_i-m_i<\epsilon\]
    а значит
    \[\underline{\underline{S}}(T)-\overline{\overline{S}}(T)<\epsilon\cdot \variation\limits_{a}^{b}g(x) \Rightarrow \inf\limits_T \underline{\underline{S}}(T)=\sup\limits_T \overline{\overline{S}}(T)=I\]
    \[|\sigma_f(g,T)-I|<\epsilon\cdot \variation\limits_{a}^{b}g(x)\]
    значит, существует предел
    \[\lim\limits_{d\to 0}\sigma_f(g,T)=I\]
\end{proof} 
\begin{comm}
    В условиях теоремы существует интеграл
    \[\int\limits_{a}^{b} g(x)\ d(f(x))\]
    по теореме об интегрировании по частям.
\end{comm} 
\subsection{Связь интеграла Римана и интеграла Римана-Стилтьеса}
Знаки $(R)$ и $(S)$ обозначают, что рассматривается интеграл по Риману и Риману-Стилтьесу соответственно.
\begin{theorem}
    Если $f(x)\in \mathcal{V}[a,b],\ g(x)\in \mathcal{D}[a,b],\ g'(x)\in \mathcal{R}[a,b]$, то
    \[(S)\int\limits_{a}^{b}f(x)\ d(g(x))=(R)\int\limits_{a}^{b}f(x)\cdot g'(x)\ dx\]
\end{theorem} 
\begin{proof}
    Оба интеграла существуют по условиям теоремы, значит, достаточно будет доказать для какой-то выбранной разметки. По формуле Лагранжа:
    \[\sum\limits_{i=1}^{n}f(\xi_i)(g(x_i)-g(x_{i-1}))=\sum\limits_{i=1}^{n}f(\xi_i)\cdot g'(\zeta)(x_i-x_{i-1})\]
    Теперь возьмем в качестве разметки интеграла Римана-Стилтьеса разметку $\zeta_i$, и получаем равенство
    \[\sum\limits_{i=1}^{n}f(\zeta_i)(g(x_i)-g(x_{i-1}))=\sum\limits_{i=1}^{n}f(\zeta_i)g'(\zeta_i)(x_i-x_{i-1})\]
\end{proof} 
\subsection{Теоремы о среднем}
\begin{theorem}
    (Первая теорема о среднем для интеграла Римана-Стилтьеса)\\ 
    Пусть $f(x)\in \mathcal{C}[a,b],\ g(x)$ монотонно возрастает. Тогда $\exists\ c\in [a,b]$ такая, что
    \[\int\limits_{a}^{b}f(x)\ d(g(x))=f(c)(g(b)-g(a))\]
\end{theorem} 
\begin{proof}
    Если $g(b)=g(a)$, то равенство верно.
    Пусть $g(b)>g(a),\\ m=\min\limits_{[a,b]} f(x), M=\max\limits_{[a,b]} f(x)$
    \[m\cdot(g(b)-g(a))\leq\int\limits_{a}^{b}f(x)\ d(g(x))\leq M\cdot(g(b)-g(a))\]
    отсюда
    \[m\leq \frac{\int\limits_{a}^{b}f(x)\ d(g(x))}{g(b)- g(a)}\leq M\]
\end{proof} 
\begin{theorem}
    (Вторая теорема о среднем для интеграла Римана-Стилтьеса)\\
    Пусть $f(x)\in \mathcal{C}[a,b],\ g(x)$ монотонно возрастает. Тогда $\exists\ c\in [a,b]$ такая, что
    \[\int\limits_{a}^{b}g(x)\ d(f(x))=g(a)\cdot(f(c)-f(a))+g(b)\cdot (f(b)-f(c))\]
\end{theorem} 
\begin{proof} Возьмем интеграл по частям и применим первую теорему о среднем:
    \begin{multline*}
        \int\limits_{a}^{b}g(x)\ d(f(x))=f(x)\cdot g(x)|^b_a-\int\limits_{a}^{b}f(x)\ d(g(x))=\\=
        g(b)f(b)-g(a)f(a)-f(c)(g(b)-g(a))=\\
        =g(a)(f(c)-f(a))+g(b)(f(b)-f(c))
    \end{multline*}
\end{proof} 
\begin{consequense}
    (Вторая теорема о среднем для интеграла Римана)\\
    Пусть $f(x)\in \mathcal{C}[a,b],\ g(x)$ монотонно возрастает. Тогда
    \[(R)\int\limits_{a}^{b}f(x)\cdot g(x)\ dx=g(a)\cdot \int\limits_{a}^{c}f(x)\ dx+g(b)\cdot \int\limits_{c}^{b}f(x)\ dx\]
\end{consequense} 
\begin{proof} Перейдем к интегралу Римана-Стилтьеса и воспользуемся второй теоремой о среднем для него:
    \begin{multline*}
        (R)\int\limits_{a}^{b}f(x)\cdot g(x)\ dx=(S) \int\limits_{a}^{b}g(x)\ d\left(\int\limits_{a}^{x}f(t)\ dt\right)=\\
        =g(a)\cdot \int\limits_{a}^{c}f(t)\ dt+g(b)\cdot \int\limits_{c}^{b}f(t)\ dt
    \end{multline*}
\end{proof} 