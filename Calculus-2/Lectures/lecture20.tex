\subsection{Условный экстремум}
\begin{definition}
Пусть функции $f_i$ определены на $B(\bar{x}_0),\ i=0,\dots, m,\\
\Omega_F=\{\bar{x}:\ f_i(\bar{x})=0,\ i=1,\dots,m\},\ \bar{x}_0\in \Omega_F$. Если
\[\forall \delta>0,\ \forall \bar{x}\in \mathring{B}_{\delta}(\bar{x}_0)\cap \Omega_f: f_0(\bar{x}) >\ (<)\ f_0(\bar{x}_0)\]
то говорят, что $f_0(\bar{x})$ имеет в точке $\bar{x}_0$ условный минимум (максимум), а уравнения $f_i(\bar{x})=0,\ i=1,\dots,m$ называют условиямми связи.
\end{definition} 
\begin{comm}
    Пусть $f_i(\bar{x})\in \mathcal{C}(B(\bar{x}_0)),\ m<n,\ \{\grad f_i\}_{i=1}^m$ - линейно независимы, тогда существует
    \[\frac{D(f_1,\dots, f_m)}{D(x_1,\dots,x_m)}\ne 0 \Rightarrow \begin{cases}
        f_1=0,\\
        \vdots\\
        f_m=0.
    \end{cases}
    \Rightarrow \begin{cases} % по теореме о неявной функции
        x_1=\phi_1(x_{m+1},\dots,x_n),\\
        \vdots\\
        x_m=\phi_1(x_{m+1},\dots,x_n).
    \end{cases}
    \]
    Значит
    \[f_0(\phi_1,\dots,\phi_m,x_{m+1},\dots,x_n)=g(x_{m+1},\dots,x_n)\]
    и нужно найти безусловный экстремум функции $g$, которая зависит от меньшего числа переменных. 
\end{comm} 
\begin{theorem} (Необходимое условие экстремума)\\ %ВЕзде считаем что градиенты ЛНЗ по $_i=0^m$
    Пусть $f_i\in \mathcal{C}(B(\bar{x}_0)),\ i=0,\dots,m$. Если $f_0$ имеет в точке $\bar{x}_0$ условный экстремум, то $\{\grad f_i\}_{i=0}^m$ - линейно зависимы
\end{theorem} 
\begin{proof}
    Пусть существует экстремум в точке $\bar{x}_0$, а градиенты линейно независимы. Тогда
    \[\frac{D(f_1,\dots,f_m,f_0)}{D(x_1,\dots,x_m,x_{m+1})}\ne 0 \Rightarrow
    \begin{cases}
        y_1=f_1,\\
        \vdots\\
        y_m=f_m,\\
        y_{m+1}=f_0
    \end{cases}    
    \]
    отсюда
    \[\begin{cases}
        x_1=\phi_1(y_1,\dots,y_m,y_{m+1},x_{m+2},\dots,x_n),\\
        x_2=\phi_2(y_1,\dots,y_m,y_{m+1},x_{m+2},\dots,x_n),\\
        \vdots\\
        x_m=\phi_m(y_1,\dots,y_m,y_{m+1},x_{m+2},\dots,x_n),\\
        x_{m+1}=\phi_{m+1}(y_1,\dots,y_m,y_{m+1},x_{m+2},\dots,x_n).
    \end{cases}\]
    Рассотрим обратное отображение в точку $\bar{x}_0$:
    \[\begin{cases}
        x_{01}=\phi_1(\underbrace{0,0,\dots,0}_m,f_0(\bar{x}_0), x_{0m+2}, \dots, x_{0n}),\\ %нули получаются из условия связи, если подставить в него икс 0
        x_{02}=\phi_2(\underbrace{0,0,\dots,0}_m,f_0(\bar{x}_0), x_{0m+2}, \dots, x_{0n}),\\
        \vdots\\
        x_{0m}=\phi_{m}(\underbrace{0,0,\dots,0}_m,f_0(\bar{x}_0), x_{0m+2}, \dots, x_{0n}),\\
        x_{0m+1}=\phi_{m+1}(\underbrace{0,0,\dots,0}_m,f_0(\bar{x}_0), x_{0m+2}, \dots, x_{0n})
    \end{cases}
    \]
    Теперь возьмем точку, близкую к $\bar{x}_0$
    \[
    \begin{cases}
        x_{\delta_1}=\phi_1(\underbrace{0,0,\dots,0}_m,f_0(\bar{x}_0)+\delta,x_{\delta_{m+2}},\dots,x_{\delta_n}),\\
        \vdots\\

    \end{cases}
    \]
    Если взять $\delta<0$, то получим точку, в которой значение больше $f_0(\bar{x}_0)$, а если $\delta>0$, то получим точку, в которой значение меньше $f_0(\bar{x}_0)$.
\end{proof} 
\subsection{Метод множителей Лагранжа}
\begin{definition}
    Рассмотрим задачу на условный экстремум, $f_i=0$. Функция 
    \[L(\bar{x},\bar{\lambda})=f_0(\bar{x})+\sum\limits_{i=1}^{m}\lambda_i\cdot f_i(\bar{x})\]
    называется функцией Лагранжа задачи об условном экстремуме.
\end{definition} 
\begin{theorem}
    Если $f_0$ имееет условный экстремум в точке $\bar{x}_0$ при условии связи $f_i=0,\ i=1,\dots,m$, то существует набор констант $\{\lambda_i\}_{i=1}^m$ таких, что 
    \[dL=(\bar{x}_0,\bar{\lambda})=0\]
\end{theorem} 
\begin{proof}
    Если выполнено условие теоремы, то $\{\grad f_i\}_{i=0}^m$ - линейно зависимы, но $\{\grad f_i\}_{i=1}^m$ - линейно независимы, значит,
    \[dL=df_0+\sum\limits_{i=1}^{m}\lambda_i\cdot df_i\]
    то есть существует набор $\{\lambda_i\}$, что $dL=0$. 
\end{proof} 
\begin{theorem} (Достаточное условие)\\
    Рассмотрим задачу на условный экстремум, $f_i=0,\ i=1,\dots m,\ f_i\in \mathcal{C}^2(B(\bar{x}_0)),\ i=0,\dots,m$ Если
    \[d^2L|_{df_1=\dots=df_m=0}\]
    знакоопределен, то существует условный экстремум. Если он знаконеопределен, то экстремума не существует.
\end{theorem} 
\begin{proof}
    Без доказательсва.
\end{proof} 
\subsection{Абсолютный экстремум}
\begin{definition}
    Пусть $f(\bar{x})$ определена на $A\subset \R^n$. Если $\exists\ \sup\limits_A f(\bar{x})\ (\inf\limits_A f(\bar{x}))$ и $\exists\ \bar{x}_{max}$ такой, что $f(\bar{x}_{max})=\sup\limits_A f(\bar{x})\ (\dots)$, то говорят, что $f(\bar{x})$ имеет абсолютный минимум (максимум) на $A$.
    \\ Если $A$ - компакт, то ищем локальные максимумы и минимумы среди точек, имеющих полную окрестность; далее рассматриваем подмножество $A$ на единицу меньшей размерности и там повторяем процедуру, и так далее до одноточечных множеств.
\end{definition} 
