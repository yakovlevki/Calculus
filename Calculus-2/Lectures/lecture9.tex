\begin{theorem}
    (2-й критерий квадрируемости)\\
    Фигура $A$ квадрируема $\Leftrightarrow \mu(\partial A)=0$.
\end{theorem}
\begin{proof}\tab
    \begin{itemize}
        \item[$(\Rightarrow):$] $A$ - квадрируема $\Rightarrow$ по первому критерию квадрируемости:
        \[\forall \epsilon>0\ \exists\ P_{\epsilon},\ Q_{\epsilon}: \mu(P_{\epsilon})-\mu(Q_{\epsilon}) <\epsilon\]
        $\partial A\subset P\setminus Q,\ Q$ - внутренние точки $A,\ \R^2\setminus P$ - внешние точки $A$.\\
        В частности,\ $\partial A\subset P_{\epsilon}\setminus Q_{\epsilon} \Rightarrow \mu^*(\partial A)<\epsilon \Rightarrow \mu(\partial A)=0$.
        \item[$(\Leftarrow):$] 
        $\mu(\partial A)=0 \Rightarrow \forall \epsilon>0\ \exists\ P_{\epsilon}\supset \partial A,\ \mu(P_{\epsilon})<\epsilon \Rightarrow \exists\ h>0,\\
        \partial A\subset \cup (\text{кв. сетка с шагом h})=A_2: \mu(A_2)<72\epsilon$ (по лемме ниже).\\
        $A_1=\cup(\text{квадраты сетки, целиком состоящие из внутренних точек A})\\
        \Rightarrow A\subset A_1\cup A_2 \Rightarrow A_1\cup A_2=P,\ A_1=Q,\ \mu(P)-\mu(Q)=\mu(A_2)<72\epsilon$
    \end{itemize}
\end{proof} 
\begin{lemma}
    Если $B$ покрывается $P$ с $\mu(P)<\epsilon$, то существует $h>0$ такое, что $B\subset \cup$(кв. сетка с шагом h),\ $\mu(\cup(\text{кв. сетка с шагом h}))<72\epsilon$.
\end{lemma}   
\begin{proof}
    $P$ - фигура $\Rightarrow P$ - это объединение треугольников\\
    $\Rightarrow P$ - объединение прямоугольных треугольников с $\mu<\epsilon \Rightarrow P$ лежит в объединении прямоугольников с $\mu<2\epsilon \Rightarrow P$ лежит в объединении квадратов с $\mu<4\epsilon \Rightarrow P$ лежит в объединении квадратов со сторонами, параллельными осям координат с $\mu<8\epsilon \Rightarrow$ возьмем $h$, равное стороне наименьшего квадрата, и построим сетку с шагом $h \Rightarrow \mu(\cup(\text{кв. сетка с шагом h}))<72\epsilon$. 
\end{proof} 
\subsection{Квадрируемость простой спрямляемой кривой и криволинейной трапеции}
\begin{theorem}
    Если $\bar{\gamma}(t)$ - простая спрямляемая фигура, то $\mu(\bar{\gamma}(t))=0$.
\end{theorem} 
\begin{proof}
    Делим $\bar{\gamma}(t)$ на $n$ одинаковых по длине кусков. $\{\bar{\gamma}(t_k)\}_{k=1}^{n+1}$. $\bar{\gamma}(t)\subset \cup(\text{квадратов с центрами в}\  \bar{\gamma}(t_k)\ \text{и стороной}\ |\frac{2\bar{\gamma}(t)|}{n}|)$. 
    \[\mu(\cup(\text{кв...}))<\frac{4|\bar{\gamma}(t)|^2}{n^2}\cdot (n+1)\to 0\]
\end{proof} 
\begin{theorem}
    Пусть $f(x)\in \mathcal{R}[a,b],\ f(x)\geq 0$, тогда фигура $A:$ 
    \[A=\{(x,y): x\in[a,b],\ 0\leq y\leq f(x)\}\] квадрируема и 
    \[\mu(A)=\int\limits_{a}^{b}f(x)\ dx\]
\end{theorem} 
\begin{proof}
    \[f(x)\in \mathcal{R}[a,b] \Rightarrow \forall \epsilon>0\ \exists\ \delta>0,\ \forall T: d(T)<\delta: \underline{\underline{S}}(T)-\overline{\overline{S}}(T)<\epsilon\]
\end{proof} 