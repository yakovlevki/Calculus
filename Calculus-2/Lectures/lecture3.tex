\subsection{Суммы Дарбу. Критерий Дарбу интегрируемости по Риману}
Далее рассматриваем разбиения $T^+$
\begin{definition}
    Пусть $T_{1}$ и $T_{2}$ - разбиения отрезка $[a,b]$ такие, что $T_1\subset T_2$. Тогда $T_2$ называется измельчением $T_1$.
\end{definition} 
\begin{definition}
    Пусть $f(x)$ ограничена на $[a,b],\ \{x_i\}_{i=1}^n=T$ - разбиение $[a,b]$
    \[m_i=\inf\limits_{[x_i, x_{i+1}]} f(x),\ M_i=\sup\limits_{[x_i, x_{i+1}]}f(x)\] 
    \[\overline{\overline{S}}_f(T)=\sum\limits_{i=0}^{n-1}m_i(x_{i+1}-x_i),\ \underline{\underline{S}}_f(T)=\sum\limits_{i=0}^{n-1}M_i(x_{i+1}-x_i)\]
    Тогда $\overline{\overline{S}}_f(T)$ называется нижней суммой Дарбу, а $\underline{\underline{S}}_f(T)$ верхней суммой Дарбу.
\end{definition} 
\begin{numlemma}
    Пусть $T_1$ - измельчение $T$. Тогда
    \[\overline{\overline{S}}(T)\leq \overline{\overline{S}}(T_1)\ \ \text{и}\ \ \underline{\underline{S}}(T)\geq \underline{\underline{S}}(T_1)\]
\end{numlemma} 
\begin{proof}
    Докажем для нижней суммы. Рассмотрим случай, когда\\
    $T_1=T\cup \{x_j'\},\ x_j'\in [x_j, x_{j+1}]$. Тогда сократятся все отрезки кроме $[x_j,x_{j+1}]:$
    \begin{multline*}
        \overline{\overline{S}}(T_1)-\overline{\overline{S}}(T)=m_{1j}(x_j'-x_j)+m_{2j}(x_{j+1}-x_j')-m_j(x_{j+1}-x_j)=\\
        =m_{1j}(x_j'-x_j)+m_{2j}(x_{j+1}-x_j')-m_j(x_j'-x_j)-m_j(x_{j+1}-x_j')\geq 0
    \end{multline*}
    значит, по индукции, это верно для любого измельчения.
\end{proof} 
\begin{numlemma}
    \[\forall\ T_1, T_2: \overline{\overline{S}}(T_1)\leq \underline{\underline{S}}(T_2)\]
\end{numlemma} 
\begin{proof}
    Рассмотрим объединение любых двух разбиений $T_1$ и $T_2$:\\ $T=T_1\cup T_2$. Тогда $T$ является измельчением и $T_1$ и $T_2$. Тогда по лемме 1 получаем:
    \[\overline{\overline{S}}(T_1)\leq \overline{\overline{S}}(T)\ \ \text{и}\ \ \underline{\underline{S}}(T)\leq \underline{\underline{S}}(T_2) \Rightarrow \overline{\overline{S}}(T_1)\leq \underline{\underline{S}}(T_2)\]
\end{proof} 
\begin{numlemma} $\forall\ T_{[a,b]}:$
    \[\overline{\overline{S}}(T)=\inf\limits_{\{\xi_i\}}\sum\limits_{i=1}^{n}f(\xi_i)(x_i-x_{i-1})\]
    \[\underline{\underline{S}}(T)=\sup\limits_{\{\xi_i\}}\sum\limits_{i=1}^{n}f(\xi_i)(x_i-x_{i-1})\]
\end{numlemma} 
\begin{proof}
    Докажем для верхней суммы. Рассмотрим множество \\
    $\{X_i: X_i\subset \R\}_{i=1}^n$ такое, что\\
    $X_i$ - ограничено $\forall i$, а также $\{a_i\}_{i=1}^n: a_i\geq 0 \ \forall i$ Тогда
    \[\forall \epsilon>0,\ \forall i=\{1,\dots,n\}\ \exists\ x_i\in X_i: x_i>\sup X_i-\epsilon\] 
    Значит 
    \[\sum\limits_{i=1}^{n}a_i x_i>\sum\limits_{i=1}^{n}a_i \sup X_i-\epsilon\cdot \sum\limits_{i=1}^{n}a_i\] 
    \[\sup\limits_{\{x_i\}} \sum\limits_{i=1}^{n}a_i x_i\geq \sum\limits_{i=1}^{n}a_i \sup X_i\]
    но 
    \[\sum\limits_{i=1}^{n}a_i x_i \leq \sum\limits_{i=1}^{n} a_i\sup X_i\]
    отсюда получаем:
    \[\sup\limits_{\{x_i\}} \sum\limits_{i=1}^{n}a_i x_i=\sum\limits_{i=1}^{n}a_i \sup X_i\]
\end{proof} 
\begin{theorem} (Критерий Дарбу интегрируемости по Риману)\\
    $f(x)\in \mathcal{R}[a,b] \Leftrightarrow f$ - ограничена и 
    \[\forall \epsilon>0\ \exists\ \delta_{\epsilon}>0,\ \forall\ T_{[a,b]}: d(T)<\delta_{\epsilon}: \underline{\underline{S_f}}(T)-\overline{\overline{S}}_f(T)<\epsilon\]
\end{theorem} 
\begin{proof} \ 
    \begin{itemize}
        \item[$(\Rightarrow):$]
        \[\exists\ I=\int\limits_{a}^{b}f(x)\ dx \Rightarrow \forall \epsilon>0\ \exists\ \delta_{\epsilon}>0,\ \forall T(\xi): d(T)<\delta_{\epsilon}\]
        \[I-\frac{\epsilon}{3}<\sigma_f(T(\xi))<I+\frac{\epsilon}{3}\]
        \[\vline\ \overline{\overline{S_f}}(T)-I\ \vline \leq \frac{\epsilon}{3},\ \ \  \vline\ \underline{\underline{S_f}}(T)-I\ \vline \leq \frac{\epsilon}{3}\]
        $\Rightarrow \underline{\underline{S}}(T)-\overline{\overline{S}}(T)<\epsilon$.
        \item[$(\Leftarrow):$]
        \begin{equation}
            \forall \epsilon>0\ \exists\ \delta_{\epsilon}>0,\ \forall T: d(T)<\delta_{\epsilon}: \underline{\underline{S}}(T)-\overline{\overline{S}}(T)<\epsilon
        \end{equation}
        из леммы 2 по аксиоме полноты: 
        \begin{equation}
            \exists\ I\in \R,\ \forall\ T_1, T_2: \overline{\overline{S}}(T)\leq I\leq\underline{\underline{S}}(T)
        \end{equation}
        из $(1)$ следует, что $I$ - единственно, но 
        \begin{equation}
            \forall\ T(\xi): \overline{\overline{S}}(T)\leq \sigma_f(T(\xi))\leq \underline{\underline{S}}(T)
        \end{equation}
         из (2) и (3) получаем:
        \[\vline\ \sigma_f(T(\xi))-I \ \vline<\epsilon\]    
    \end{itemize}
\end{proof} 
\subsection{Классы интегрируемых функций}
\begin{theorem}
    Если $f(x)\in \mathcal{C}[a,b]$, то $f(x)\in \mathcal{R}[a,b]$
\end{theorem} 
\begin{proof}
    $f(x)\in \mathcal{C}[a,b] \Rightarrow f(x)$ - равномерно непрерывна на $[a,b]$, т.е
    \[\forall \epsilon>0\ \exists\ \delta_{\epsilon}>0, \forall x_1, x_2\in [a,b]: |x_1-x_2|<\delta_{\epsilon}: |f(x_1)-f(x_2)|<\epsilon\]
    Пусть $T: d(T)<\delta$
    \[\underline{\underline{S}}(T)-\overline{\overline{S}}(T)=\sum\limits_{i=1}^{n}(M_i-m_i)(x_i-x_{i-1})=\sum\limits_{i=1}^{n}(f(x_{i_{max}})-f(x_{i_{min}}))(x_i-x_{i-1})<\epsilon(b-a)\]
\end{proof} 
\begin{theorem}
    Пусть $f(x)$ - монотонна на $[a,b]$. Тогда $f(x)\in \mathcal{R}[a,b]$
\end{theorem} 
\begin{proof}
    Докажем для неубывающей. Если $f(x)=const$, то очевидно.
    Пусть $d(T)<\frac{\epsilon}{f(b)-f(a)}$
    \begin{multline*}
        \underline{\underline{S}}(T)-\overline{\overline{S}}(T)=\sum\limits_{i=1}^{n}(f(x_i)-f(x_{i-1}))(x_i-x_{i-1})<\\
        <\sum\limits_{i=1}^{n}(f(x_i)-f(x_{i-1}))\cdot \frac{\epsilon}{f(b)-f(a)}=\\
        =\frac{\epsilon}{f(b)-f(a)}\cdot (f(b)-f(a))=\epsilon
    \end{multline*}
\end{proof} 
