\section{Лекция 2}
\subsection{Критерий Дарбу интегрируемости на брусе}
\begin{definition}
    Разность сумм Дарбу
    \begin{equation*}
        \Omega(f, T) = \underline{\underline{S}}(f, T) - \overline{\overline{s}}(f, T)
    \end{equation*}
    называется $\Omega$-суммой.
\end{definition}
\begin{comm}
    Пусть $T = \{\Delta_j\}$
    \[\Omega(f, T) = \sum \limits_i \left(\ \sup \limits_{x \in \Delta_i} f(x) - \inf \limits_{x \in \Delta_i} f(x)\right)\cdot |\Delta_i|\]
    $\sup \limits_{x \in \Delta_i} f(x) - \inf \limits_{x \in \Delta_i} f(x) := \omega_i$ называют колебаниями $f$ на $\Delta_i$.
\end{comm}
\noindentВ одномерном случае доказывалось утверждение
\begin{theorem}[Критерий Дарбу интегрируемости на отрезке]
    $ \\$Следующие условия эквивалентны:
    \begin{enumerate}
        \item $f \in \mathcal{R}[a, b]$
        \item $I^*(f, [a,b]) = I_*(f, [a,b])$
        \item $\forall \epsilon > 0 \ \exists\ \delta > 0: \forall T: d(T) < \delta: \Omega(f, T) < \epsilon$
        \item $\forall \epsilon > 0 \ \exists\ T: \Omega(f, T) < \epsilon$
    \end{enumerate}
\end{theorem}
\noindent Рассмотрим аналогичное утверждение и докажем его
\begin{theorem}[Критерий Дарбу интегрируемости на брусе]
    $ \\$Следующие условия эквивалентны:
    \begin{enumerate}
        \item $f \in \mathcal{R}(\Pi)$
        \item $I^*(f, \Pi) = I_*(f, \Pi)$
        \item $\forall \epsilon > 0 \ \exists\ \delta > 0: \forall T,\ d(T) < \delta: \Omega(f, T) < \epsilon$
        \item $\forall \epsilon > 0 \ \exists\ T: \Omega(f, T) < \epsilon$
    \end{enumerate}
\end{theorem}
\newpage
\begin{proof} Докажем по схеме $(1)\Rightarrow(2) \Rightarrow(4) \Rightarrow (3) \Rightarrow (1)$
    \begin{enumerate}
        \item $(1) \Rightarrow (2)$:
            \[\exists\ I\in \R: \forall \epsilon > 0\ \exists\ \delta > 0: \forall\ T, H,\ d(T) < \delta:|I - \sigma(f, T, H)| < \epsilon\]
            Знаем, что 
            \[\underline{\underline{S}}(f, T) = \sup\limits_H \sigma(f, T, H),\ \ \overline{\overline{s}}(f, T) = \inf\limits_H \sigma(f, T, H)\]
            тогда
            \[I - \epsilon < \sigma(f, T, H) < I + \epsilon\]
            отсюда
            \[I - \epsilon \leq \overline{\overline{s}}(f, T) < I + \epsilon, \ \ I - \epsilon < \underline{\underline{S}}(f, T) \leq I + \epsilon\]
            таким образом $\exists\ I\in \R: \forall \epsilon > 0 \ \exists\ \delta > 0: \forall T,\ d(T) < \delta$:
            \[|I - \overline{\overline{s}}(f, T)| < \epsilon,\ \ |I - \overline{\overline{s}}(f, T)|<\epsilon\]
            \[\lim \limits_{d(T) \rightarrow 0} \overline{\overline{s}}(f, T) = \lim \limits_{d(T) \rightarrow 0} \underline{\underline{S}}(f, T) = I\]
            \[\forall \epsilon>0\ \exists\ T,\ d(T)<\delta: \overline{\overline{s}}(f,T)\geq I-\epsilon \Rightarrow \sup\limits_{T}\overline{\overline{s}}(f,T) = I_*\geq I\]
            \[\forall \epsilon>0\ \exists\ T,\ d(T)<\delta: \underline{\underline{S}}(f,T)\leq I+\epsilon \Rightarrow \inf\limits_{T}\underline{\underline{S}}(f,T)=I^*\leq I\]
            При этом известно, что $I_*\leq I\leq I^*$. Таким образом, $I_*=I=I^*$
        \item $(2)\Rightarrow(4)$:
            \[\sup \limits_{T} \underline{\underline{S}}(f, T) = \inf \limits_{T} \overline{\overline{s}}(f, T)\]
            По свойству точных граней $\forall \epsilon > 0 \ \exists\ T_1, T_2$:
            \[0 \leq I - \overline{\overline{s}}(f, T_1) < \frac{\epsilon}{2},\ \  0 \leq \underline{\underline{S}}(f, T_2)-I < \frac{\epsilon}{2}\]
            \[0 \leq \underline{\underline{S}}(f, T_2) - \overline{\overline{s}}(f, T_1) < \epsilon\]
            Тогда если $T=T_1\cup T_2$, то 
            \[0\leq \underline{\underline{S}}(f,T)-\overline{\overline{s}}(f,T)\leq \underline{\underline{S}}(f,T_2)-\overline{\overline{s}}(f,T_1)<\epsilon\]
        \item $(4) \Rightarrow (3)$:
            \[\forall \epsilon>0\ \exists\ T_{\epsilon}: \Omega(f,T_{\epsilon})<\frac{\epsilon}{2}\]
            Пусть разбиение $T_1$ получено из разбиения $T$ добавлением $p$ разрезов. Тогда
            \[0\leq \underline{\underline{S}}(f,T)-\underline{\underline{S}}(f,T')\leq (M-m)\cdot d^{k-1}\cdot d(T)\cdot p\]
            \[0\leq \Omega(f,T_1)\leq \Omega(f,T_{\epsilon})<\frac{\epsilon}{2}\]
            \begin{multline*}
                0\leq \Omega(f,T)-\Omega(f,T_1)=(\underline{\underline{S}}(f,T)-\overline{\overline{s}}(f,T))-(\underline{\underline{S}}(f,T_1)-\overline{\overline{s}}(f,T_1))=\\
                =(\underline{\underline{S}}(f,T)-\underline{\underline{S}}(f,T_1))-(\overline{\overline{s}}(f,T_1)-\overline{\overline{s}}(f,T))\leq 2(M-m)\cdot d^{k-1}\cdot d(T)\cdot p\overset{(1)}{<}\frac{\epsilon}{2}
            \end{multline*}
            (1): при условии что мы берем 
            \[\delta:=\frac{\epsilon}{4(M-m)\cdot d^{k-1}\cdot p}>0\]
            Итак при $d(T)<\delta$:
            \[\Omega(f,T)=(\Omega(f,T)-\Omega(f,T_1))+\Omega(f,T_1)<\frac{\epsilon}{2}+\frac{\epsilon}{2}=\epsilon\]
        \item $(3) \Rightarrow (1)$:
        \[\forall \epsilon>0\ \exists\ \delta>0\ \forall T,\ d(T)<\delta: \underline{\underline{S}}(f,T)-\overline{\overline{s}}(f,T)<\epsilon\]
        Отсюда следует что выполнено условие $(2)$: $I_*=I^*$, а это означает, что
        \[\overline{\overline{s}}(f,T)\leq I\leq \underline{\underline{S}}(f,T)\]
        Также известно, что
        \[\overline{\overline{s}}(f,T)\leq \sigma(f,T,H)\leq \underline{\underline{S}}(f,T)\]
        объединяя эти два факта, получим, что
        \[|I-\sigma(f,T,H)|\leq \underline{\underline{S}}(f,T)-\overline{\overline{s}}(f,T)<\epsilon\]
    \end{enumerate}
\end{proof}
\begin{theorem}
    Пусть $\Pi$ --- замкнутый брус в $\R^k,\ f\in \mathcal{C}(\Pi)$. Тогда $f\in \mathcal{R}(\Pi)$.
\end{theorem}
\begin{proof}
    Пусть $f\in \mathcal{C}(\Pi) \Rightarrow f$ - равномерно непрерывна на $\Pi\\
    (f\in \mathcal{UC}(f))$:
    \[\forall \epsilon>0\ \exists\ \delta>0,\ \forall x_1,x_2,\ \rho(x_1,x_2)<\delta: |f(x_1)-f(x_2)|<\frac{\epsilon}{2|\Pi|}\]
    Хотим показать, что
    \[\Omega(f,T)=\sum\limits_{i}\omega_i\cdot |\delta_i|<\epsilon\]
    При этом 
    \[\omega_i=\sup\limits_{x_1\in \Delta_i}f(x_1)-\inf\limits_{x_2\in \Delta_i}f(x_2)\leq \frac{\epsilon}{2|\Pi|}\]
    таким образом
    \[\Omega(f,T)\leq \frac{\epsilon}{2|\Pi|}\sum\limits_{i}|\Delta_i|=\frac{\epsilon}{2}<\epsilon\]
\end{proof}

